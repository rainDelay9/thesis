\section{Introduction}\label{sec:intro}

The \emph{Byzantine agreement (BA)} problem~\cite{PSL80,LSP82} is arguably one of the most important problems in theoretical computer science. In a $t$-resilient, $n$-party BA, a set of $n$ parties, $t$ of which might be corrupted, wish to agree on a common output bit that is one of the honest parties' input bits. Byzantine agreement originally started as a central concept in distributed computing~\cite{PSL80,LSP82}, and soon after became an important ingredient in \emph{secure multiparty computation}~\cite{Yao82,GMW87}. In recent years BA has gained renewed interest due to its impact on crypto-currencies~\cite{Micali16,GHMVZ17,PS17}.



\subsection{Our Result}\label{sec:intro:ourResult}
In the following pages we will show several lower bounds for randomized Byzantine agreement protocols in various models, culminating in \cref{mainThm}.

\mnote{I don't know how to write this conjecture without defining a projection for strings}

\begin{conjecture}\label{mainConjecture}
	Let $\gamma > 0$, $s \in \Theta(n)$ and $C \subset \zo^{n\cdot k}$ such that for all but a negligible fraction of $S \in {\binom{[n]}{s}}$ it holds that $\size{C|^k_{S}} < \gamma \cdot 2^s$. Then, $\size{C} \in o(1)$.
\end{conjecture}
(See Preliminaries for the definition of $C|^k_{S}$)

\begin{theorem}\label{mainThm}
	Let $\Pi$ be an \batwo{n}{t}, where $t$ is any constant fraction of $n$, and let $\delta$ be a lower-bound for the probability that for any input all honest parties halt after two rounds, regardless of adversarial behavior. Then, if \cref{mainConjecture} holds, $\delta \in o(1)$. \mnote{Parties must commit to randomness?}
\end{theorem}

\subsection{Our Techniques}\label{sec:intro:technique}

\subsection{Additional Related Work}\label{sec:relatedWork}

\subsection{Open Questions}

\subsection*{Paper Organization}
