

\section{Lower Bounds on First-Round Halting}\label{sec:FirstRound}

In this section, we present our lower bound for the probability of first-round halting in Byzantine agreement protocols.
\begin{theorem}[Bound on first-round halting. \cref{thm:FirstRound:Arb} restated]\label{thm:FirstRound:Arb:Res}
\ThmFirstRoundArb
\end{theorem}

Let $\Pi$ be as in \cref{thm:FirstRound:Arb:Res}. We assume for ease of notation that an honest party that runs more than one round outputs $\perp$ (it will be clear that the attack, described  below, does not benefit from this change). Finally, we omit the security parameter from the parties' input list, it will be clear though that the adversaries we present are efficient \wrt the security parameter.


\begin{lemma}[Neighboring executions]\label{lemma:FirstRound:Arb}
	Let $\vv,\vv' \in \zn$ be with $\ham(\vv,\vv') \le t$. Then for both $\o\in \zo$:
	\[
	\pr{\Pi(\vv') \in  \omin^n  \setminus \mon} \ge	\pr{\Pi(\vv) \in \omin^n} - (1- \gamma) - 4\alpha - \err.
	\]
\end{lemma}
Namely, the lemma bounds from below the probability that in a random honest execution of the protocol on input $\vv'$, at least one party halts in the first round while outputting $\o$.

We prove \cref{lemma:FirstRound:Arb} below, but first use it to prove \cref{thm:FirstRound:Arb:Res}. We also make use of the following immediate observation.
\begin{claim}[Supermajority execution]\label{claim:FirstRoundound:Arb:Validity}
Let $\vv \in \zn$  and $\o \in \zo$ be such that $\ham(\vv,\o^n) \le t$. Then,
$\pr{\Pi(\vv) \in \omin^n} \ge 1 - \alpha - \beta$.
\end{claim}
\begin{proof}
Let $\cA \subset [n]$ be a subset of size $n-t$ such that $\vv_\cA = \o^{\size{\cA}}$. The claimed validity of $\Pi$ yields that
\begin{align*}
\pr{\Pi(\vv)_\cA \notin \omin^{\size{\cA}}} < \beta.
\end{align*}
This follows from $\beta$-validity of $\Pi$ and the fact that an honest party cannot distinguish between an execution of $\Pi(\vv)$ and an execution of $\Pi(b^n)$ in which all parties not in $\cA$ act as if their input bit is as in $\vv$. Hence, by the claimed agreement of $\Pi$

\begin{align*}
\pr{\Pi(\vv) \notin \omin^n} < \alpha + \beta.
\end{align*}
\end{proof}




\begin{proof}[Proof of \cref{thm:FirstRound:Arb:Res}]~
We separately prove the theorem  for $t \ge n/3$  and for $t \ge n/4$.

\paragraph{The case $t \ge n/3$.}
We assume for simplicity that $(n-t)/2\in \N$, let $\vz= 0^t 1^{\cnt} 0^{\fnt} $ and let $\vo = 1^t 1^{\cnt} 0^{\fnt}$. Note that $\ham(\vz,\vo) = t$, and that  for both $\o\in \zo$ it holds that $\ham(\vv_\o,\o^n)\le t$. \Inote{the proof if $(n-t/2) \notin \N$ is a bit tedious. Matan, please verify} Hence, by \cref{claim:FirstRoundound:Arb:Validity},
%validity (\cref{claim:FirstRoundound:Arb:Validity}),
for both $\o\in \zo$:
\begin{align}
\pr{\Pi(\vv_\o) \in \omin^n} \ge 1 - \alpha - \beta.
\end{align}
Applying  \cref{lemma:FirstRound:Arb}  to $\vv= \vz$ and $\vv'= \vo$ yields that
	\begin{align*}
\pr{\Pi(\vo) \in \set{0,\perp}^n \setminus \mon} \ge 1 - 5\alpha - \beta - (1- \gamma) - \err,
\end{align*}
yielding that  $5\alpha + 2\beta + (1-\gamma)  + \err\ge 1$.

\paragraph{The case $t \ge n/4$.}  In this case there are no two vectors that are $t$ apart in Hamming distance, and still each of them has $n-t$ entries of opposite values. Rather, we consider the two vectors $\vz= 0^t 0^t 0^t 1^{n-3t}$ and  $\vo= 1^t 1^t 0^t 1^{n-3t}$ of distance $2t$. For both $\o\in \zo$, the vector $\vv_\o$ has at least $n-t$ entries with $\o$ and is of distance $t$ from the vector $\vstar =1^t 0^t 0^t  1^{n-3t}$.
%\rnote{as a result, for every $b\in\zo$, $\pr{\Pi(\vstar) \in \omin^n \setminus \mon}\leq 1/2$ \Inote{I dont think it is needed, but add it if you like}}

As in the first part of the proof, Applying  \cref{claim:FirstRoundound:Arb:Validity,lemma:FirstRound:Arb}  on $\vv_b$ and $\vstar$, for both $\vo\in \zo$, yields that
\begin{align*}
\pr{\Pi(\vstar) \in \omin^n \setminus \mon} \ge 1 - 5\alpha - \beta - (1- \gamma) - \err,
\end{align*}
yielding that $2(5\alpha + \beta + (1- \gamma) + \err)  \ge 1$.
\end{proof}
	
	
\newcommand{\PPf}{\Pi^\cP}	
\subsection{Proving Lemma~\ref{lemma:FirstRound:Arb}}
\begin{proof}[Proof of \cref{lemma:FirstRound:Arb}]
Fix $b \in \zo$ and let $\delta = \pr{\Pi(\vv) \in \omin^n}$.  Let $\cP$ be the coordinates in which $\vv$ and $\vv'$ differ, and let $\oP = n \setminus \cP$. Let $I$ be the index (a function of the parties' coins and setup parameters) of the smallest party in  $\oP$  that halts in the first round   and outputs the same value, both if the parties in $\cP$ send their messages according to input   $\vv$  and if they do that according to $\vv'$. We let  $I=0$  if there is no such party, and (abusing notation)  sometimes identify  $I$  with the event that $I\neq 0$, \eg $\pr{I}$ stands for $\pr{I\ne 0}$. \mnote{Huh?\Inote{better?}}  Clearly,


%and let $\gamma'$ be the probability that $\Pc_1$ halts in the first round of $\Pi(\vv)$. Recall that by our convention, an honest  party outputs value in $\zo$ iff it halts in the first round.

\begin{align*}
\delta \le  \pr{\Pi(\vv) \in  \omin^n \qand  I}+ (1- \pr{I})
\end{align*}
%Since $\gamma' \ge \gamma$,
and thus
\begin{align}\label{eq:FirstRound:Arb:1}
\pr{\Pi(\vv) \in  \omin^n \qand   I} \ge \delta - (1-\pr{I}).
\end{align}

	
 It follows that
\begin{align}\label{eq:FirstRound:Arb:2}
\pr{\Pi(\vv')  \in \omin^n \setminus \mon} &\ge  \pr{\Pi(\vv') \in  \omin^n \qand   I}\\
&=\pr{\Pi(\vv') \in  \omin^n \qand   \Pi(\vv')_I= b}\nonumber\\
&\ge \pr{\Pi(\vv')_I= b} -  \alpha\nonumber\\
&= \pr{\Pi(\vv)_I  =  \o} -  \alpha\nonumber\\
&\ge \pr{\Pi(\vv) \in  \omin^n \land   \Pi(\vv)_I  =  \o} - 2\alpha\nonumber\\
&= \pr{\Pi(\vv) \in  \omin^n \land   I} - 2\alpha\nonumber\\
&\ge \delta - (1-\pr{I}) - 2\alpha.\nonumber
\end{align}
The  first inequality and the equalities hold by the definition of $I$.  The second  and third  inequalities  hold by agreement, and  the last inequality holds by  \cref{eq:FirstRound:Arb:1}. We conclude the proof showing that:
\begin{align}\label{claim:FirstRound:Arb}
	\pr{I} \ge \gamma - \err  - 2\alpha.
\end{align}

  Let $E_h$ be the event that  each party  in $\oP$ either does not halt when the parties in $\cP$ act according to $\vv$ or does not halt when they act according to $\vv'$. Let $E_a$ be the event that $E_h$ does not occur, but $I = 0$  (\ie the parties that halt in the first round, output different values according the $\cP$ input.  Clearly $I = 0  \Longleftrightarrow E_h \lor  E_a$.

Consider the  adversary  that in the first round acts toward a random subset of $\cP$ according to input $\vv$ and towards the remaining parties according to  $\vv'$, and  aborts at the end of this  round. It is  clear that  if $E_a$ occurs, the  above adversary violates agreement with probability $1/2$. Thus, $\pr{E_a} \le 2\alpha$.

It is also clear that when $E_h$ occurs, the above attacker  fails to prevent an honest party
%parties
in $\oP$ from halting in the first round only if the following event happens:  each party in $\oP $ does not halt in $\Pi(\vv'')$ for some $\vv'' \in \set{\vv,\vv'}$, but the adversary acts towards each of these parties on the input in which it does halt. The latter event happen with probability at most  $2^{-\size{\oP}} \le \FirstErr = \err$. Thus, $\pr{E_h} \le 1 - \gamma - \err$. We conclude that
\begin{align*}
	\pr{I} \ge  1 - \pr{E_h} - \pr{E_a} \ge \gamma - \err  - 2\alpha.
\end{align*}
Finally, we note that if the protocol has public randomness, the (now rushing) attacker does not have to guess what input to act upon.
%on.
Rather, after seeing the first round randomness, it  \emph{finds}  an input $\vv'' \in\set{\vv,\vv'}$ such that at least one party in $\oP$ does not halt in $\Pi(\vv'')$ or violates agreement, and acts according to this input. Hence, the bound on $I$ changes to
\begin{align*}
\pr{I} \ge \gamma - \alpha,
\end{align*}
proving the theorem statement for such protocols.
\end{proof}
