\newcommand{\DF}{D_\F}
\newcommand{\SDF}{\Supp(D_\F)}


\section{Lower Bounds on Second-Round Halting}\label{sec:SecondRound}
In this section, we prove lower bounds for second-round halting of Byzantine agreement protocols. In \cref{sec:TwoRoundProtcol:Arbitrary}, we prove a bound for arbitrary protocols, and in \cref{sec:TwoRoundProtcol:PR}, we give a much stronger bound for public-randomness protocols (the natural extension of public-coin protocols to the 'with-input' setting).

\subsection{Arbitrary Protocols}\label{sec:TwoRoundProtcol:Arbitrary}
In this section, we prove  our lower bound for  second-round halting of arbitrary protocols.


\begin{theorem}[Bound on second-round halting, arbitrary protocols.  \cref{thm:SecondRound:Arb} restated]\label{thm:SecondRound:Arb:Res}
\ThmSecondRoundArb
\end{theorem}

Let $\Pi$ be as in \cref{thm:SecondRound:Arb:Res}. We assume for ease of notation that an honest party that runs more than two rounds outputs $\perp$ (it will be clear that the attack, described  below, does not benefit from this change). We also assume \wlg that the honest parties in an execution of $\Pi$ never halt in one round (by adding a dummy round if needed). Finally, we omit the security parameter from the parties' input list, it will be clear though that the adversaries we present are efficient \wrt the security parameter.

Let $k =\ceil{n/4}$ and let $h = \ceil{(n-k)/(t-k)}$. The theorem is easily implied by the next lemma.

\begin{lemma}[Neighboring executions]\label{lemma:SecondRound:Arb}
	Let $\vv,\vv' \in \zn$ be with $\ham(\vv,\vv') \le k$. Then for every $\o\in \zo$:
\[
\pr{\Pi(\vv') = \o^n} \ge \pr{\Pi(\vv) = \o^n} - h(h+1)(2\alpha + 1-\gamma) - \alpha.
\]
%\rnote{$\o$ is too similar to $0$, maybe use a different notation\Inote{We also use it in the PR proof. Lets stick to that..}}
%Then $\eex{\cS \gets \bns}{I_{\vf,\cS}(i,b)} \ge 1 - \frac{i+1}{4h+8} - \chi$ for all $ i \in (h)$.
\end{lemma}
Namely, the lemma bounds from below the probability that in a random honest execution of the protocol on input $\vv'$ all parties halt within two rounds while outputting $\o$.


We prove \cref{lemma:SecondRound:Arb} below, but first use it to prove \cref{thm:SecondRound:Arb:Res}. We also make use of the following immediate observation.
\begin{claim}[Supermajority execution]\label{claim:SecondRound:Arb:Validity}
	Let $\vv \in \zn$  and $\o \in \zo$ be such that $\ham(\vv,\o^n) \le t$. Then,
	$\pr{\Pi(\vv) = \o^n } \ge 1 - \alpha - \beta - (1- \gamma)$.
\end{claim}
\begin{proof}
	The same argument as in the proof of \cref{claim:FirstRoundound:Arb:Validity} yields that
	\begin{align*}
	\pr{\Pi(\vv) \notin \omin^n} < \alpha + \beta.
	\end{align*}
	Thus, by $\gamma$-second-round halting
	\begin{align*}
	\pr{\Pi(\vv) \ne \o^n} < \alpha + \beta + (1-\gamma).
	\end{align*}
\end{proof}
\begin{proof}[Proof of \cref{thm:SecondRound:Arb:Res}]
Consider the  vectors $\vz= 0^k 0^k 0^k 1^{n-3k}$, $\vo= 1^k 1^k 0^k 1^{n-3k}$ and $\vstar =1^k 0^k 0^k 1^{n-3k}$. Note that for both $\o\in \zo$   it holds that $\ham(\vv_b,\o^n)\le t$  (since $n/4 \leq k\leq t$),  and that $\ham(\vv_b,\vstar)=k$. Applying \cref{lemma:SecondRound:Arb,claim:SecondRound:Arb:Validity} for each of these vectors, yields that for both $\o\in \zo$:
\begin{align*}
\pr{\Pi(\vstar) = \o^n} &\ge 1- \alpha - \beta -(1- \gamma) - h(h+1)(2\alpha + 1-\gamma) -\alpha\\
&\ge 1- \beta - (h+1)^2(2\alpha + 1-\gamma).
\end{align*}
Note that $w=h+1$, which implies $\beta +w^2(2\alpha + 1-\gamma) \ge 1/2$, and the proof follows by a simple calculation.
\end{proof}

\subsubsection{Proving \cref{lemma:SecondRound:Arb}}
We assume for ease of notation that $\ham(\vv,\vv')=k$  (and not $\le k$). Let $\ell= \floor{\ept n}$  (hence, $h = \ceil{(n-k)/\ell}$). Assume for ease of notation that $h\cdot \ell = n-k $ (\ie no rounding), and for a $k$-sized subset of parties $\cP \subset [n]$, let $ \cL^\cP_1,\dots,\cL_h^\cP$ be an arbitrary partition of $\oP = [n] \setminus \cP$ into $\ell$-sized subsets. Consider the following family of protocols:


\newcommand{\PP}{\Pi^\cP}

{ \samepage
\begin{protocol}[$\PP_{d}$]\label{prot:main:Arb} ~
\begin{description}
	\item [Parameters:] A subset $\cP\subseteq [n]$ and an index $d\in (h)$.
	\item [Input:] Every party $\Pc_i$ has an input bit $v_i\in\zo$.
	
	\item [First round:] ~
	\begin{description}
		\item[Party $\Pc_i \in \cP$.]
		If $d=0$ [\resp $d=h$], act honestly according to $\Pi$ \wrt input bit~$v_i$ [\resp $1-v_i$].
		Otherwise,		
		\begin{enumerate}
			\item Choose random coins honestly (\ie uniformly at random).
			
			\item To each party in $\bigcup_{j \in \set{1,\ldots,d}} \cL_j^\cP$: send a message according to input $1-v_i$.
			
			\item To each party in $\bigcup_{j \in \set{d+1 ,\ldots, h}} \cL_j^\cP$: send a message according to input $v_i$ (real input).
			
			\item Send no messages to the other parties in $\cP$.
		\end{enumerate}
		
		\item[Other parties.] Act according to $\Pi$.
	\end{description}
	
	\item [Second round:]~
	
	\begin{description}
		\item[Party $\Pc_i \in \cP$.]
        If $d=0$ [\resp $d=h$], act honestly according to $\Pi$ \wrt input bit~$v_i$ [\resp $1-v_i$]; otherwise, abort.
		
		\item[Other parties.] Act honestly according to $\Pi$.
	\end{description}
\end{description}
\end{protocol}
}

Namely, the ``pivot'' parties in $\cP$ gradually shift their inputs from their real input to its negation according to parameter $d$. Note that protocol $\PP_{0}(\vv)$ is equivalent to an honest execution of protocol $\Pi(\vv)$, and $\PP_{h}(\vv)$ is equivalent to an honest execution of $\Pi(\vv')$, for $\vv'$ being $\vv$ with the coordinates in $\cP$ negated. \cref{lemma:SecondRound:Arb} easily follows by the next claim about \cref{prot:main:Arb}. In the following we let $\delta = \pr{\Pi(\vv)_\oP = \o^{\size{\oP}}}$.

\newcommand{\ds}{{d^\ast}}
\begin{claim}\label{claim:SecondRound:Arb}
For any $k$-sized subset $\cP\subset [n]$ and $d\in (h)$ it holds that
\[
\pr{\PP_d(\vv)_\oP = \o^{\size{\oP}}} \ge \delta- d (h+1)(2\alpha + 1-\gamma).
\]
\end{claim}
We prove \cref{claim:SecondRound:Arb} below, but first use it to prove \cref{lemma:SecondRound:Arb}.
 \begin{proof}[Proof of \cref{lemma:SecondRound:Arb}]
	By \cref{claim:SecondRound:Arb}
	\begin{align*}
	\pr{\PP_h(\vv)_\oP = \o^{\size{\oP}}} \ge \delta - h(h+1)(2\alpha + 1-\gamma).
	\end{align*}
	Since $\PP_h(\vv)$ is just an honest execution of $\Pi(\vv')$, by agreement
	\begin{align*}
	\pr{\Pi(\vv') = \o^n} &\ge \delta - h(h+1)(2\alpha + 1-\gamma) - \alpha.
	\end{align*}
\end{proof}




\begin{proof}[Proof of \cref{claim:SecondRound:Arb}]
The proof is by induction on $d$. The base case $d=0$ holds by definition. Suppose for contradiction the claim does not hold, and let $\ds\in(h-1)$
%$\ds\ge 0$
be such that the claim holds for $\ds$ but not for $\ds+1$. Let $\gamma_d$ be the probability that all honest parties halt in the second round of a random execution of $\PP_d(\vv)$. The assumption about $\ds$ yields that
\begin{align}\label{eq:main:Arb:01}
\pr{\PP_\ds(\vv)_\oP = \o^{\size{\oP}}} \ge \delta- \beta- \ds (h+1)(2\alpha + 1-\gamma)
\end{align}

and
\begin{align}\label{eq:main:Arb:02}
\pr{\PP_{\ds+1}(\vv)_\oP \in \zo^{\size{\oP}} \setminus \sset{\o^{\size{\oP}}}} > 1-\left(\delta - \beta - (\ds+1) (h+1)(2\alpha + 1-\gamma)\right)- (1-\gamma_d)
\end{align}


We note that for every $d\in (h)$
\begin{align}\label{eq:main:Arb:03}
\frac{1- \gamma_d}{h+1} \le 1-\gamma
\end{align}


Indeed, otherwise, the adversary that corrupts the parties in $\cP$ and acts like $\PP_d$ for a random $d\in (h)$, violates the $\gamma$-second-round-halting property of $\Pi$. (The reader is reminded that we denote by $(h)$ the set $\sset{0,1,\ldots,h}$) We conclude that
\begin{align}\label{eq:main:Arb:1}
\lefteqn{\ppr{\vr}{\PP_\ds(\vv;\vr)_\oP = \o^{\size{\oP}} \qand \PP_{\ds+1}(\vv;\vr)_\oP \in \zo^{\size{\oP}} \setminus \sset{\o^{\size{\oP}}}}}\\
& \ge 1 - \left(1- \ppr{\vr}{\PP_\ds(\vv;\vr)_\oP = \o^{\size{\oP}}} \right) - \left(1- \ppr{\vr}{\PP_{\ds+1}(\vv;\vr)_\oP \in (\zo^{\size{\oP}} \setminus \sset{\o^{\size{\oP}}}}\right)\nonumber\\
& > (h+1) (2\alpha + 1 - \gamma) - (1-\gamma_d) \nonumber\\
& \ge (h+1) 2\alpha.\nonumber
\end{align}
for $\vr$ being the randomness of the parties. The first inequality is by \cref{eq:main:Arb:01,eq:main:Arb:02}, and the second one by \cref{eq:main:Arb:03}.

Consider the adversary that samples $d\gets (h)$, corrupts the parties in $\cP \cup \cL^\cP_{d+1}$, and acts towards a uniform random subset of the honest parties according to $\PP_d$ and to the remaining parties according to $\PP_{d+1}$. \cref{eq:main:Arb:1} yields that the above adversary causes disagreement with probability larger than $(h+1) 2\alpha/2(h+1) = \alpha$. Since it corrupts at most $t$ parties, this contradicts the assumption about $\Pi$.
%\rnote{does it need to be a random subset? \Inote{Yes, so with probability at least 1/2 two parties output different values} the size doesn't matter, right? \Inote{should be uniform}}
\end{proof}




\subsection{Public-Randomness Protocols}\label{sec:TwoRoundProtcol:PR}
In this section, we prove  our lower bound for  second-round halting of public-randomness  protocols.


\begin{theorem}[Lower bound on second-round halting, public-randomness protocols. \cref{thm:SecondRound:PR} restated] \label{thm:SecondRound:PR:Res}
	\ThmSeconRoundPR
\end{theorem}

Assume \cref{con:IsoBot} holds. Let $\Pi$ be as in the theorem statement, and assume $\gamma= \epg$ in the case $t \ge (1/3 +\ept)n$  and $\gamma=  \half +  \epg$ in the case  $t \ge (1/4 +\ept)n$. We prove that in both cases $\alpha$, the agreement error,  has to be larger than some universal constant depending only  on $\ept$ and $\epg$.


Let  $\lambda =\epg /10$ and $\sigma= \ept/4$. Recall that $\bot_\cS(x)$ is defined by $
\bot_\cS(x)_i=
\begin{cases}
\bot, & i\in \cs,\\
x_i, & \text{otherwise}.
\end{cases}
$.  \cref{con:IsoBot} yields that there exists $\delta>0$ such that the following holds for large enough $n$: let $\Sigma$ be a finite alphabet and let $\cA_0,\cA_1 \subset \sbn$  be  two  sets such that for both $b\in \zo$:

\begin{align*}%\label{eq:1}
\ppr{\cs\gets \bns}{\ppr{\vr \gets \sn}{\vr,\bot_{\cS}(\vr) \in \cA_b} \ge  \lambda } \ge 1-\delta.
\end{align*}

Then,
\begin{align}\label{eq:IsoBot}
\ppr{\vr\gets \sn,\cS \gets \bns}{\forall b\in \zo\colon  \set{\vr,\bot_{\cS}(\vr)}  \cap \cA_b \neq \emptyset} \ge  \delta
\end{align}


We assume that  $\alpha = \min\set{\delta\lambda \ept/10, \beta = \lambda^2/2}$  and derive a contradiction.   Fix  $n$  that is large enough for  \cref{eq:IsoBot} to hold and that $\ppr{\cs \gets \bns}{\size{\cs} > 2\sigma n} = 2^{- \Theta(n\cdot \sigma^2)} \le \alpha$, \ie $n> \Theta(\log 1/\alpha)/\sigma^2)$.


As in the proof of \cref{thm:SecondRound:Arb:Res}, we assume for ease of notation that an honest party that runs more than two round outputs $\perp$, and that the honest parties in $\Pi$ never halt in one round. We also  omit the security parameter from the parties input list. We  assume \wlg that in the first round, the party flips no coin, since such coins can be added to the setup parameter.

We use the following notation: the setup parameter and second round randomness of the parties in $\Pi$ are identified with elements of $\cF$ and $\cR$, respectively. We denote by $f_i$ and $r_i$ the setup parameter and the second part randomness of party $\Pc_i$ in $\Pi$, and let $\DF$ be the joint distribution of the parties setup parameters (by definition, the joint distribution of the second round randomness is the product distribution $\rn$). For $\vv \in \zn$, $\vf=(f_1,\ldots,f_n) \in \SDF$ and $\vr=(r_1,\ldots,r_n) \in\rn$, let $\Pi(\vv;(\vf,\vr))$ denote the execution of $\Pi$ in which party $\Pc_i$ gets input $v_i$, setup parameter $f_i$ and second-round randomness $r_i$. We naturally apply this notation for the variants of $\Pi$ considered in the proof.



For $\cs\subseteq[n]$, let $\Pi^\cs$ be the variant of $\Pi$ in which the parties in $\cs$ halt at the end of the first round.   Let $k = \ceil{{t-\ept \cdot n}}$ (\ie $k = \ceil{n/3}$ if $t\ge (1/3 + \ept)n$, and   $k = \ceil{n/4}$ if $t\ge (1/4 + \ept)n$).    The heart of the proof lies in the following  lemma.

\begin{lemma}[Neighboring executions]\label{lemma:SecondRound:PR}
Let $\vv,\vv' \in \zn$ be with $\ham(\vv,\vv') \le k$, let  $\o\in \zo$ and let $\oS = [n] \setminus \cs$. Then  with probability at least $\gamma- 7\lambda- \frac{\alpha +  \pr{\Pi(\vv) \ne \o^n}}{\lambda}$ over $\vf \gets \DF$, it holds that
\begin{align*}
\ppr{\cs \gets \bns}{\ppr{\vr \gets \rn}{\Pi(\vv';(\vf,\vr)) = \o^n \qand \Pi^\cs(\vv';(\vf,\vr))_\oS = \o^{\size{\oS}} } \ge \lambda}\ge 1-\delta
\end{align*}
\end{lemma}
Namely, in an execution of $\Pi(\vv')$, all honest parties halt after two rounds and output $\o$, regardless of whether a  random subset of parties  aborts after the first round. \cref{lemma:SecondRound:PR} is proven in \cref{sec:lemma:SecondRound:PR}, but let us first use it to prove \cref{thm:SecondRound:PR}. We make use of the following immediate observation:
\begin{claim}[Supermajority execution]\label{claim:SecondRound:PR:Validity}
Let $\vv \in \zn$  and $\o \in \zo$ be such that $\ham(\vv,\o^n) \le t$. Then
	$\pr{\Pi(\vv) \in  \set{\o,\perp}^n } \ge  1 -   \alpha - \beta $.
\end{claim}
\begin{proof}
The proof of this claim uses an identical argument as in the proof of \cref{claim:FirstRoundound:Arb:Validity}.	
\remove{
Let $\cA \subset [n]$ be a subset of size $n-t$  such that $\vv_\cA = \o^{\size{\cA}}$. The claimed validity f  $\Pi$ yields that
\begin{align*}
\ppr{\vr}{\Pi(\vv;(\vf,\vr))_\cA \notin\omin^{\size{\cA}}} < \beta_\vf(\vv).
\end{align*}
Hence	by the claimed agreement,
\begin{align*}
\ppr{\vr}{\Pi(\vv;(\vf,\vr)) \notin  \omin^n} < \alpha_\vf(\vv) + \beta_\vf(\vv).
\end{align*}
}
%Finally  by the claimed  second0round halting
%\begin{align*}
%\ppr{\vr}{\Pi(\vv;(\vf,\vr)) \neq  \o^n} < \alpha_\vf(\vv) + \beta_\vf(\vv) + (1-\beta_\vf(\vv)).
%\end{align*}
\end{proof}


\paragraph{Proving \cref{thm:SecondRound:PR}.}
\begin{proof}[Proof of \cref{thm:SecondRound:PR}]~
We separately prove the case $t \ge (1/3 + \ept)n$ and $t \ge (1/4 + \ept)n$.



\paragraph{The case $t \ge (1/3 + \ept)n$.}
Let $\vz= 0^k 1^{\cnk} 0^{\fnk} $ and let $\vo= 1^k 1^{\cnk} 0^{\fnk} $. Note  that $\ham(\vz,\vo)=k$ and that for both $\o\in \zo$ it holds that $\ham(\vv_b,\o^n) \le t$ \Inote{Matan, please verify carefully}.  We will use  \cref{lemma:SecondRound:PR,claim:SecondRound:PR:Validity}  to prove that  $\Pi(\vv_1) = 0^n$ with  noticeable probability, contradicting the validity of the protocol.

Recall that $\lambda = \epg/8$ and $\alpha,\beta \le \lambda^2/2$.    \cref{claim:SecondRound:PR:Validity} yields that for both $\o\in\zo$:
\begin{align}\label{eq:SecondRound:PR:1}
\pr{\Pi(\vv_\o) \in  \set{\o,\perp}^n} \ge 1- \alpha - \beta \ge 1- \lambda^2
\end{align}
Applying \cref{lemma:SecondRound:PR}  \wrt $\vv_0$, yields  that  with probability  at least $ \lambda$ over  $\vf \gets \DF $:
\begin{align*}
\ppr{\vr}{\Pi(\vo;(\vf,\vr)) = 0^n}\ge \lambda
\end{align*}

and therefore
\begin{align*}
\pr{\Pi(\vo) = 0^n}\ge  2 \lambda^2
\end{align*}
in contradiction to \cref{eq:SecondRound:PR:1}.



\paragraph{The case $t \ge (1/4 + \ept)n$.} Consider the  vectors $\vz= 0^k 0^k 0^k 1^{n-3k}$,  $\vo= 1^k 1^k 0^k 1^{n-3k}$ and $\vstar =1^k 0^k 0^k  1^{n-3k}$. Note that for both $\o\in \zo$ it holds that $\ham(\vv_\o,\o^n) \le t$ and that $\ham(\vv_\o,\vstar)=k$ \Inote{Matan...}. Applying  \cref{lemma:SecondRound:PR,claim:SecondRound:PR:Validity}  on $\vv_b$ and $\vstar$, for both $\o\in \zo$, yields that  $\Pi^\cs(\vstar) = \o^n$ with  noticeable probability. This will allow us to use \cref{con:IsoBot} to lowerbound the protocol's agreement.

A similar  calculation to that done in the  $t\ge (1/3 + \epg)n$ part  yields that by \cref{lemma:SecondRound:PR,claim:SecondRound:PR:Validity},  for both $\o\in \zo$:  with probability at least  $\frac12 + 2\lambda$ over  $\vf \gets \DF $ it holds that
\begin{align*}
\ppr{\cs \gets \bns}{\ppr{\vr \gets \rn}{\Pi(\vstar;(\vf,\vr)) = \o^n \qand \Pi^\cs(\vstar;(\vf,\vr))_\oS = \o^{\size{\oS}} } \ge  \lambda}\ge 1-\delta
\end{align*}
It follows that there exists a set $\cT \subseteq \SDF$ with $\ppr{\vf \gets \DF}{\cT}\ge 4\lambda$, such that  for every $\vf\in \cT$, for \emph{both} $\o\in \zo$:
\begin{align}\label{eq:mainPR:4}
\ppr{\cs \gets \bns}{\ppr{\vr \gets \rn}{\Pi(\vstar;(\vf,\vr)) = \o^n \qand \Pi^\cs(\vstar;(\vf,\vr))_\oS = \o^{\size{\oS}} } \ge \lambda}\ge 1-\delta
\end{align}

We  assume \wlg that if a party gets $\perp$ as its second-round random coins, it aborts after the first round. For $\vr \in \rbot$, let $ \cE(\vr)$ be the indices of the $\bot$'s in $\vr$. For $\vf \in \SDF$ and $\o \in \zo$, let
\begin{align}
\cA^\vf_\o = \set{\vr \in \rbot \colon \Pi(\vstar;(\vf,\vr))_{\overline{\cE(\vr)}} = \o^{\size{\overline{\cE(\vr)}}}}
\end{align}
By \cref{eq:mainPR:4},  for $\vf \in \cT$ and $\o\in \zo$, it holds that
\begin{align}
\ppr{\cs\gets \bns}{\ppr{\vr \gets \rn}{\vr,\bot_{\cS}(\vr) \in \cA^\vf_\o} \ge\lambda} \ge 1-\delta
\end{align}
Hence by \cref{con:IsoBot},  see  \cref{eq:IsoBot}, for  $\vf \in \cT$ it holds that
\begin{align*}
\ppr{\vr\gets \rn,\cS \gets \bns}{\forall \o \in \zo\colon   \set{\vr,\bot_{\cS}(\vr)}\cap \cA_b \neq \emptyset} > \delta
\end{align*}
That is,
\begin{align}\label{eq:PR:3}
\ppr{\vr\gets \cR^n,\cS \gets \bns}{\forall \o\in \zo  \quad \exists \cs_\o \in \set{\cs,\emptyset} \colon  \Pi^{\cS_\o}(\vstar;(\vf,\vr))_{\overline{\cS_\o}}= \o^{ \size{\overline{\cS_\o}}}} >\delta
\end{align}

{\samepage
	\noindent
	Consider the following adversary:
	
	\begin{algorithm}[$\Ac$]~
		
		\begin{description}
			\item[Pre-interaction.]
			
			Corrupt a random subset $\cS \gets \bns $ conditioned on $ \size{\cs} \le 2\sigma n$.
			
			\item[First round.] Act according to $\Pi$.
			
			\item[Second round.] Sample  $\cs_0,\cs_1$ at random from $\set{\emptyset,\cs}$, and act towards some honest parties according to $\Pi^{\cs_0}$  and towards the others according to  $\Pi^{\cs_1}$ .
		\end{description}
	\end{algorithm}
}
By \cref{eq:PR:3}, the above adversary violates the agreement of $\Pi$ on input $\vstar$ with probability larger than $\delta - \ppr{\cs \gets \bns}{\size{\cs} >2\sigma n} >\alpha$, in contradiction with the assumed agreement of $\Pi$.
\end{proof}


\newcommand{\VV}{\cV^{\cP}}
\renewcommand{\PP}{\Pi^{\cP,\cS}}
\newcommand{\ops}{{\overline{\cP \cup \cS}}}

\subsubsection{Proving \cref{lemma:SecondRound:PR}}\label{sec:lemma:SecondRound:PR}
We  assume for simplicity that $\ham(\vv,\vv')=k$ (and not $\le k$). Let $\ell= \floor{(t -k)/2}$ and let $h = \ceil{(n-k)/\ell}$.  Assume for ease of notation that $h\cdot \ell = n-k $ (\ie no rounding), and for  $k$-size subset of parties $\cP \subset [n]$, let $ \cL^\cP_1,\dots,\cL_h^\cP$ be an arbitrary  partition of $\oP = [n] \setminus \cP$ into $\ell$-size subsets. Consider the following protocol family.


{ \samepage
\begin{protocol}[$\PP_d$]\label{prot:main:2} ~
\begin{description}
    \item [Parameters:] subsets  $\cP,\cS \subseteq [n]$ and  index $d\in (h)$.
    \item [Input:] Party $\Pc_i$ has setup parameter $f_i$ and input bit $v_i$.

	\item [First one:] ~
	\begin{description}
		\item[Party  $\Pc_i \in \cP$.] ~
		If $d=0$ [\resp $d=h$], act honestly according to  $\Pi$ \wrt input bit $v_i$ [\resp $1-v_i$].
		Otherwise,
		\begin{enumerate}
			\item Choose random  coins honestly (\ie uniformly at random).
			
			\item To each  party in $\bigcup_{i \in \set{1,\ldots,d}} \cL_j^\cP$:  send message according to input $ 1-v_i$.
			
			\item To each  party in $\bigcup_{i \in \set{d+1 ,\ldots, h}} \cL_j^\cP$: send  message according to input $v_i$ (real input).
			
			\item Send no messages to the other parties in $\cP$.
		\end{enumerate}
		
		\item[Other parties.]   Act according to  $\Pi$.
	\end{description}
	
	\item [Second round:]~
	\begin{description}
		\item[Parties in $\cP \setminus \cs$.] If $d=0$ [\resp $d=h$], act honestly according to  $\Pi$ \wrt input bit $v_i$ [\resp $1-v_i$]; otherwise, abort.
		
		\item[Parties in $\cS$.] Abort.
		

		
		\item[Other parties.]   Act according to  $\Pi$.
	\end{description}
\end{description}
\end{protocol}
}
Namely, the ``pivot'' parties in $\cP$ shift their inputs from their real input to the flipped one  according to parameter $d$. The ``aborting'' parties in $\cs$ abort at the end of the first round. Note that protocol $\PP_{0}$ is the same as protocol $\Pi^\cs$, and $\PP_h(\vv)$ acts like  $\Pi^\cs(\vv')$, for $\vv'$ being $\vv$ with   the coordinates in $\cP$ flipped. In the following we let $\Pi_d = \Pi^\emptyset_d$.


For  $\cP, \cS \subseteq [n]$,  $d\in (h)$ and  $c\in \zo$,  let
$$
\VV_{d,c} = \set{(\vf,\cs,\vr) \colon  \quad \PP_d(\vv;(\vf,\vr))_\ops = c^{\size{\ops}}}.
$$
letting $\ops = [n] \setminus (\cP \cup \cs)$.		 Namely,  $\VV_{d,c} $ are the sets, setup parameters and random strings  on  which  honest parties in $\PP_{d}$  halt in the second-round and output $c$. Let $\chi =  \pr{ \Pi(\vv) \ne \o^n} $ and let
$$
\cT_{d,c} = \set{\vf \colon \ppr{\cs \gets \bns}{\ppr{\vr \gets \rn}{(\vf,\cS,\vr),(\vf,\emptyset,\vr) \in \cV_{d,c} } \ge \lambda } \ge 1-\delta}
$$
The proof of   \cref{lemma:SecondRound:PR} immediately follows by  the  next lemma.
\begin{lemma}\label{lemma:SecondRound:PRR}
For  every  $k$-size subset $\cP \subset [n]$   and  $d\in [h]$, it holds that
\begin{align*}
\pr{\cT_{d,\o} } \ge   \gamma- 7\lambda- \frac{\chi + \alpha }{\lambda}.
\end{align*}
\end{lemma}
\begin{proof}[Proof of \cref{lemma:SecondRound:PR}]
Immediate by \cref{lemma:SecondRound:PRR}.
\end{proof}
\newcommand{\tV}{\widetilde{\cV}}
\newcommand{\tT}{{\widetilde{\cT}}}


The rest of this subsection is devoted for proving  \cref{lemma:SecondRound:PRR}. Fix  a $k$-size subset $\cP \subset [n]$  and omit it from the notation when clear from the context.
Let
$$
\tV_{d,c} = \set{(\vf,\cs,\vr) \colon   \forall a\in \zo   \quad \PP_{d+a}(\vv;(\vf,\vr))_\ops = c^{\size{\ops}}}.
$$
Namely,  $\tV_{d,c}$ are the sets, setup parameters and random strings  on  which  honest parties in $\PP_{d+a}$  halt in the second-round and output $c$,  \emph{regardless} whether the parties in $\cS$ aborts \emph{and} whether the parties in $\cP$ act toward those in $\cL_{d+1}$ according to input $0$ or $1$.
Let
$$
\tT_{d,c} = \set{\vf \colon \ppr{\cs \gets \bns}{\ppr{\vr \gets \rn}{(\vf,\cS,\vr),(\vf,\emptyset,\vr) \in \tV_{d,c} } \ge \lambda} \ge 1-\delta},
$$
let $\tT_{d} = \tT_{d,0}\cup \tT_{d,1}$ and let  $\tT = \bigcap_{i \in (h-1)}\tT_{d}$. \cref{lemma:SecondRound:PRR} is proved via the following claims (following  probabilities are taken over $ \vf  \gets \DF$).


\begin{claim}\label{claim:PR:00}
$\pr{\tT} \ge   \gamma- 6\lambda$.
\end{claim}


\begin{claim}\label{claim:PR:01}
$\pr{\cT_{1,\o} \mid \tT} \ge   1- \frac{\chi + \alpha }{\pr{\tT}\cdot \lambda}$.
\end{claim}


\begin{claim}\label{claim:PR:0}
$\pr{\cT_{d+1,\o} \mid \tT} <     \eta$ implies $\pr{\cT_{d,\oo} \mid \tT}  \ge 1- \eta$.
\end{claim}
\begin{proof}
	Immediate by the definitions of $\tT$ and $\cT$.
\end{proof}


\begin{claim}\label{claim:PR:1}
$\pr{\cT_{d,0} \mid \tT}  + \pr{\cT_{d,1} \mid \tT}  \le 1 +   \lambda/(h\cdot\Pr[\tT])$ for  every $d \in [h-1]$.
\end{claim}

We prove \cref{claim:PR:00,claim:PR:01,claim:PR:1} below, but first use the above claims  for proving \cref{lemma:SecondRound:PR}.


\paragraph{Proving  \cref{lemma:SecondRound:PRR}.}


\begin{proof}[Proof of \cref{lemma:SecondRound:PRR}.]
We first prove that  for every $d\in [h]$:
\begin{align*}
\pr{\cT_{d,\o} \mid \tT} \ge   1- \frac{\delta + \chi }{\pr{\tT}\cdot \lambda} - \frac{d\lambda}{h\cdot\Pr[\tT]}.
\end{align*}	
The proof is by induction on $d$. The base case  $d=1$ is by \cref{claim:PR:01}, and the induction follows by the combination of  \cref{claim:PR:0,claim:PR:1}. Applying the above for $d=h$, we get that
\begin{align}\label{eq:lemma:SecondRound:PR:1}
\pr{\cT_{h,\o} } \ge   \pr{\tT}- \frac{\delta + \chi }{\lambda} - \lambda,
\end{align}	
and the proof  follows by \cref{claim:PR:00}.
\end{proof}





So it is left to prove  \cref{claim:PR:00,claim:PR:01,claim:PR:1}. Note that the following adversaries corrupt at most $k + \ell + 2\sigma n \le t$ and thus they make a valid attack.  Since our security model consider rushing adversaries, and $\Pi$ is pubic randomness, we assume the adversary  knows   $\vf = (f_1,\ldots,f_n)$ \emph{before} sending its first round messages.

\paragraph{Proving \cref{claim:PR:00}.}
This is the only part in proof where we exploit the fact that the protocol is secure against    adaptive adversaries.
\begin{proof}[Proof of \cref{claim:PR:00}.]
	
	
	Consider the following  \emph{rushing adaptive} adversary.
	
	{ \samepage
	\begin{algorithm}[$\Ac$]\label{alg:PR:00}~
		
\begin{description}
	
\item[Pre interaction:]~




Corrupt the parties in $\cP$.


		\item[First round.] ~
		
		Let $\vf$ be the parties' setup parameters.
	
	
	   Do $1/\lambda\delta$ times:
		\begin{enumerate}
			
			\item  Sample  $\cS \gets \bns $ conditioned on $ \size{\cs} \le 2\sigma n$.
			\item  	For each   $i\in (h-1)$: estimate $\xi_i =  \ppr{\vr \gets \rn}{(\vf,\cs,\vr),(\vf,\emptyset,\vr)  \in \tV_{i}}$ by taking $\Theta( \log (h/ \lambda))$ samples of $\vr$.
			
			\item Let $d = \argmin_{i\in (h-1)} \set{\xi_i}$.
			
			\item If $\xi_d < 2\lambda$, break the loop.
		
			

		\end{enumerate}
	
	
			
		\item Corrupt the parties in $\cs \cup \cL_{d+1}$, and act  according to  $\Pi_d$.
			
			
			
			
			
			
			\item[Second round.]~
			
			Let $\vr$ be the parties' second round randomness.
			
			
			If $(\vf,\emptyset,\vr) \notin \cV_{d+a}$ for some $a\in \zo$,
			
			\quad act  according to $\Pi_{d+a}$.
			
			
			Else,
			
			\quad Let $a\in \zo$ be such that  $(\vf,\cs,\vr) \notin \cV_{d+a}$, set to $0$ if not such value exists.
			
			\quad Act  according to $\Pi^{\cs}_{d+a}$.
			
	
\end{description}			
	\end{algorithm}
}

Let $D$ be the value of $d$  chosen by $\Ac$ (at the first round of the protocol).
Since $\ppr{\cs \gets \bns}{ \size{\cs} > 2\sigma n} \le \alpha < \delta/2$,  if   $\vf \notin \tT$ then except with probability $\lambda$ it holds that  $\xi_D \le 2\lambda$.  Where  if $\xi_D \le 2\lambda$, then in the interaction with $\Ac$ the honest parties both halt in the second round and output the same value  with probability at most $5\lambda$. Where since $\alpha < \lambda$, the honest parties halt in the second round of such interaction with probability smaller than $6\lambda$. We conclude that the honest parties halt in the second round  under the above  attack with probability smaller than $\pr{\tT} + \pr{\neg \tT}\cdot   6\lambda \le \pr{\tT} +  6\lambda$, yielding that  $\pr{\tT} >    \gamma- 6\lambda$.
\end{proof}




\paragraph{Proving \cref{claim:PR:01}.}


\newcommand{\oH}{{\overline{\cH}}}

\begin{proof}[Proof of \cref{claim:PR:01}.]
By definition, for $\vf \in \cT_{1,\o}$ it holds that
\begin{align*}
\ppr{\vr \gets \rn}{\Pi_1(\vv;(\vf,\vr))_\oH  = \oo^{\size{\oH}} } = \ppr{\vr \gets \rn}{(\vf,\emptyset,\vr) \in \cV_{1,\oo} } \ge \lambda
\end{align*}
letting $\cH = \cP \cup \cL_1$ and $\oH = [n] \setminus\cH$.  Let $\eta = \ppr{\vf}{\cT_{1,\oo} \mid \tT}$, clearly,	$\ppr{\vf}{\cT_{1,\o} \mid \tT} = 1 -\eta$. By the above
\begin{align}
\pr{\Pi_1(\vv)_\oH =  \oo^{\size{\oH}}} \ge \pr{\tT}\cdot \eta \cdot \lambda
\end{align}
Finally, we notice  that
\begin{align}
\pr{\Pi_1(\vv) =  \oo^{\size{\cH}}} + \pr{\Pi(\vv) =  \o^n}  \le  1+ \alpha
\end{align}
Otherwise, the adversary corrupting the parties in  $\cH$, and acting toward the first honest parties according to $\Pi$ and toward the rest according to $\Pi_1$ violates the $\alpha$-agreement of $\Pi$. We conclude that $ \pr{\tT}\cdot \eta \cdot \lambda   \le \chi + \alpha $, and therefore $\eta \le \frac{\chi + \alpha }{\pr{\tT}\cdot \lambda}$.
 \end{proof}










\paragraph{Proving \cref{claim:PR:1}.}



The proof uses \cref{con:IsoBot} is an similar way it is used in 	the second part of the proof of the theorem.


\begin{proof}[Proof of \cref{claim:PR:1}.]
For $\vr \in \rbot$, let $ \cE(\vr)$ be the indices of the $\bot$'s in $\vr$.
We  assume \wlg that a party aborts upon  getting $\perp$ as its second round random coins. For $\vf \in \SDF$, $d\in [h-1]$ and $\o \in \zo$, let
\begin{align}
\cA^\vf_\o = \set{\vr \in \rbot \colon \Pi_d(\vv;(\vf,\vr))_{\overline{\cP \cup \cL_d \cup \cE(\vr)}} = \o^{\size{\overline{\cP \cup \cL_d \cup \cE(\vr)}}}}
\end{align}
By definition,  for $\vf \in \cT_{d,0} \cap \cT_{d,1}$ and  $\o\in \zo$, it holds that
\begin{align}
\ppr{\cs\gets \bns}{\ppr{\vr \gets \rn}{\vr,\bot_{\cS}(\vr) \in \cA^\vf_\o} \ge \lambda} \ge 1-\delta
\end{align}
By \cref{con:IsoBot},   see  \cref{eq:IsoBot}, for  $\vf \in \cT_{d,0} \cap \cT_{d,1}$ it holds that
\begin{align*}%\label{eq:2}
\ppr{\vr\gets \rn,\cS \gets \bns}{\forall \o \in \zo \colon \set{\vr,\bot_{\cS}(\vr)}\cap \cA^f_b \neq \emptyset} > \delta
\end{align*}
That is,
\begin{align}\label{eq:PR:1:4}
\ppr{\vr\gets \cR^n,\cS \gets \bns}{\forall \o\in \zo  \quad \exists \cs_\o \in \set{\cs,\emptyset} \colon  \Pi^{\cS_\o}_{d}(\vv;(\vf,\vr))_{ \overline{\cP \cup \cL_d \cup\cS_\o} }= \o^{ \size{\overline{ \cP \cup \cL_d \cup\cS_\o}}}} >\delta
\end{align}
In pursuit of contradiction, assume that $\pr{\cT_{d,0} \mid \tT}  + \pr{\cT_{d,1} \mid \tT}  \ge 1 +  \lambda/(h\cdot \Pr[\tT])$ for  some $d \in [h-1]$. It follows that
\begin{align}\label{eq:PR:1:5}
\lefteqn{\ppr{\stackrel{\vf \gets \DF}{\vr\gets \cR^n,\cS \gets \bns}}{\forall \o\in \zo  \quad \exists  \cs_\o \in \set{\cs,\emptyset}\colon  \Pi^{ \cs_\o}_{d}(\vv;(\vf,\vr))_{ \overline{\cP \cup \cL_d \cup \cs_\o} }= \o^{ \size{\overline{ \cP \cup \cL_d \cup \cs_\o}}}}}\\
&> \pr{\cT_{d,0} \cap \cT_{d,1}}\cdot \delta  \hskip30em \nonumber\\
&\ge \pr{\tT}\cdot  \pr{\cT_{d,0} \cap \cT_{d,1} \mid \tT} \cdot \delta  \nonumber\\
&\ge \pr{\tT}\cdot   \lambda/(h\cdot \Pr[\tT]) \cdot \delta  \nonumber\\
&= \lambda\delta /h  \nonumber\\
&> 8\alpha.\nonumber
\end{align}
The first inequality is by \cref{eq:PR:1:4}, the second one by the assumption that  $\pr{\cT_{d,0} \mid \tT}  + \pr{\cT_{d,1} \mid \tT}  \ge  1 +  \lambda/(h\cdot \Pr[\tT])$, and the last one by the definition of $\alpha$.

{\samepage
\noindent
Consider the following rushing adversary:

\begin{algorithm}[$\Ac$]~
	
		\begin{description}
		\item[Preinteraction.]  ~
		
		
		\begin{enumerate}
			\item  For each $i\in [h-1]$, estimate
			
			
			$\xi_i  = \ppr{\vr\gets \cR^n,\cS \gets \bns}{\forall \o\in \zo  \quad \exists  \cs_\o \in \set{\cs,\emptyset} \colon  \Pi^{ \cs_\o}_{d}(\vv;(\vf,\vr))_{\overline{ \cP \cup \cL_d \cup \cs_\o}}= \o^{ \size{\overline{ \cP \cup \cL_d \cup \cs_\o}}}} $ by taking $\Theta(\log (h/\alpha)/\alpha$ samples.
			
			
			Let $d = \argmax\set{\xi_i}$.
			
		
			
			\item Sample a random $\cS \gets \bns $ conditioned on $\size{\cs} \le 2\sigma n$.
		\end{enumerate}
		
		 Corrupt the parties in $\cP \cup \cS\cup \cL_d$.
	
	\item[First round.] Act according to $\Pi_d$.
	
	\item[Second round.] Sample  $\cs_0,\cs_1$ at random from $\set{\emptyset,\cs}$, and act towards some honest parties according to $\Pi^{\cs_0}_{d,0}$  and towards the others according to  $\Pi^{\cs_1}_{d,1}$ .
\end{description}
\end{algorithm}
}
By \cref{eq:PR:1:5} and since $\ppr{\cs \gets \bns}{\cs \ge 2\sigma n} \le \alpha$, the above adversary violates the agreement of $\Pi$ on input $\vv$ with probability larger than $\alpha$, in contradiction to the assumed agreement of $\Pi$.
\end{proof}



























