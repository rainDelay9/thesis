\def\IsDraft{} % Uncomment for the draft version

\documentclass[11pt]{article}
\usepackage{amsfonts,amsmath,amssymb,boxedminipage,color,url,fullpage,nccmath,amsthm}


\usepackage{color}

\definecolor{weborange}{rgb}{.8,.3,.3}
\definecolor{webblue}{rgb}{0,0,.8}
\definecolor{internallinkcolor}{rgb}{0,.5,0}
\definecolor{externallinkcolor}{rgb}{0,0,.5}


% pageanchor=false solves the warning "pdfTeX warning (ext4): destination with the same identifier (name{page.}) has been already used, duplicate ignored"
\usepackage[pageanchor=false,
colorlinks=true,
urlcolor=externallinkcolor,
linkcolor=internallinkcolor,
filecolor=externallinkcolor,
citecolor=internallinkcolor,
pdfstartview=FitH]{hyperref}

%\usepackage[pagebackref,
%hyperfootnotes=false,
%colorlinks=true,
%urlcolor=externallinkcolor,
%linkcolor=internallinkcolor,
%filecolor=externallinkcolor,
%citecolor=internallinkcolor,
%breaklinks=true,
%pdfstartview=FitH,
%pdfpagelayout=OneColumn]{hyperref}



\usepackage{enumerate,paralist}
\usepackage[numbers]{natbib} % \citet{foo} -> Foo et al. [5]
\usepackage[labelfont=bf]{caption}
\usepackage{aliascnt}
\usepackage{cleveref}
\usepackage{xspace}

\usepackage{relsize}
\usepackage{graphics}
\usepackage[usestackEOL]{stackengine}

%\usepackage{lmodern}% to remove LaTeX font warnings (Using lmodern removes this restriction by allowing font sizes at arbitrary sizes)
\usepackage[T1]{fontenc}

\newcommand{\remove}[1]{}
\newcommand{\Draft}[1]{\ifdefined\IsDraft\texttt{ #1} \fi}
\newcommand{\LLNCS}[1]{\ifdefined\IsLLNCS #1 \fi}
\newcommand{\NotLLNCS}[1]{\ifdefined\IsLLNCS\else #1 \fi}
\newcommand{\TLLNCS}[2]{\ifdefined\IsLLNCS#1\else #2 \fi}
\newcommand{\NotAnon}[1]{\ifdefined\IsAnon\else #1 \fi}


%%% revision and notes marker %%%%%%%%%%%%%%%%%%%%%%%%%%%%%%%%%%%%

\ifdefined\IsDraft
\newcommand{\authnote}[2]{{\bf [{\color{red} #1's Note:} {\color{blue} #2}]}}
\else
\newcommand{\authnote}[2]{}
\fi


\ifdefined\IsDraft
\newcommand{\changed}[3]{~\textbf{#1 Changed:} [{\color{red} #2}] {\color{blue} #3}}
\else
\newcommand{\changed}[3]{#3}
\fi

\ifdefined\IsDraft
\newcommand{\deleted}[2]{{~\textbf{Deleted (#1):}~{\color{red} #2 }}}
\else
\newcommand{\deleted}[2]{}
\fi
\ifdefined\IsDraft
\newcommand{\added}[2]{{ ~\textbf{Added(#1):}~{\color{blue} #2}}}
\else
\newcommand{\added}[2]{#1}
\fi
%%% Environments %%%%%%%%%%%%%%%%%%%%%%%%%%%%%%%%%%%%%%%%%%%%%%%%


\newcommand{\sdotfill}{\textcolor[rgb]{0.8,0.8,0.8}{\dotfill}} %change to \cdotfill later

\newenvironment{protocol}{\begin{proto}}{\vspace{-\topsep}\sdotfill\end{proto}}
\newenvironment{algorithm}{\begin{algo}}{\vspace{-\topsep}\sdotfill\end{algo}}
\newenvironment{experiment}{\begin{expr}}{\vspace{-\topsep}\sdotfill\end{expr}}
\newenvironment{assumption}{\begin{assum}}{\vspace{-\topsep}\sdotfill\end{assum}}
\newenvironment{scenario}{\begin{scen}}{\vspace{-\topsep}\sdotfill\end{scen}}
%\newenvironment{inclaim}{\begin{quote}\vspace{-\topsep}\begin{claiminproof}}{\end{claiminproof}\end{quote}\vspace{\topsep}}
\newenvironment{inprop}{\begin{quote}\vspace{-\topsep}\begin{propinproof}}{\end{propinproof}\end{quote}\vspace{\topsep}}
\newenvironment{inproof}{\begin{quotation}\vspace{-3\topsep}}{\end{quotation}\vspace{\topsep}}

\newcommand{\Ensuremath}[1]{\ensuremath{#1}\xspace}
\newcommand{\MathAlg}[1]{\mathsf{#1}}

\newcommand{\MathAlgX}[1]{\Ensuremath{\MathAlg{#1}}}

%%% Frames %%%%%%%%%%%%%%%%%%%%%%%%%%%%%%%%%%%%%%%%%%%%%%%%%%%%%%
% nfbox{CAPTION}{LABEL}{BODY}

\newcommand \mycaption {\small }     % Used as temporary variables in nf* environments to store caption and label for postamble.
\newcommand \mylabel {}

\ifdefined\IsLLNCS
\newenvironment{nfbox}[3]{
	\renewcommand \mycaption {#1}
	\renewcommand \mylabel {#2}
	\begin{center}\small
		\begin{tabular}{|ll|}
			\hline
			\hspace{.3ex}
			\begin{minipage}{.97\linewidth}
				\vspace{0.5ex}
				#3}
			{\smallskip
				\captionof{figure}{\mycaption}
				\label{\mylabel}
			\end{minipage}
			&\hspace{.3ex} \\
			\hline
		\end{tabular}
	\end{center}    %\vspace{-5ex}
}
\else
\newenvironment{nfbox}[3]{
	\renewcommand \mycaption {#1}
	\renewcommand \mylabel {#2}
	\begin{center}\small
		\begin{tabular}{|ll|}
			\hline
			\hspace{.3ex}
			\begin{minipage}{.94\linewidth}
				\vspace{0.5ex}
				#3}
			{\smallskip
				\captionsetup{type=figure}
				%         \captionof{figure}{\mycaption}
				%         \label{\mylabel}
			\end{minipage}
			&\hspace{-0ex} \\
			\hline
		\end{tabular}
		\captionof{figure}{\mycaption}
		\label{\mylabel}
	\end{center}    %\vspace{-5ex}
}
\fi


%%% General %%%%%%%%%%%%%%%%%%%%%%%%%%%%%%%%%%%%%%%%%%%%%%%%%%%%%%

%\newcommand{\etal}{{et~al.\xspace}
\newcommand{\aka} {also known as,\xspace}
\newcommand{\resp}{resp.,\xspace}
\newcommand{\ie}  {i.e.,\xspace}
\newcommand{\eg}  {e.g.,\xspace}
\newcommand{\whp}  {with high probability\xspace}
\newcommand{\wrt} {with respect to\xspace}
\newcommand{\wlg} {without loss of generality\xspace}
\newcommand{\Wlg} {Without loss of generality\xspace}
\newcommand{\cf}{cf.,\xspace}
\newcommand{\vecc}[1]{\left\lVert #1 \right\rVert}
\newcommand{\abs}[1]{\left\lvert #1 \right\rvert}
\newcommand{\ceil}[1]{\left\lceil #1 \right\rceil}
\newcommand{\ip}[1]{\iprod{#1}}
\newcommand{\iprod}[1]{\langle #1 \rangle_2}
\newcommand{\set}[1]{\ens{#1}}
\newcommand{\sset}[1]{\{#1\}}
\newcommand{\paren}[1]{\left(#1\right)}
\newcommand{\Brack}[1]{{\left[#1\right]}}
\newcommand{\inner}[2]{\left[#1,#2\right]}
\newcommand{\floor}[1]{\left \lfloor#1 \right \rfloor}

\newcommand{\assign}{\ensuremath{\mathrel{\vcenter{\baselineskip0.5ex \lineskiplimit0pt \hbox{\scriptsize.}\hbox{\scriptsize.}}}=}}

\newcommand{\iith}[1] {$#1$'th\xspace}
\newcommand{\ith}           {\iith{i}}
\newcommand{\jth}           {\iith{j}}
\newcommand{\kth}           {\iith{k}}
\newcommand{\rth}           {\iith{r}}


\newcommand{\cond}{\;|\;}
\newcommand{\half}{\tfrac{1}{2}}
\newcommand{\third}{\tfrac{1}{3}}
\newcommand{\quarter}{\tfrac{1}{4}}
%\newcommand{\eqdef}{\mathbin{\stackrel{\rm def}{=}}}
\newcommand{\eqdef}{:=}
\newcommand{\equivdef}{:\equiv}

\newcommand{\DSR}{{\bf DSR}}
\newcommand{\HSG}{{\bf HSG}}
\newcommand{\R}{{\mathbb R}}
\newcommand{\N}{{\mathbb{N}}}
\newcommand{\Z}{{\mathbb Z}}
\newcommand{\F}{{\cal F}}
\newcommand{\io}{\class{i.o. \negmedspace-}}
\newcommand{\zo}{\{0,1\}}
\newcommand{\zn}{{\zo^n}}
\newcommand{\zl}{{\zo^\ell}}
\newcommand{\zs}{{\zo^\ast}}

\newcommand{\suchthat}{{\;\; : \;\;}}
\newcommand{\deffont}{\em}
\newcommand{\getsr}{\mathbin{\stackrel{\mbox{\tiny R}}{\gets}}}
\newcommand{\xor}{\oplus}
\newcommand{\al}{\alpha}
\newcommand{\be}{\beta}
\newcommand{\eps}{\varepsilon}
%\newcommand{\ci} {\stackrel{\rm{c}}{\equiv}}
%\newcommand{\deltaci} {\stackrel{\rm{\delta}}{\equiv}}
%\newcommand{\statclose} {\stackrel{\rm{s}}{\equiv}}
\newcommand{\ci} {\equiv_c}
\newcommand{\deltaci} {\ci^\delta}
\newcommand{\statclose} {\equiv_s}
\newcommand{\deltaclose} {\statclose^\delta}
\newcommand{\deltaequiv} {\equiv^\delta}
\newcommand{\from}{\leftarrow}
\newcommand{\la}{\gets}

\newcommand{\negfunc}{\e}
\newcommand{\prot}{\pi}


\newcommand{\poly}{\operatorname{poly}}
\newcommand{\polylog}{\operatorname{polylog}}
\newcommand{\loglog}{\operatorname{loglog}}
\newcommand{\logstar}{\operatorname{log^\ast}}
\newcommand{\GF}{\operatorname{GF}}
\newcommand{\Exp}{\Ex}
\newcommand{\Var}{\operatorname{Var}}
\newcommand{\Hall}{\operatorname{H}}
\newcommand{\Hmin}{\operatorname{H_{\infty}}}
\newcommand{\HRen}{\operatorname{H_2}}
\newcommand{\Ext}{\operatorname{Ext}}
\newcommand{\Con}{\operatorname{Con}}

\newcommand{\nxtmsg}{\alpha}

\newcommand{\Com}{\MathAlg{Com}}
\newcommand{\Sign}{\MathAlgX{Sign}}
\newcommand{\Vrfy}{\MathAlgX{Vrfy}}
\newcommand{\SignGen}{\MathAlgX{SignGen}}

\newcommand{\Recon}{\MathAlgX{Recon}}
\newcommand{\Share}{\MathAlgX{Share}}

\newcommand{\EncGen}{\MathAlgX{Gen}}
\newcommand{\Enc}{\MathAlgX{Enc}}
\newcommand{\Dec}{\MathAlgX{Dec}}

\newcommand{\Time}{\operatorname{time}}
\newcommand{\negl}{\operatorname{neg}}
\newcommand{\Diam}{\operatorname{Diam}}
\newcommand{\Cut}{\operatorname{Cut}}
\newcommand{\Supp}{\operatorname{Supp}}
\newcommand{\GL}{\operatorname{GL}}
\newcommand{\maj}{\operatorname*{maj}}
\newcommand{\argmax}{\operatorname*{argmax}}
\newcommand{\argmin}{\operatorname*{argmin}}

\newcommand{\dk}{\mathit{dk}}
\newcommand{\ek}{\mathit{ek}}
\newcommand{\vek}{\vect{\ek}}
\newcommand{\sk}{\mathit{sk}}
\newcommand{\vk}{\mathit{vk}}
\newcommand{\vvk}{\vect{\vk}}

\newcommand{\decomval}{\rho}
\newcommand{\comval}{c}
\newcommand{\vcomval}{\vect{\comval}}
\newcommand{\comrnd}{\tilde{\comval}}
\newcommand{\vcomrnd}{\vect{\comrnd}}
\newcommand{\encval}{e}
\newcommand{\vencval}{\vect{\encval}}
\newcommand{\rndaugct}{r}
\newcommand{\decomaugct}{\rho}
\newcommand{\comaugct}{\sigma}
\newcommand{\vcomaugct}{\vect{\comaugct}}
\newcommand{\sval}{s}
\newcommand{\vsval}{\vs}
\newcommand{\svalhat}{\hat{\sval}}
\newcommand{\vsvalhat}{\hat{\vect{\sval}}}
\newcommand{\rval}{r}

%\newcommand{\accept}{\mathtt{accept}}
%\newcommand{\reject}{\mathtt{reject}}
\newcommand{\accept}{\ensuremath{\mathtt{Accept}}}
\newcommand{\reject}{\ensuremath{\mathtt{Reject}}}

%\newcommand{\fail}{\mathtt{fail}}
\newcommand{\fail}{\mathsf{fail}}

\newcommand{\halt}{\mathsf{Halt}}

\newcommand{\MathFam}[1]{\mathcal{#1}}
\newcommand{\ComFam}{\MathFam{COM}}
\newcommand{\HFam}{\MathFam{H}}
\newcommand{\FFam}{\MathFam{F}}
\newcommand{\GFam}{\MathFam{\GT}}
\newcommand{\Dom}{\MathFam{D}}
\newcommand{\Rng}{\MathFam{R}}
\newcommand{\Code}{\MathFam{C}}

%%% Computational Problems %%%%%%%%%%%%%%%%%%%%%%%%%%%%%%%%%%%%%%%

\def\textprob#1{\textmd{\textsc{#1}}}
\newcommand{\mathprob}[1]{\mbox{\textmd{\textsc{#1}}}}
\newcommand{\SAT}{\mathprob{SAT}}
\newcommand{\HamPath}{\textprob{Hamiltonian Path}}
\newcommand{\StatDiff}{\textprob{Statistical Difference}}
\newcommand{\EntDiff}{\textprob{Entropy Difference}}
\newcommand{\EntApprox}{\textprob{Entropy Approximation}}
\newcommand{\CondEntApprox}{\textprob{Conditional Entropy Approximation}}
\newcommand{\QuadRes}{\textprob{Quadratic Residuosity}}
\newcommand{\CktApprox}{\mathprob{Circuit-Approx}}
\newcommand{\DNFRelApprox}{\mathprob{DNF-RelApprox}}
\newcommand{\GraphNoniso}{\textprob{Graph Nonisomorphism}}
\newcommand{\GNI}{\mathprob{GNI}}
\newcommand{\GraphIso}{\textprob{Graph Isomorphism}}
\newcommand{\GI}{\mathprob{Graph Isomorphism}}
\newcommand{\MinCut}{\textprob{Min-Cut}}
\newcommand{\MaxCut}{\textprob{Max-Cut}}

\newcommand{\yes}{YES}%{{\sc yes}}
\newcommand{\no}{NO}%{{\sc no}}


%%% Complexity Classes %%%%%%%%%%%%%%%%%%%%%%%%%%%%%%%%%%%%%%%%%%

\newcommand{\class}[1]{\mathrm{#1}}
\newcommand{\coclass}[1]{\class{co\mbox{-}#1}} % and their complements
\newcommand{\BPP}{\class{BPP}}
\newcommand{\NP}{\class{NP}}
\newcommand{\coNP}{\coclass{NP}}
\newcommand{\RP}{\class{RP}}
\newcommand{\coRP}{\coclass{RP}}
\newcommand{\ZPP}{\class{ZPP}}
\newcommand{\RNC}{\class{RNC}}
\newcommand{\RL}{\class{RL}}
\newcommand{\coRL}{\coclass{RL}}
\newcommand{\IP}{\class{IP}}
\newcommand{\coIP}{\coclass{IP}}
\newcommand{\IA}{\class{IA}}
\newcommand{\coIA}{\coclass{IA}}
\newcommand{\pcIP}{\class{public\mbox{-}coin~IP}}
\newcommand{\pcIA}{\class{public\mbox{-}coin~IA}}
\newcommand{\copcIA}{\coclass{(public\mbox{-}coin~IA)}}
\newcommand{\AM}{\class{AM}}
\newcommand{\coAM}{\class{coAM}}
\newcommand{\MA}{\class{MA}}
\renewcommand{\P}{\class{P}}
\newcommand\prBPP{\class{prBPP}}
\newcommand\prRP{\class{prRP}}
\newcommand\prP{\class{prP}}
\newcommand{\Ppoly}{\class{P/poly}}
\newcommand{\DTIME}{\class{DTIME}}
\newcommand{\ETIME}{\class{E}}
\newcommand{\BPTIME}{\class{BPTIME}}
\newcommand{\AMTIME}{\class{AMTIME}}
\newcommand{\ioAMTIME}{\class{[i.o. \negmedspace-\negmedspace AMTIME]}}
\newcommand{\coAMTIME}{\class{coAMTIME}}
\newcommand{\NTIME}{\class{NTIME}}
\newcommand{\coNTIME}{\class{coNTIME}}
\newcommand{\AMcoAM}{(\class{AM} \cap \class{coAM}) \class{TIME}}
\newcommand{\NPcoNP}{(\class{N} \cap \class{coN}) \class{TIME}}
\newcommand{\EXP}{\class{EXP}}
\newcommand{\SUBEXP}{\class{SUBEXP}}
\newcommand{\qP}{\class{\tilde{P}}}
\newcommand{\PH}{\class{PH}}
\newcommand{\SV}{\class{SV}}
\newcommand{\SVpoly}{\class{SV/poly}}
\newcommand{\NEXP}{\class{NEXP}}
\newcommand{\PSPACE}{\class{PSPACE}}
\newcommand{\NE}{\class{NE}}
\newcommand{\coNE}{\class{coNE}}

% zero knowledge complexity classes
\newcommand{\ZK}{\ensuremath{\class{ZK}}}
\newcommand{\SZK}{\SZKP}
\newcommand{\CZK}{\CZKP}
\newcommand{\SZKa}{\SZKA}
\newcommand{\CZKa}{\CZKA}
\newcommand{\SZKP}{\class{SZKP}}
\newcommand{\SZKA}{\class{SZKA}}
\newcommand{\CZKP}{\class{CZKP}}
\newcommand{\CZKA}{\class{CZKA}}
\newcommand{\coSZKP}{\coclass{SZKP}}
\newcommand{\coSZKA}{\coclass{SZKA}}
\newcommand{\coCZKP}{\coclass{CZKP}}
\newcommand{\coCZKA}{\coclass{CZKA}}
\newcommand{\NISZKP}{\class{NI\mbox{-}SZKP}}

\newcommand{\HVSZK}{\HVSZKP}
\newcommand{\HVCZK}{\HVCZKP}
\newcommand{\HVSZKa}{\HVSZKA}
\newcommand{\HVCZKa}{\HVCZKA}
\newcommand{\HVSZKP}{\class{HV\mbox{-}SZKP}}
\newcommand{\HVCZKP}{\class{HV\mbox{-}CZKP}}
\newcommand{\HVSZKA}{\class{HV\mbox{-}SZKA}}
\newcommand{\HVCZKA}{\class{HV\mbox{-}CZKA}}




%Cref issues
%make all reference start with uppercase
\renewcommand{\cref}{\Cref}


\TLLNCS{
	\newaliascnt{claiml}{theorem}
	\newtheorem{claiml}[claiml]{Claim}
	\aliascntresetthe{claiml}
	
	\renewenvironment{claim}{\begin{claiml}}{\end{claiml}}
}
{
	\newtheorem{theorem}{Theorem}[section]
	
	
	\newaliascnt{lemma}{theorem}
	\newtheorem{lemma}[lemma]{Lemma}
	\aliascntresetthe{lemma}
	
	%\newaliascnt{figure}{theorem}
	%\newtheorem{figure}[figure]{Figure}
	%\aliascntresetthe{figure}
	
	\newaliascnt{claim}{theorem}
	\newtheorem{claim}[claim]{Claim}
	\aliascntresetthe{claim}
	
	
	\newaliascnt{corollary}{theorem}
	\newtheorem{corollary}[corollary]{Corollary}
	\aliascntresetthe{corollary}
	
	\newaliascnt{proposition}{theorem}
	\newtheorem{proposition}[proposition]{Proposition}
	\aliascntresetthe{proposition}
	
	
	\newaliascnt{conjecture}{theorem}
	\newtheorem{conjecture}[conjecture]{Conjecture}
	\aliascntresetthe{conjecture}
	
	\newaliascnt{adversary}{theorem}
	\newtheorem{adversary}[adversary]{Adversary}
	\aliascntresetthe{adversary}	
	
	
	\newaliascnt{definition}{theorem}
	\newtheorem{definition}[definition]{Definition}
	\aliascntresetthe{definition}
	
	
	\newaliascnt{remark}{theorem}
	\newtheorem{remark}[remark]{Remark}
	\aliascntresetthe{remark}
	
	
	\newaliascnt{example}{theorem}
	\newtheorem{example}[example]{Example}
	\aliascntresetthe{example}
}

\crefname{lemma}{Lemma}{Lemmas}
\crefname{figure}{Figure}{Figures}
\crefname{claim}{Claim}{Claims}
\crefname{corollary}{Corollary}{Corollaries}
\crefname{proposition}{Proposition}{Propositions}
\crefname{conjecture}{Conjecture}{Conjectures}
\crefname{definition}{Definition}{Definitions}
\crefname{remark}{Remark}{Remarks}
\crefname{exmaple}{Example}{Examples}



\newaliascnt{construction}{theorem}
\newtheorem{construction}[construction]{Construction}
\aliascntresetthe{construction}
\crefname{construction}{Construction}{Constructions}

\newaliascnt{fact}{theorem}
\newtheorem{fact}[fact]{Fact}
\aliascntresetthe{fact}
\crefname{fact}{Fact}{Facts}


\newaliascnt{notation}{theorem}
\newtheorem{notation}[notation]{Notation}
\aliascntresetthe{notation}
\crefname{notation}{Notation}{Notation}




\crefname{equation}{Equation}{Equations}





%%%%%%%%%%%%%%%%%%%%%%%%%%%%%%%%%%%%%%%%%%%%%%%%%%%%%%%%%%%%%%%%%%%%%%%%%
%\theoremstyle{protocol}
%what is proto
\newaliascnt{proto}{theorem}

%the name to appear for the environ
\newtheorem{proto}[proto]{Protocol}

%the name to appear in the reference
\aliascntresetthe{proto}
\crefname{proto}{protocol}{protocols}
%%%%%%%%%%%%%%%%%%%%%%%%%%%%%%%%%%%%%%%%%%%%%%%%%%%%%%%%



\newaliascnt{algo}{theorem}
\newtheorem{algo}[algo]{Algorithm}
\aliascntresetthe{algo}
%\crefname{algorithm}{algorithm}{algorithms}
\crefname{algo}{algorithm}{algorithms}


\newaliascnt{expr}{theorem}
\newtheorem{expr}[expr]{Experiment}
\aliascntresetthe{expr}
\crefname{experiment}{experiment}{experiments}



\newaliascnt{assum}{theorem}
\newtheorem{assum}[assum]{Assumption}
\aliascntresetthe{assum}
\crefname{assumption}{assumption}{assumptions}

\newaliascnt{scen}{theorem}
\newtheorem{scen}[scen]{Scenario}
\aliascntresetthe{scen}
\crefname{scenario}{scenario}{scenarios}


%
\newcommand{\itemref}[1]{Item~\ref{#1}}




%%% Proof environments %%%%%%%%%%%%%%%%%%%%%%%%%%%%%%%%%%%%%%%%%%%

\def\FullBox{$\Box$}
\def\qed{\ifmmode\qquad\FullBox\else{\unskip\nobreak\hfil
		\penalty50\hskip1em\null\nobreak\hfil\FullBox
		\parfillskip=0pt\finalhyphendemerits=0\endgraf}\fi}

\def\qedsketch{\ifmmode\Box\else{\unskip\nobreak\hfil
		\penalty50\hskip1em\null\nobreak\hfil$\Box$
		\parfillskip=0pt\finalhyphendemerits=0\endgraf}\fi}

\newenvironment{proofidea}{\begin{trivlist} \item {\it Proof Idea.}} {\end{trivlist}}

%\newenvironment{proofof}[1]{\begin{proof}[Proof of~#1]}{\end{proof}}

\newenvironment{proofsketch}{\begin{trivlist} \item {\it Proof sketch.}} {\qed\end{trivlist}}

\newenvironment{proofskof}[1]{\begin{trivlist} \item {\it Proof Sketch of~#1.}} {\end{trivlist}}

%\newenvironment{claimproof}{\begin{quotation}\noindent\begin{proof}[Proof of Claim]}{\end{proof}\end{quotation}}


%%% Statistically Hiding Commitments %%%%%%%%%%%%%%%%%%%%%%%%%%%%%

\newcommand{\IHpro}{\mathsf{IH}}

\newcommand{\twobind}{\mathrm{bind}}
\newcommand{\uow}{\mathrm{uow}}


\newcommand{\wt}[1]{\widetilde{#1}}
\newcommand{\wtau}{\widetilde{\tau}}
\newcommand{\wkappa}{\widetilde{\kappa}}
\newcommand{\wlambda}{\widetilde{\lambda}}
\newcommand{\wsigma}{\widetilde{\sigma}}
\newcommand{\wtsi}{\wsigma}

\newcommand{\Tau}{\mathrm{T}}           % denotes the set of transcripts

\newcommand{\GammaRV}{\mathit{\Gamma}}
\newcommand{\PiRV}{\mathit{\Pi}}

\newcommand{\simgeq}{\; \raisebox{-0.4ex}{\tiny$\stackrel
		{{\textstyle>}}{\sim}$}\;}
\newcommand{\simleq}{\; \raisebox{-0.4ex}{\tiny$\stackrel
		{{\textstyle<}}{\sim}$}\;}

\newcommand{\eex}[2]{\Ex_{#1}\left[#2\right]}
\newcommand{\ex}[1]{\Ex\left[#1\right]}
%\newcommand{\Ex}[1]{\Exp \left[#1\right]}
\newcommand{\Ex}{\mathop{\mathbf E}}
%\renewcommand{\Pr}{{\mathrm {Pr}}}
\newcommand{\pr}[1]{\Pr\left[#1\right]}
\newcommand{\ppr}[2]{\Pr_{#1}\left[#2\right]}

\newcommand{\Ph}{P}

\newcommand{\two}{{1-out-of-2}} %{\ensuremath{\tbinom{2}{1}}}
\newcommand{\e}{\varepsilon}
\newcommand{\Sa}{\mathsf{S}}
\newcommand{\Ra}{\mathsf{R}}
\newcommand{\Sb}{\mathbb{S}}
\newcommand{\Rb}{\mathbb{R}}
\newcommand{\Sc}{\Sa}
\newcommand{\Rc}{\Ra}
\newcommand{\Sf}{S}
\newcommand{\Rf}{R}
\newcommand{\Sn}{\mathbf{S}}
\newcommand{\Rn}{\mathbf{R}}
\newcommand{\Amp}{\operatorname{\mathsf{Amplify}}}
\newcommand{\si}{\sigma}
\newcommand{\Ac}{\MathAlgX{A}}
\newcommand{\Dc}{\MathAlgX{D}}

\newcommand{\Pc}{\mathsf{P}}
\newcommand{\Pch}{{\widehat{\mathsf{P}}}}
\newcommand{\Bc}{\mathsf{B}}
\newcommand{\Cc}{\mathsf{C}}
\newcommand{\sfA}{\mathsf{A}}
\newcommand{\sfB}{\mathsf{B}}
\newcommand{\sfC}{\mathsf{C}}
\newcommand{\sfP}{\mathsf{P}}
\newcommand{\sfE}{\mathsf{E}}
\newcommand{\sfH}{\mathsf{H}}
\newcommand{\sfO}{\mathsf{O}}



\newcommand{\B}{\mathcal{B}}
\newcommand{\reveal}{\operatorname{Reveal}}
\newcommand{\cp}{\operatorname{CP}}
\newcommand{\Xh}{\mathbf{X}}
\newcommand{\Vh}{\mathbf{V}}
\newcommand{\cps}{\operatorname{CP^{1/2}}}
\newcommand{\ga}{\gamma}
\newcommand{\avg}{\operatorname{Avg}}
\newcommand{\Tset}{\mathcal{T}}
\newcommand{\Sih}{\ensuremath{S_\mathrm{IH}}}
\newcommand{\Rih}{\ensuremath{R_\mathrm{IH}}}
\newcommand{\Vih}{\ensuremath{V_\mathrm{IH}}}
\newcommand{\concat}{\circ}
\newcommand{\bA}{\overline{A}}
\newcommand{\bV}{\overline{V}}
\newcommand{\bB}{\overline{B}}
\newcommand{\bW}{\overline{W}}
\newcommand{\dI}{\mathcal{I}}
\newcommand{\dJ}{\mathcal{J}}
\newcommand{\de}{\delta}
\newcommand{\Rl}{R_{\ell}}
\newcommand{\taui}{\tau^{(1)}}
\newcommand{\tauii}{\tau^{(2)}}
\newcommand{\aS}{S^*}
\newcommand{\hS}{\hat{S}}
\renewcommand{\H}{H}
\newcommand{\Nf}{\widetilde{N}}
\newcommand{\La}{L_{\alpha}}
\newcommand{\whd}{\widehat{d}}
\newcommand{\wtd}{\widetilde{d}}
\newcommand{\wtalpha}{\widetilde{\alpha}}
\newcommand{\wtbeta}{\widetilde{\beta}}
\newcommand{\whtau}{\widehat{\tau}}
\newcommand{\zb}{\mathbf{z}}
\newcommand{\zh}{\hat{z}}
\newcommand{\ens}[1]{\left\{#1\right\}}
\newcommand{\size}[1]{\left|#1\right|}
\newcommand{\ssize}[1]{|#1|}
\newcommand{\bsize}[1]{\bigr|#1\bigr|}
\newcommand{\enss}[1]{\{#1\}}
\newcommand{\cindist}{\approx_c}
\newcommand{\sindist}{\approx_s}
\newcommand{\com}{\operatorname{\mathsf{Com}}}
\newcommand{\tP}{\widetilde{P}}
\newcommand{\tf}{\tilde{f}}
\newcommand{\out}{\operatorname{out}}
\newcommand{\open}{\operatorname{openings}}
\newcommand{\trans}{{\operatorname{trans}}}
\newcommand{\mut}{{\widetilde{\mu}}}


\newcommand{\transc}[1]{{\sf tra}(#1)}
%\newcommand{\view}{\operatorname{view}}
\newcommand{\View}{\operatorname{View}}

\newcommand{\viewr}[1]{\view_{R^*}(#1)}

\newcommand{\indist}{\approx}
\newcommand{\dist}[1]{\mathbin{\stackrel{\rm {#1}}{\equiv}}}
\newcommand{\Uni}{{\mathord{\mathcal{U}}}}

\newcommand{\phase}{phase}
\newcommand{\HR}{\mathsf{\mbox{2-to-1-}Transform}}
\newcommand{\HRfull}{\mathsf{\mbox{2-to-1-}FullTransform}}

\newcommand{\reduce}{\leq_{\mathrm{p}}}

%%% Zero Knowledge Characterization %%%%%%%%%%%%%%%%%%%%%%%%%%%%%%

\newcommand{\vadcond}{Vadhan condition}
\newcommand{\OWF}{OWF}
\newcommand{\OWFno}{OWF NO}
\newcommand{\OWFyes}{OWF YES}

\newcommand{\Sbb}{S}

\newcommand{\prob}[1]{\mathsf{\textsc{#1}}}
\newcommand{\yesinstance}{\mathrm{Y}}
\newcommand{\noinstance}{\mathrm{N}}

\newcommand{\SD}{\prob{SD}}
\newcommand{\SDP}[2]{{\SD\paren{#1,#2}}}
\newcommand{\SDy}{\SD_\yesinstance}
\newcommand{\SDn}{\SD_\noinstance}
\newcommand{\ED}{\prob{ED}}
\newcommand{\EDy}{\ED_\yesinstance}
\newcommand{\EDn}{\ED_\noinstance}
\newcommand{\EA}{\prob{EA}}
\newcommand{\EAy}{\EA_\yesinstance}
\newcommand{\EAn}{\EA_\noinstance}
\newcommand{\cEA}{\overline{\EA}}
\newcommand{\cEAy}{\cEA_\yesinstance}
\newcommand{\cEAn}{\cEA_\noinstance}
\newcommand{\EAS}{\prob{EA'}}
\newcommand{\EASy}{\EAS_\yesinstance}
\newcommand{\EASn}{\EAS_\noinstance}
\newcommand{\cEAS}{\overline{\EAS}}
\newcommand{\cEASy}{\cEAS_\yesinstance}
\newcommand{\cEASn}{\cEAS_\noinstance}
\newcommand{\FlatEA}{\prob{FlatEA}}
\newcommand{\FlatEAy}{\FlatEA_\yesinstance}
\newcommand{\FlatEAn}{\FlatEA_\noinstance}
\newcommand{\cFlatEA}{\overline{\FlatEA}}
\newcommand{\cFlatEAy}{\cFlatEA_\yesinstance}
\newcommand{\cFlatEAn}{\cFlatEA_\noinstance}

\newcommand{\CEA}{\prob{CEA}}
\newcommand{\CEAy}{\CEA_\yesinstance}
\newcommand{\CEAn}{\CEA_\noinstance}

\newcommand{\cM}{{\cal{M}}}
\newcommand{\cL}{{\cal{L}}}
\newcommand{\cI}{{\cal{I}}}
\newcommand{\cJ}{{\cal{J}}}
\newcommand{\cW}{{\cal{W}}}
%\newcommand{\cV}{{\cal{V}}}
%\newcommand{\cR}{{\cal{R}}}
%\newcommand{\cA}{{\cal{A}}}
%\newcommand{\cB}{{\cal{B}}}
%\newcommand{\cF}{{\cal{F}}}



\newcommand{\cPi}{\overline{\Pi}}
\newcommand{\piy}{\Pi_\yesinstance}
\newcommand{\pin}{\Pi_\noinstance}
\newcommand{\cpiy}{\cPi_\yesinstance}
\newcommand{\cpin}{\cPi_\noinstance}
\newcommand{\gammay}{\Gamma_\yesinstance}
\newcommand{\gamman}{\Gamma_\noinstance}
\newcommand{\Pip}{\Pi'}
\newcommand{\pipy}{\Pip_\yesinstance}
\newcommand{\pipn}{\Pip_\noinstance}
\newcommand{\FFamY}{\FFam_\yesinstance}
\newcommand{\FFamN}{\FFam_\noinstance}
\newcommand{\Gammap}{\Gamma'}
\newcommand{\gammapy}{\Gammap_\yesinstance}
\newcommand{\gammapn}{\Gammap_\noinstance}
\newcommand{\Gammapp}{\Gamma''}
\newcommand{\gammappy}{\Gammapp_\yesinstance}
\newcommand{\gammappn}{\Gammapp_\noinstance}


\newcommand{\Jset}{J}
\newcommand{\Iset}{I}
\newcommand{\Iy}{\Iset_\yesinstance}
\newcommand{\In}{\Iset_\noinstance}

\newcommand{\KL}{\operatorname{KL}}
\newcommand{\I}{\mathcal{I}}
\newcommand{\J}{\mathcal{J}}

%%% others %%%%%%%%%%%%%%%%%%%%%%%%%%%%%%%%%%%%%%%%%%%%%%%%%%%%%%

\newcommand{\person}[1]{#1}
\def\state{{\sf state}}
\newcommand{\ppt}{{\sc ppt}\xspace}
\newcommand{\pptm}{{\sc pptm}\xspace}

\newcommand{\fhp}{{\sc PEP }}
\newcommand{\nfhp}{{\sc PEP}}
\newcommand{\co}{{\cal{O}}}
\newcommand{\cH}{{\cal{H}}}
\newcommand{\cA}{\mathcal{A}}
\newcommand{\cB}{\mathcal{B}}
\newcommand{\cS}{\mathcal{S}}
\newcommand{\cU}{\mathcal{U}}
\newcommand{\cV}{\mathcal{V}}
\newcommand{\cD}{\mathcal{D}}
\newcommand{\cR}{\mathcal{R}}
\newcommand{\cT}{\mathcal{T}}
\newcommand{\cP}{\mathcal{P}}
\newcommand{\cG}{\mathcal{G}}
\newcommand{\cF}{\mathcal{F}}
\newcommand{\cZ}{\mathcal{Z}}
\newcommand{\cE}{\mathcal{E}}
\newcommand{\cC}{\mathcal{C}}
\newcommand{\cQ}{\mathcal{Q}}
\newcommand{\rh}{H}


\newcommand{\oS}{{\overline{\cS}}}
%\newcommand{\st}{\suchthat}
\newcommand{\st}{\text{ s.t.\ }}

\newcommand{\twise}{s}
\newcommand{\ndec}{t}
\newcommand{\npi}{n}
\newcommand{\qdecc}{\mu_{\Dec}}
\newcommand{\qdec}{\qdecc(\ndec)}
\newcommand{\qdecinv}{\mu_{\Dec}^{-1}(\npi)}


\newcommand{\qaa}{\mu_{\Ac}}
\newcommand{\qa}{\qaa(\npi)}
\newcommand{\qdecaa}{\mu_{\Dec}}
\newcommand{\qdeca}{\qdecaa(\mu_{\Ac}(\npi))}

\newcommand{\giveh}{\star}

\newcommand{\KDI}{\mathrm{KDI}}
\newcommand{\OWP}{\mathrm{OWP}}
\newcommand{\Bad}{{{\mathcal{B}}\mathrm{ad}}}
\newcommand{\BAD}{{\mathcal{BAD}}}
\newcommand{\baad}{\mathrm{bad}}

\newcommand{\ebvA}{\ex{Bias_{\Adv}^{\pi} \mid V = v}}
\newcommand{\ebvB}{\ex{Bias_{\Bc}^{\psi} \mid V = v}}
%\newcommand{\ebv}{\ex{B_{S_2} \mid V = v}}
\newcommand{\ebv}{\ex{B_\cT \mid V = v}}
\newcommand{\Advpi}{\Adv^{\pi}}
\newcommand{\AdvpiI}{\Adv^{\pi}_\cT}
%\newcommand{\AdvpiI}{\Adv^{\pi}_{\cs,\I}}
%\newcommand{\AdvpiI}{\Adv^{\pi}_{\cs,\cT}}
%\newcommand{\Advpsi}{\Adv^{\psi_{\cs}}}
\newcommand{\Advpsi}{\Adv^{\psi}}
%\newcommand{\Advpsiprime}{\Adv^{\psi'}}
%\newcommand{\Advpsiprime}{\Adv^{\psi_{\cs, \J}}}
\newcommand{\Advpsiprime}{\Adv^{\psi}}
\newcommand{\AdvpiIprime}{{\Adv_{\cT}^{\pi}}}
%\newcommand{\AdvpiIprime}{{\Adv_{(\cs, \J), \cT}^{\pi}}}
\newcommand{\Jcoll}{\J_1,\ldots,\J_\numcalls}
%\newcommand{\comStar}{\committee_{i^\ast_{\committee}}}
\newcommand{\comStar}{\committee^\ast}
\newcommand{\comStarRV}{C_{\IS_{\committee}}}

%%%%%%%%%%%%%%%%%%%%%%%%%%%%%%%%%%%%%%%%%%%%  PR



\newcommand{\NextM}{\operatorname{GetNextView}}
\newcommand{\Sampler}{{\mathsf{Sam}}}
\newcommand{\Challenger}{{\mathsf{Chalenger}}}

\newcommand{\Acc}{\operatorname{Accept}}
\newcommand{\Halt}{{\cal{S}}}
\newcommand{\HaltV}{S}
\newcommand{\Coins}{{r^k}}
\newcommand{\CoinsV}{{R^k}}

\newcommand{\Expr}{\operatorname{Exp}}
\newcommand{\Emb}{\operatorname{Emb}}
\newcommand{\p}{{{\mathsf{P}}}}
\newcommand{\D}{{{\mathsf{D}}}}
\newcommand{\pEx}{\p_{\View,\Is}}
\newcommand{\GlobalLrgEfc}{\operatorname{GlobalEffect}}
\newcommand{\LocalLrgEfc}{\operatorname{LocalEffect}}
\newcommand{\Typical}{\operatorname{Typical}}
\newcommand{\TypicalS}{\operatorname{TypicalSets}}
\newcommand{\TypicalC}{\operatorname{TypicalCoins}}
%\newcommand{\TypicalPairs}{\operatorname{TypicalPairs}}

\newcommand{\hlen}{\ell}
\newcommand{\len}{m}
\newcommand{\rd}{s}

\newcommand{\cu}{{\cal{U}}}
\newcommand{\cs}{{\cal{S}}}
\newcommand{\cv}{{\cal{V}}}
\newcommand{\cj}{{\cal{J}}}
\newcommand{\cm}{{\cal{M}}}
\newcommand{\cx}{{\cal{X}}}
\newcommand{\cy}{{\cal{Y}}}
\newcommand{\ct}{{\cal{T}}}
\newcommand{\cw}{{\cal{W}}}


\newcommand{\msg}{{\mathsf{msg}}}
\newcommand{\rc}{{r^k}}
\newcommand{\rcV}{{R^k}}
\newcommand{\round}{{\operatorname{round}}}

\newcommand{\is}{i^\ast}
\newcommand{\Is}{{I^\ast}}
\newcommand{\js}{j^\ast}
\newcommand{\ls}{l^\ast}

\newcommand{\conc}{\circ}

\newcommand{\Tableofcontents}{
	\ifdefined\IsLLNCS \else
	\thispagestyle{empty}
	\pagenumbering{gobble}
	\clearpage
	\setcounter{tocdepth}{2}
	\tableofcontents
	\thispagestyle{empty}
	\clearpage
	\pagenumbering{arabic}
	\fi
}

\newcommand{\vect}[1]{{ \boldsymbol{#1}}}

\newcommand{\vb}{\vect{b}}
\newcommand{\vc}{\vect{c}}
\newcommand{\ve}{\vect{e}}
\newcommand{\vf}{\vect{f}}
\newcommand{\vm}{\vect{m}}
\newcommand{\vr}{\vect{r}}
\newcommand{\vs}{\vect{s}}
\newcommand{\vv}{\vect{v}}
\newcommand{\vx}{\vect{x}}
\newcommand{\vy}{\vect{y}}
\newcommand{\vw}{\vect{w}}
\newcommand{\vS}{\vect{S}}
%\newcommand{\vWS}{\mbox{\footnotesize{$\vect{\WS}$}}}

\newcommand{\rinput}{r_\textsf{input}}
\newcommand{\rmask}{r_\textsf{mask}}
\newcommand{\vrmask}{\vr_\textsf{mask}}
\newcommand{\rprot}{r_\textsf{prot}}
\newcommand{\vrprot}{\vr_\textsf{prot}}


\newcommand{\party}[1]{%
	\IfEqCase{#1}{%
		{1}{\Ac}
		{2}{\Bc}
		{3}{\Cc}
		% you can add more cases here as desired
	}[\PackageError{\party}{Undefined option to party: #1}{}]%
}%

\newcommand{\partyk}[2]{%
	\party{#1}^{#1}
}%

\newcommand{\vparty}[2]{%
	#1_{\party{#2}}
}%

\newcommand{\xparty}[1]{%
	\vparty{x}{#1}
}%

\newcommand{\wparty}[1]{%
	\vparty{w}{#1}
}%


\newcommand{\Adv}{\Ac} % {\Dc}
\newcommand{\ptp}{{point-to-point}\xspace}
\newcommand{\secParam}{\kappa}

\newcommand{\QParty}{\MathAlgX{Q}}
\newcommand{\Party}{\MathAlgX{P}}
\newcommand{\cParty}{\MathAlgX{P}^\ast}
\newcommand{\TParty}{\MathAlgX{\tilde P}}
\newcommand{\Pone}{\Party_1}
\newcommand{\Ptwo}{\Party_2}
\newcommand{\Pthree}{\Party_3}
\newcommand{\Pfour}{\Party_4}
\newcommand{\Pfive}{\Party_5}
\newcommand{\pone}{\Pone}
\newcommand{\ptwo}{\Ptwo}
\newcommand{\pthree}{\Pthree}
\newcommand{\pfour}{\Pfour}
\newcommand{\pfive}{\Pfive}
\newcommand{\cpone}{\Pone^\ast}
\newcommand{\cptwo}{\Ptwo^\ast}
\newcommand{\cpthree}{\Pthree^\ast}
\newcommand{\cpfour}{\Pfour^\ast}
\newcommand{\cpfive}{\Pfive^\ast}

%% VIEW and OUTPUT defs

\newcommand{\viewps}[2]{%
	\view_{#1}^{#2}
%	\text{VIEW}_{#1}^{#2}
}%

\newcommand{\outps}[2]{%
	\outputrv_{#1}^{#2}
%	\text{OUTPUT}_{#1}^{#2}
}%

\newcommand{\nmf}[3]{%
	\emph{MSG}(\pi,#1,#2,#3)
}%

%\newcommand{\ba}[3]{%
%	$(#1,#2,#3)$-BA}%
\newcommand{\ba}{%
	$(n,t,\alpha,\beta)$-BA }%

\newcommand{\batwo}[2]{%
	$#1$-\emph{party},$#2$-\emph{resilient} Byzantine agreement}%

\newcommand{\res}[1]{%
	\emph{$#1$-resilient}}%

\newcommand{\sound}[1]{%
	\emph{$#1$-sound}}%

\newcommand{\epsound}{\sound{\epsilon} }
\newcommand{\epsoundness}{\emph{$\epsilon$-soundness} }
\newcommand{\tres}{\res{t} }
\newcommand{\tresiliency}{\emph{$t$-resiliency} }
\newcommand{\andtxt}{\text{ and }}

\newcommand{\val}{\emph{validity}}
\newcommand{\agr}{\emph{agreement}}

\newcommand{\HalfP}{\frac n2}
\newcommand{\ThirdP}{\frac n3}

\newcommand{\Sim}{\MathAlgX{S}}
\newcommand{\aux}{z}
\newcommand{\ys}{y^\ast}

\newcommand{\SMbox}[1]{\mbox{\scriptsize {\sc #1}}}
\newcommand{\REAL}{\SMbox{REAL}}
\newcommand{\IDEAL}{\SMbox{IDEAL}}
\newcommand{\HYBRID}{\SMbox{HYBRID}}
\newcommand{\HYB}{\SMbox{HYB}}

\newcommand{\bigbrack}{{\vphantom{2^{2^2}}}}
\mathchardef\mhyphen="2D

\newcommand{\committee}{{\cal{C}}}
%\newcommand{\committee}{\mbox{\footnotesize{$\cal{C}$}}}
\newcommand{\vCS}{\vect{\cal{C}}}
%\newcommand{\vCS}{\mbox{\footnotesize{$\vect{\cal{C}}$}}}
%P\newcommand{\vWS}{\mbox{\footnotesize{$\vect{\cal{C}}$}}}

\newcommand{\CS}{{\cal{C}}}
\newcommand{\DS}{{\cal{D}}}
\newcommand{\IS}{{\mathcal{I}}}
\newcommand{\JS}{{\mathcal{J}}}
\newcommand{\mS}{{\mathcal{S}}}
\newcommand{\MS}{{\cal{M}}}
\newcommand{\PS}{{\cal{P}}}
%\newcommand{\WS}{{\cal{W}}}
%\newcommand{\WS}{\mbox{\footnotesize{$\cal{C}$}}}
\newcommand{\ID}{{\cal{J}}}
\newcommand{\OUT}{{\cal{V}}}

\newcommand{\outvalue}{\MathAlgX{out}}
\newcommand{\IdentifyParty}{\MathAlgX{IdentifyParty}}

\newcommand{\Comp}{\MathAlgX{Compiler}}
\newcommand{\CompNHMNI}{\MathAlgX{Compiler}_{\mbox{\tiny $\MathAlgX{no\mhyphen hm,no\mhyphen in}$}}}
\newcommand{\CompHMNI}{\MathAlgX{Compiler}_{\mbox{\tiny $\MathAlgX{hm,no\mhyphen in}$}}}
\newcommand{\CompNHM}{\MathAlgX{Compiler}_{\mbox{\tiny $\MathAlgX{no\mhyphen hm}$}}}
\newcommand{\CompHM}{\MathAlgX{Compiler}_{\mbox{\tiny $\MathAlgX{hm}$}}}
\newcommand{\Compiler}[2]{\MathAlgX{\Comp}^{{#1}\rightarrow{#2}}}
\newcommand{\CompilerHMNI}[2]{\CompHMNI^{{#1}\rightarrow{#2}}}
\newcommand{\CompilerNHMNI}[2]{\CompNHMNI^{{#1}\rightarrow{#2}}}
\newcommand{\CompilerHM}[2]{\CompHM^{{#1}\rightarrow{#2}}}
\newcommand{\CompilerNHM}[2]{\CompNHM^{{#1}\rightarrow{#2}}}

\newcommand{\rounds}{\MathAlgX{rounds}}
\newcommand{\fout}[1]{\MathAlgX{SS_{out}}({#1})}
\newcommand{\foutss}[3]{\MathAlgX{SS_{out}^{({#2},{#3})}}({#1})}
\newcommand{\finout}[4]{\MathAlgX{SS}_{\MathAlgX{in \mhyphen out}}^{{#2} \rightarrow ({#3},{#4})}({#1})}
\newcommand{\fin}[3]{\MathAlgX{SS}_{\MathAlgX{in}}^{{#2} \rightarrow {#3}}({#1})}
\newcommand{\faugct}{f_{\MathAlgX{aug\mhyphen ct}}}
\newcommand{\zkmany}{\MathAlgX{ZK}^{\textsc{1:M}}}
\newcommand{\Renc}{R_{\textsf{enc}}}
\newcommand{\fcomor}{f_{\MathAlgX{com \mhyphen or}}}
\newcommand{\felect}{f_{\MathAlgX{elect}}}
\newcommand{\fcf}[1]{f^{{#1}}_{\MathAlgX{cf}}}

\newcommand{\append}{\MathAlgX{extend}}
%\newcommand{\append}{\operatorname{extend}}
\newcommand{\numcomm}{\ell}
%\newcommand{\numcomm}{w}
\newcommand{\numcalls}{s}
%\newcommand{\numcalls}{\ell}
\newcommand{\consistent}{\gamma}

\newcommand{\viewm}[2]{\text{view}_{#2}(\Party_{#1})}
\newcommand{\viewl}[3]{\text{view}_{#2}^{#3}(\Party_{#1})}
\newcommand{\outm}[2]{\text{out}_{#2}(\Party_{#1})}
\newcommand{\outl}[3]{\text{out}_{#2}^{#3}(\Party_{#1})}
\newcommand{\view}{\textsc{view}}
\newcommand{\outputprot}{\textsc{out}}
\newcommand{\outputrv}{\textsc{output}}

%%%%%%%%%%%%%%%%%%%%%%%%%%%%%%%%%%%%%%%%%%%%%%%%%%%%%
%%%%%%%%%%%%%%%%% Added by Matan %%%%%%%%%%%%%%%%%%%%
%%%%%%%%%%%%%%%%%%%%%%%%%%%%%%%%%%%%%%%%%%%%%%%%%%%%%


\newcommand{\oT}{{\overline{\cT}}}
\newcommand{\oTp}{{\overline{\cT'}}}


\newcommand{\cSt}{{\cS^\ast}}

\newcommand{\tcA}{{\Tilde{\mathcal A}}}
\newcommand{\tcAb}{{\Tilde{\mathcal{A}_b}}}
\newcommand{\tr}{\Tilde{r}}


\renewcommand{\pr}[1]{\ppr{}{#1}}

\newcommand{\Del}[2]{#1^{#2 \bot}}
\newcommand{\Dels}[2]{#1^{(#2) \bot}}
\newcommand{\pidels}[3]{\Pi(b_{#1}, \Dels{#2}{#3})}
\newcommand{\pidelsrs}[1]{\pidels{#1}{r}{\cS}}
\newcommand{\tpivot}{\cT^{\text{pivot}}}
\newcommand{\tprop}{\cT^{\text{prop}}}

\newcommand{\Chg}[2]{#1^{#2 \ast}}


\newcommand{\zob}{\set{0,1,\bot}}
\newcommand{\zobn}{\zob^n}

\renewcommand{\sb}{\set{\Sigma \cup \bot}}
\newcommand{\sn}{\Sigma^n}
\newcommand{\sel}{\Sigma^\ell}
\newcommand{\rn}{\cR^n}
\newcommand{\sbn}{\sb^n}
\newcommand{\sbl}{\sb^\ell}

\def\eps{\varepsilon}

\newcommand{\vecb}{\mathbf{b}}
\newcommand{\inp}[1]{\mathbf{#1}}
\newcommand{\zon}{\zo^n}
\newcommand{\sbotn}{\set{\Sigma \cup \bot}^n}
\newcommand{\rbotn}{\rbot^n}
\newcommand{\rbot}{\set{\cR \cup \bot}}

\newcommand{\sbotnprime}{\set{\Sigma \cup \bot}^{n'}}

\newcommand{\Pideal}{\Pi_2^{\text{ideal}}}
\newcommand{\PInPoly}{p(\cdot) \in \text{Poly}}
\newcommand{\bl}[1]{\mathbf{b}_\ell^{#1}}
\newcommand{\bli}{\bl{{i}}}
\newcommand{\blim}{\bl{i-1}}
\newcommand{\blip}{\bl{i+1}}
\newcommand{\bliabove}{\bl{{i,+}}}
\newcommand{\blibelow}{\bl{{i,-}}}
\newcommand{\mkj}[1]{M^{k,j}_{#1}}
\newcommand{\mkjzero}{\mkj{0}}
\newcommand{\mkjone}{\mkj{1}}

\newcommand{\fsrba}{FSR-BA }

\newcommand{\indi}{\cI_{\cS}(i,b)}
\newcommand{\indj}{\cJ_{\cS}(i,b)}
\newcommand{\nchoosesig}{{[n] \choose {\sigma \cdot n}}}
\newcommand{\vstar}{\mathbf{v}^{\star}}




\newcommand{\qand}{\quad \land \quad}	

%%%%%%%%%%%%%%%%%%%%%%%%%%%%%%%%%%%
\def\inote{\Inote}
\def\rnote{\Rnote}
\def\mnote{\Mnote}
\def\nnote{\Nnote}
\newcommand{\Inote}[1]{\authnote{Iftach}{#1}}
\newcommand{\Mnote}[1]{\authnote{Matan}{#1}}
\newcommand{\Rnote}[1]{\authnote{Ran}{#1}}
\newcommand{\Nnote}[1]{\authnote{Nikos}{#1}}

\newcommand{\iadded}[1]{\added{Iftach}{#1}}
\newcommand{\radded}[1]{\added{Ran}{#1}}
\newcommand{\madded}[1]{\added{Matan}{#1}}
\newcommand{\nadded}[1]{\added{Nikos}{#1}}

\newcommand{\ichanged}[2]{\changed{Iftach}{#1}{#2}}
\newcommand{\rchanged}[2]{\changed{Ran}{#1}{#2}}
\newcommand{\mchanged}[2]{\changed{Matan}{#1}{#2}}
\newcommand{\nchanged}[2]{\changed{Nikos}{#1}{#2}}

\newcommand{\ideleted}[1]{\deleted{Iftach}{#1}}
\newcommand{\rdeleted}[1]{\deleted{Ran}{#1}}
\newcommand{\mdeleted}[1]{\deleted{Matan}{#1}}
\newcommand{\ndeleted}[1]{\deleted{Nikos}{#1}}
%%%%%%%%%%%%%%%%%%%%%%%%%%%%%%%%%



\newcounter{fnnumber}

\newcommand{\adaptive}{{\MathAlgX{adaptive}\xspace}}
\newcommand{\nonadaptive}{{\MathAlgX{non\mhyphen adaptive}\xspace}}

\newcommand{\BA}{\MathAlgX{BA}}
\newcommand{\Agr}{\MathAlgX{agreement}}
\newcommand{\Vld}{\MathAlgX{validity}}
\renewcommand{\Halt}{\MathAlgX{halting}}


\newcommand{\bls}{\cB_{\ell,\sigma}}

\renewcommand{\o}{b}
\newcommand{\oo}{\overline{\o}}

\newcommand{\omin}{\set{\o,\bot}}
\newcommand{\oomin}{\set{\oo,\bot}}
\newcommand{\vz}{\vv_0}
\newcommand{\vo}{\vv_1}
\newcommand{\oP}{{\overline{\cP}}}
\newcommand{\bns}{\mathbf{D}_{n,\sigma}}
\newcommand{\bne}{\cB_{n,\eps}}
\newcommand{\err}{\mathsf{err}}
\newcommand{\fnk}{\floor{(n-k)/2}}
\newcommand{\cnk}{\ceil{(n-k)/2}}
\newcommand{\fnt}{\floor{(n-t)/2}}
\newcommand{\cnt}{\ceil{(n-t)/2}}

\newcommand{\mon}{\set{\bot^n}}

\newcommand{\FirstErr}{2^{t-n}}

\title{On the Round Complexity of Randomized Byzantine Agreement
\Draft{\\{\small \sc Working Draft: Please Do Not Distribute}}
}

\author{Ran Cohen\thanks{Boston University and Northeastern University. E-mail: \texttt{rancohen@ccs.neu.edu}. Research supported by the Northeastern University Cybersecurity and Privacy Institute Post-doctoral fellowship, IARPA under award 2019-19020700009 (ACHILLES), NSF grant TWC-1664445, NSF grant 1422965, and by the NSF MACS project. Some of this work was done while the author was a post-doc at Tel Aviv University, supported by ERC starting grant 638121.}
\and Iftach Haitner\thanks{School of Computer Science, Tel Aviv University. E-mail: \texttt{iftachh@cs.tau.ac.il}. Member of the Check Point Institute for Information Security.}~\footnotemark[5] %
\and Nikolaos Makriyannis\thanks{Department of Computer Science, Technion. E-mail: \texttt{n.makriyannis@gmail.com}. Research supported by ERC advanced grant 742754.}~\footnotemark[5]
\and Matan Orland\thanks{School of Computer Science, Tel Aviv University. E-mail: \texttt{matanorland@mail.tau.ac.il}.}
\footnote{Research supported by ERC starting grant 638121.}
\and Alex Samorodnitsky\thanks{School of Engineering and Computer Science, The Hebrew University of Jerusalem.\newline{} E-mail: \texttt{salex@cs.huji.ac.il}. Research partially supported by ISF grant 1724/15.}
}

\begin{document}

\sloppy
\maketitle
\begin{abstract}
We prove lower bounds on the round complexity of \emph{randomized} Byzantine agreement (BA) protocols,
bounding the halting probability of such protocols after one and two rounds. In particular, we prove that:

\begin{enumerate}
\item BA protocols resilient against $n/3$ [\resp $n/4$] corruptions terminate (under attack) at the end of the first round with probability at most $o(1)$ [\resp $1/2+ o(1)$].

\item BA protocols resilient against $n/4$ corruptions terminate at the end of the second round with probability at most $1-\Theta(1)$.

\item For a large class of protocols (including all BA protocols used in practice) and under a plausible combinatorial conjecture, BA protocols resilient against $n/3$ [\resp $n/4$] corruptions terminate at the end of the second round with probability at most $o(1)$ [\resp $1/2 + o(1)$].
\end{enumerate}
The above bounds hold even when the parties use a trusted setup phase, \eg a public-key infrastructure (PKI).

The third bound essentially matches the recent protocol of \citeauthor{Micali17} (ITCS'17) that tolerates up to $n/3$ corruptions and terminates at the end of the third round with constant probability.
\end{abstract}

\vfill
\noindent\textbf{Keywords: Byzantine agreement; lower bound; round complexity.}

\Tableofcontents

\section{Introduction}\label{sec:intro}
Byzantine agreement (BA)~\cite{PSL80,LSP82} is one of the most important problems in theoretical computer science. In
%an $n$-party
a BA protocol, a set of $n$ parties wish to jointly agree on one of the honest parties' input bits.
The protocol is \emph{$t$-resilient} if no set of $t$ corrupted parties can collude and prevent the honest parties from completing this task.
%The protocol is \emph{$t$-resilient} if the honest parties manage to perform the above task even when $t$ of the parties are corrupt.
In the closely related problem of \emph{broadcast}, all honest parties must agree on the message sent by a (potentially corrupted) sender.
Byzantine agreement and broadcast are fundamental building blocks in distributed computing and cryptography, with applications in fault-tolerant distributed systems~\cite{CL99,KBCCEGGRWWWZ00}, secure multiparty computation~\cite{Yao82,GMW87,BGW88,CCD88}, and more recently, cryptocurrencies~\cite{SM16,GHMVZ17,PS18}.

In this work we consider the \emph{synchronous} communication model, where the protocol proceeds in rounds. It is well known that in the plain model, without any trusted setup assumptions, BA and broadcast can be solved if and only if $t<n/3$~\cite{PSL80,LSP82,FLM85,GM93}. Assuming the existence of digital signatures and a public-key infrastructure (PKI), BA can be solved in the honest-majority setting $t<n/2$, and broadcast under any number of corruptions $t<n$~\cite{DS83}. Information-theoretic variants that remain secure against computationally unbounded adversaries exist using information-theoretic pseudo-signatures~\cite{PW92}.

An important aspect of BA and broadcast protocols is their \emph{round complexity}. Deterministic $t$-resilient protocols require at least $t+1$ rounds~\cite{FL82,DS83}, which is a tight lower bound~\cite{DS83,GM93}. The breakthrough results of \citet{Ben-Or83} and \citet{Rabin83} showed that this limitation can be circumvented using randomization. In particular, \citet{Rabin83} used \emph{random beacons} (common random coins that are secret-shared among the parties in a trusted setup phase) to construct a BA protocol resilient to $t<n/4$ corruptions. Rabin's protocol fails with probability $2^{-r}$ after $r$ rounds, and requires \emph{expected} constant number of rounds to reach agreement. This line of research culminated with the work of \citet{FM97} who showed how to compute the common coins from scratch, yielding expected-constant-round BA protocol in the plain model, resilient to $t<n/3$ corruptions. \citet{KK06} gave an analogue result in the PKI-model for the honest-majority case. Recent results use trusted setup and cryptographic assumptions to establish a surprisingly small expected round complexity, namely $9$ for $t<n/3$~\cite{Micali17} and $10$ for $t<n/2$~\cite{MV17,ADDNR19}.

The expected-constant-round protocols mentioned above are guaranteed to terminate (with negligible error probability) within a poly-logarithmic number of rounds.
The lower bounds on the guaranteed termination from~\cite{FL82,DS83} were generalized by \cite{CMS89,KY86}, showing that any randomized $r$-round protocol must fail with probability at least $(c\cdot r)^{-r}$ for some constant $c$. However, to date there is no lower bound on the \emph{expected} round complexity of randomized BA.

In this work, we tackle this question and show new lower bounds for randomized BA. To make the discussion more informative, we consider a more explicit definition that bounds the halting probability within a specific number of rounds. A lower bound based on such a definition readily implies a lower bound on the expected round complexity of the BA protocol.

\subsection{The Model}\label{sec:intro:model}
We start with   describing in more details the model in which our lower bounds are given.  In the BA protocols we consider the  parties are communicating over a synchronous network of private and authenticated channels. Each party starts the protocol with an input bit and upon completion decides on an output bit. The protocol is  $t$-resilient  if  when  facing $t$ colluding parties that attack the protocol it holds that: (1) all honest parties agree on the same output bit (\emph{agreement}), and (2) if all honest parties start with the same input bit, then this is the common output bit (\emph{validity}). The protocols might have a \emph{trusted setup phase}: a trusted external party samples correlated values and distributes them between the parties. A setup phase is known to be essential for tolerating $t\geq n/3$ corruptions, and seems to be crucial for highly efficient protocols such as \cite{Micali17,SM16,MV17,ADDNR19,ACDNPRS19}. The trusted setup phase is typically implemented using (heavy) secure multiparty computation \cite{BCGTV15,BGG18}, via a public-key infrastructure, or with a random oracle (that is used to model proof of work)~\cite{PS17}. \rnote{added}

%We give a brief overview of the problem and the model (\cf \cref{?}). There are $n$ parties $\Party_1, \ldots,\Party_n$,
%$\set{P_i}_{i\in [n]}$,
%$t$ of which are corrupt, communicating via private and authenticated channels, with each $\Party_i$ admitting an input bit $v_i\in \zo$. The parties aim to compute a single bit $z\in \zo$ such that $z$ is equal to one of the honest parties' input; this property is referred to as validity/agreement. \rnote{I think it's best to separate agreement and validity here} For presentation purposes, in the present section, we will assume that validity/agreement holds perfectly against $t$ corrupted parties for the protocols we consider (we briefly discuss the non-perfect case at the end of the section, and the rest of the paper tackles the non-perfect case).



\paragraph{Locally consistent adversaries.}
The attacks presented in the paper require very limited capabilities from the corrupted parties  (a limitation that makes our bounds stronger). Specifically they might (1) prematurely abort, and (2) send messages to different parties based on \emph{differing} input bits and/or incoming messages from other corrupted parties. We emphasize that corrupted parties sample their random coins honestly (and use the same coins for all messages sent). In addition,
%And,
they do not lie about messages received from honest parties.
%Although a corrupted party may act inconsistently towards different honest parties, it is possible to enforce consistency towards each individual honest party (as explained below); therefore, we term this adversarial strategy as \emph{locally consistent}.

%\begin{definition}[Locally consistent adversaries, informal]
%A party is said to be {\sf locally consistent} \wrt a protocol, if it acts honestly apart from the following cheats:
%\begin{description}
%	\item[Aborting:] Abort prematurely.
%	
%	\item[Input and message selection:] A locally consistent party might act towards some parties as if its input is $0$ and towards other parties as if its input is $1$ (these sets of parties might overlap). %If cheating, it aborts at the end of this round.
%	
%	If the party got in one of the previous rounds several messages from another locally consistent party, it might act towards some of the parties as if it got one message and towards other parties as if got a different message.
%\end{description}
%\end{definition}

\paragraph{Public-randomness protocols.}
In many randomized
%expected-constant-round BA
protocols, including all those used in practice, cryptography is merely used to provide \emph{message authentication}---preventing a  party from lying about the messages it received---and \emph{verifiable randomness}---forcing the parties to toss their coins correctly. The description of such protocols can be greatly simplified if only security against locally consistent adversaries is required (in which corrupted parties do not lie about their coin tosses and their incoming messages from honest parties). This motivates the definition of \emph{public-randomness} protocols, where each party publishes its local coin tosses for each round (the party's first message also contains its setup parameter, if such exists).
Although our attacks apply to arbitrary BA protocols, we show even stronger lower bounds for public-randomness protocols.

We illustrate the simplicity of the model by considering the BA protocol of \citet{Micali17}. In this protocol, the  cryptographic tools, digital signatures and verifiable random functions (VRFs)\footnote{A pseudorandom function that provides a non-interactively verifiable proof for the correctness of its output.}, are used to allow the parties elect leaders and toss coins with probability $2/3$ as follows: each party $\Party_i$ in round $r$ evaluates the VRF on the pair $(i,r)$ and multicasts the result. The leader is set to be the party with the smallest VRF value, and the coin is set to be the least-significant bit of this value. Since these values are uniformly distributed $\secParam$-bit strings ($\secParam$ is the security parameter), and there are at least $2n/3$ honest parties, the success probability is $2/3$. (Indeed, with probability $1/3$, the leader is corrupted, and can send its
%the adversary can send the leader's
value only to a subset of the parties, creating disagreement.)

When considering locally consistent adversaries, \citeauthor{Micali17}'s protocol can be significantly simplified by having each party randomly sample and multicast a uniformly distributed $\secParam$-bit string (cryptographic tools and setup phase are no longer needed). Corrupted parties can still send their values to a subset of honest parties as before, but they cannot send different random values to different honest parties.

A similar simplification applies to other BA protocols that are based on leader election and coin tosses such as \cite{FM97,FG03,KK06} (private channels are used for a leader-election sub-protocol), \cite{MV17,ADDNR19} (cryptography is used for coin-tossing and message-authentication), and \cite{SM16,ACDNPRS19} (cryptography is used to elect a small committee per round).\footnote{Unlike the aforementioned protocols that use ``simple'' preprocess and ``light-weight''  cryptographic tools, the protocol of \citet{Rabin83} uses a heavy, per execution, setup phase (consisting of Shamir sharing of a random coin for every potential round) that we do not know how to  cast as a public-randomness protocol.}

\begin{proposition}[Malicious security to locally consistent public-randomness protocol, informal]
Each of the BA protocols of \cite{FM97,FG03,KK06,Micali17,SM16,MV17,ADDNR19,ACDNPRS19} induces a public-randomness BA protocol secure against locally consistent adversaries, with the same parameters.
\end{proposition}


\paragraph{A useful abstraction  for protocol design.}
To complete the picture, we remark that security against locally consistent adversaries, which  may seem somewhat weak at first sight, can be compiled using standard cryptographic techniques into security against arbitrary adversaries. This reduction  becomes lossless, efficiency-wise and security-wise, when applied to public-randomness protocols. Thus, building public-randomness protocols secure against locally consistent adversaries is a useful abstraction for protocol designers that  want to use what  cryptography has to offer, but  without being bothered with the technical details.
See more details in \cref{sec:intro:LocalToFull}.

\subsection{Our Results}\label{sec:intro:ourResult}
We present three lower bounds on the halting probability of randomized BA protocols.
To keep the following introductory discussion simple, we will assume that both validity and agreement properties hold perfectly, without error.

\paragraph{First-round halting.}
Our first result bounds the halting probability after a single communication round. This is the simplest case since parties cannot inform each other about inconsistencies they encounter. Indeed, the established lower bound is quite strong, showing an exponentially small bound on the halting probability when $t\geq n/3$, and exponentially close to $1/2$ when $t\geq n/4$.

\begin{theorem}[First-round halting, informal]\label{thm:intro:FirstRound}
Let $\Pi$ be an $n$-party BA protocol and let $\gamma$ denote the halting probability after a single communication round facing a locally consistent, static,  adversary  corrupting $t$ parties. Then,
\begin{itemize}
	\item $t \ge n/3$ implies $\gamma \le \FirstErr$ for arbitrary protocols, and $\gamma=0$ for public-randomness protocols.
	\item $t \ge n/4$ implies $\gamma \le 1/2+\FirstErr$ for arbitrary protocols, and $\gamma \leq 1/2$ for public-randomness protocols.
\end{itemize}
\end{theorem}


Note that the deterministic $(t+1)$-round, $t$-resilient BA protocol of \citet{DS83} can be cast as a locally consistent public-randomness protocol (in the plain model).\footnote{When considering locally consistent adversaries, the impossibility of BA for $t=n/3$ does not apply.}
\cref{thm:intro:FirstRound} shows that for $n=3$ and $t=1$, this two-round BA protocol is essentially optimal and cannot be improved via randomization (at least without considering complex protocols that cannot be cast as public-randomness protocols).

%\cref{thm:intro:FirstRound} shows that any protocol that can be cast as a public-randomness protocol tolerating one third of locally consistent corruptions, cannot halt in one round; hence, the expected round complexity is at least $2$.
%In particular, for $n=3$ and $t=1$, the deterministic $2$-round BA protocol is essentially optimal and cannot be improved via randomization.\footnote{Recall that when considering locally consistent adversaries, the impossibility of BA for $t=n/3$ does not apply, and the deterministic $(t+1)$-round protocol of \cite{DS83} can be cast as a public-randomness protocol in the plain model.}

\paragraph{Second-round halting for arbitrary protocols.}
Our second result considers the halting probability after two communication rounds.
%rounds of communications.
This is a much more challenging regime, as honest parties have time to detect inconsistencies in first-round messages. Our bound for arbitrary protocols in this case is weaker, and shows that when $t>n/4$, the halting probability is bounded away from $1$.

\begin{theorem}[Second-round halting, arbitrary protocols, informal]\label{thm:intro:SecondRound:Arb}
 Let $\Pi$ be an $n$-party BA protocol and let $\gamma$ denote the halting probability after two communication rounds facing a locally consistent, static, adversary  corrupting $t=(1/4+\eps)n$ parties.
Then, $\gamma \le 1 - (\eps /5)^2$.
\end{theorem}

\paragraph{Second-round halting for public-randomness protocols.}
\cref{thm:intro:SecondRound:Arb} bounds the second-round halting probability of arbitrary BA protocols away from one. For public-randomness protocol we  achieve a much stronger bound. The attack requires \emph{adaptive} corruptions (as opposed to \emph{static} corruptions in the previous case) and is based on a combinatorial conjecture that is stated below.\footnote{The attack holds even without assuming \cref{con:intro:IsoBot} when considering \emph{strongly adaptive} corruptions~\cite{GKP15}, in which an adversary sees all messages sent by honest parties in any given round and, based on the message content, decides whether to corrupt a party (and alter its message or sabotage its delivery) or not. Similarly, the conjecture is not required if each party is limited to tossing a single unbiased coin. These extensions are not formally proved in this paper.\label{footnote:no_conjecture}}
%(\cref{con:intro:IsoBot})

\begin{theorem}[Second-round halting, public-randomness protocols, informal]\label{thm:intro:SecondRound:PR}
Let $\Pi$ be an $n$-party public-randomness BA protocol and let $\gamma$ denote the halting probability after two communication rounds facing a locally consistent adversary adaptively corrupting $t$ parties.
Then, for sufficiently large $n$ and assuming \cref{con:intro:IsoBot} holds,
\begin{itemize}
\item $t > n/3$ implies $\gamma=0$.
\item $t > n/4$ implies $\gamma \leq 1/2$.
\end{itemize}
%\rnote{should the theorem use $t=\beta\cdot n$ for a constant $0<\beta<1$, and then the first bound is for $\beta>1/3$ and the second for $\beta>1/4$?}
\end{theorem}

\cref{thm:intro:SecondRound:PR} shows that for sufficiently large $n$, any public-randomness protocol tolerating $t>n/3$ locally consistent corruptions cannot halt in less than three rounds (unless  \cref{con:intro:IsoBot}  is false). In particular, its expected round complexity must be at least three.

To understand the meaning of this result, recall the protocol of \citet{Micali17}. As discussed above, this protocol can be cast as a public-randomness protocol tolerating $t<n/3$ adaptive locally consistent corruptions. The protocol proceeds by continuously running a three-round sub-protocol until halting, where each sub-protocol consists of a coin-tossing round, a check-halting-on-$0$ round, and a check-halting-on-$1$ round. Executing a single instance of this
%three-round
sub-protocol demonstrates a halting probability of $1/3$ after three rounds.
By \cref{thm:intro:SecondRound:PR}, a protocol that tolerates slightly more corruptions, \ie $(1/3 +\eps) \cdot n$, for arbitrarily small $\eps>0$, cannot halt in fewer rounds.


\paragraph{Our techniques.}
Our attacks follow the spirit of many lower bounds on the round complexity on BA and broadcast~\cite{FL82,DS83,KY86,DRS90,GKKO07,AH10}. The underlying idea is to start with a configuration in which validity assures the common output is $0$, and gradually adjust it, while retaining the same output value, into a configuration in which validity assures the common output is $1$.  (For the simple case of deterministic protocols, each step of the argument requires the corrupted parties to lie about their input bits and incoming messages from other corrupted parties, but %otherwise behave honestly.)

Our main contribution, which departs from the aforementioned paradigm, is adding another dimension to the attack by aborting a random subset of parties (rather than simply manipulating the input and incoming messages). This change allows us to bypass a seemingly inherent barrier for this approach. We refer the reader to \cref{sec:Technique} for a detailed overview of our attacks.

\paragraph{The combinatorial conjecture.}
We conclude the present section by motivating and stating the combinatorial conjecture assumed in  \cref{thm:intro:SecondRound:PR}, and discussing its plausibility. We believe the conjecture to be of independent interest, as it relates to topics from Boolean functions analysis such as influences of subsets of variables \cite{Odonnel14} and isoperimetric-type inequalities \cite{Mossel2006,Mossel2013}. The nature of our conjecture makes the  following paragraphs somewhat technical, and  reading them can be postponed until after going over the description of our attack in \cref{sec:Technique}.

The analysis of our attack naturally gives rise to an isoperimetric-type inequality. For limited types of protocols, we manage to prove it using Friedgut's theorem \cite{Friedgut} about approximate juntas and the KKL theorem~\cite{KKL88}. For arbitrary protocols, however, we only manage to reduce our attack to the conjecture below.

We require the following notation before stating the conjecture. Let $\Sigma$ denote some finite set.
For $\vx\in \sn$ and $\cS \subseteq [n]$, define the vector $\bot_\cS(\vx) \in \set{\Sigma \cup \bot}^n$ by assigning all entries indexed by $\cS$ with the value $\bot$, and all other entries according to $\vx$. Finally, let $\bns$ denote the distribution induced over subsets of $[n]$ by choosing each element with probability $\sigma$ independently at random.

\def\MainConj{
For any $\sigma,\lambda >0$ there exists $\delta>0$ such that the following holds for large enough $n\in \N$: let  $\Sigma$ be a finite alphabet, and let $\cA_0,\cA_1 \subseteq \sbn$  be  two  sets such that for both $b\in \zo$:

\begin{align*}%\label{eq:1}
\ppr{\cs\gets \bns}{\ppr{\vr \gets \Sigma^n}{\vr,\bot_{\cS}(\vr) \in \cA_b} \ge  \lambda } \ge 1-\delta.
\end{align*}
Then,
\begin{align*}%\label{eq:2}
\ppr{\substack{\cS \gets \bns \vspace{.05in}\\ \vr\gets \Sigma^n }}{\forall b\in \zo\colon  \set{\vr,\bot_{\cS}(\vr)}  \cap \cA_b \neq \emptyset} \ge  \delta.
\end{align*}
}


\begin{conjecture}\label{con:intro:IsoBot}
\MainConj
\end{conjecture}

\noindent
Consider two large sets $\cA_0$ and $\cA_1$ which are ``stable'' in the following sense:  for both $\o\in \zo$, with probability $1-\delta$ over $\cS\la \bns$, it holds that both $\vr$ and $\bot_{\cS}(\vr)$ belong to $\cA_\o$, with probability at least $\lambda$ over $\vr$.  \cref{con:intro:IsoBot} stipulates  that with high probability ($\ge \delta$), the vectors $\vr$ and $\bot_{\cS}(\vr)$ lie in opposite sets (\ie one is in $\cA_0$ and the other $\cA_{1-\o}$), for random $\vr$ and $\cS$. It is somewhat reminiscent of the following flavor of isoperimetric inequality: for any two large sets $\cB_0$ and $\cB_1$, taking a random element from $\cB_0$ and resampling a few coordinates, yields an element in $\cB_1$ with large probability. Less formally, one can ``move'' from one set to the other by manipulating a few coordinates~\cite{Mossel2006,Mossel2013}.


A few remarks are in order. First, it suffices for our purposes to show that $\delta$ is a noticeable (\ie inverse polynomial) function of $n$, rather than independent of $n$.\footnote{We remark that it is rather easy to show that $\delta\ge 2^{-n}$, which is not good enough for our purposes.} We opted for the latter as it gives a stronger attack. Second, the conjecture holds for ``natural'' sets such as balls, \ie $\cA_0$ and $\cA_1$ are balls centered around $0^n$ and $1^n$ of constant radius,\footnote{The alphabet $\Sigma$ is not necessarily Boolean, and there are a couple of subtleties in defining balls.} and ``prefix'' sets, \ie sets of the form $\cA_\o=\o^k \times \set{\Sigma \cup \bot}^{n-k}$. Furthermore, the claim can be proven
%we know how prove the statement
when the probabilities over $\cS$ and $\vr$ are reversed, \ie ``with probability $\lambda$ over $\vr$, it holds that both $\vr$ and $\bot_{\cS}(\vr)$ belong to $\cA_\o$ with probability at least $1-\delta$ over $\cS$'', instead of the above. Interestingly, this weaker statement boils down to the aforementioned isoperimetric-type inequality (c.f.~\cite{Mossel2006} for the Boolean case and \cite{Mossel2013} for the non Boolean case).
	
	
We conclude by pointing out that, as mentioned in \cref{footnote:no_conjecture},
%above,
the conjecture is not needed for certain limited cases that are not addressed in detail in the present paper. One such case is sketched out in \cref{sec:Technique}.


\ifdefined\IsFullVersion
\subsection{Locally Consistent Security to Malicious Security}\label{sec:intro:LocalToFull}

As briefly mentioned in \cref{sec:intro:model}, protocols that are secure against locally consistent adversaries can be compiled to tolerate arbitrary malicious adversaries.
The compiler requires a PKI for digital signatures and verifiable random functions (VRFs)~\cite{MRV99}. A VRF is a pseudorandom function with an additional property: using the secret key and an input $x$, the VRF outputs a pseudorandom value $y$ along with a proof string $\pi$; using the public key, everyone can use $\pi$ to verify whether $y$ is the output of $x$. We consider a trusted setup phase for establishing the PKI, where every party generates keys for a VRF and for a signature scheme, and publishes the corresponding public keys.

Given a protocol that is secure against locally consistent adversaries, the compiled protocol proceeds as follows, round by round.
Each party $\Party_i$ sets its random coins for the \rth round $\rho_i^r$ (together with a proof $\pi_i^r$) by evaluating the VRF over the pair $(i,r)$.
Next, for every $j\in[n]$, party $\Party_i$ uses these coins to compute the message $m^r_{i\to j}$ for $\Party_j$, signs $m^r_{i\to j}$ along with the VRF proof $\pi^r_i$ as $\sigma^r_{i\to j}$, and sends $(m^r_{i\to j},\pi_i^r,\sigma^r_{i\to j})$ to $\Party_j$.
Finally, $\Party_i$ proves to each $\Party_j$ using a zero-knowledge proof of knowledge that:
\begin{enumerate}
    \item
    There exist an input bit $b$, random coins $\rho_i^r$, as well as random coins and incoming messages $\rho^{r'}_i$ and $(m^{r'}_{1\to i},\ldots,m^{r'}_{n\to i})$ for every $r'<r$, such that: (1) $\pi_i^r$ verifies that $\rho_i^r$ is the VRF output of $(i,r)$ (using the VRF public key of $\Party_i$), and (2) the message $m^r_{i\to j}$ is the output of the next-message function of $\Party_i$ when applied to these values.
    \item
    For every $r'<r$, the input bit $b$ and the random coins $\rho^{r''}_i$ and incoming messages $(m^{r''}_{1\to i},\ldots,m^{r''}_{n\to i})$ for every $r''<r'$, are the same as those used to generate $m^{r'}_{i\to j}$.
    \item
    For $r>1$, the messages received in the previous round are properly signed. That is, for every $k\in[n]$, there is a signature $\sigma^{r-1}_{k\to i}$ of the message $m^{r-1}_{k\to i}$ that verifies under the signature-verification key of $\Party_k$.
\end{enumerate}

When considering public-randomness protocols, the above compilation can be made much more efficient. Instead of proving in zero knowledge the consistency of each message, each party $\Party_i$ concatenates to each message all of its incoming messages from the previous round. A receiver can now locally verify the coins used by $\Party_i$ are the VRF output of $(i,r)$ (as assured by the VRF), that the incoming messages are properly signed, and that the message is correctly generated from the internal state of $\Party_i$ (which is now visible and verified).

%\rnote{the VRF keys should be honestly generated, or alternatively we need a common random string that is independent of the PKI and is used as an additional seed for the VRF}
\begin{theorem}[Locally consistent to malicious security, folklore, informal]\label{thm:local_to_malicious}
Assume PKI for digital signatures and VRF, then a BA protocol secure against locally consistent adversaries, can be compiled into a maliciously secure BA protocol with the same parameters, apart from a constant blowup in the round complexity (no blowup for public-randomness protocols).
\end{theorem}

\Inote{prove in the appendix}
\Inote{when proved, give reference to the section }

\subsection{Additional Related Work}\label{sec:relatedWork}

Following the work of \citet{FM97} in the two-thirds majority setting, \citet{KK06} improved the expected round complexity to $23$, and \citet{Micali17} to $9$. In the honest-majority setting, \citet{FG03} showed expected-constant-round protocol and \citet{KK06} expected $56$ rounds. \citet{MV17} adjusted the technique from \cite{Micali17} to the honest-majority case. \citet{ADDNR19} achieved expected $10$ rounds assuming static corruptions and expected $16$ rounds assuming adaptive corruptions. \citet{ACDNPRS19} constructed an expected-constant-round protocol tolerating $(1/2-\epsilon)\cdot n$ adaptive corruptions with sublinear communication complexity. In the dishonest-majority setting, \citet{GKKO07} constructed a broadcast protocol with expected $O(k)$ rounds, tolerating $t<n/2+k$ corruptions.

%\rnote{add more asynchronous BA references. In the asynchronous setting, \citet{CR93} gave the first analogue result to \cite{FM97}.}

\citet{AH10} extended the results of \citet{CMS89} and of \citet{KY86} on guaranteed termination of randomized BA protocols to the asynchronous setting, and provided a tight lower bound.

Randomized protocols with expected constant round complexity have \emph{probabilistic termination}, which requires delicate care \wrt composition (\ie their usage as subroutines by higher-level protocols). Parallel composition of randomized BA protocols was analyzed in \cite{Ben-Or83,FG03}, sequential composition in \cite{LLR06}, and universal composition in \cite{CCGZ16,CCGZ17}.


\subsection{Open Questions}\label{sec:OpenQuest}
Our attack on two-round halting of public-randomness protocols is based on \cref{con:intro:IsoBot}. In this work we prove special cases of this conjecture, but proving the general case remains an open challenge.

A different interesting direction is to bound the halting probability of protocols when $t<n/4$. It is not clear how to extend our attacks to this regime.
\fi

\subsection*{Paper Organization}

\ifdefined\IsFullVersion
In \cref{sec:Technique} we present a technical overview of our attacks. The formal model and the exact bounds are stated in \cref{sec:OurResult}. The proof of the first-round halting is given in \cref{sec:FirstRound}, and for second-round halting in \cref{sec:SecondRound}.
\else
In \cref{sec:Technique} we present a technical overview of our attacks. Due to space limitations, we differ the related work to \cref{sec:intro_cont}. The formal model and the exact bounds are stated in \cref{sec:OurResult}. The proof of the first-round halting is given in \cref{sec:FirstRound}, and for second-round halting in \cref{sec:SecondRound}.
\fi



\newcommand{\wb}[1]{\overline{#1}}
\newcommand{\rr}{\mathbf{r}}
\newcommand{\cN}{\mathcal{N}}
\newcommand{\cK}{\mathcal{K}}
\newcommand{\ham}{\mathrm{dist}}
%\newcommand{\ssize}[1]{|#1|}


\newcommand{\oh}{\overline{\cH}}
\newcommand{\oC}{\overline{\cC}}

\section{Our Techniques}\label{sec:Technique}
\ifdefined\IsFullVersion
In this section, we outline our techniques for proving our results. We start with explaining our bound for first-round halting of arbitrary protocols (\cref{thm:intro:FirstRound}).  We then   move to  second-round halting, starting with the weaker bound for arbitrary protocols (\cref{thm:intro:SecondRound:Arb}), and then move to the much stronger bound for public-randomness protocols (\cref{thm:intro:SecondRound:PR}).
\else
In this section, we outline our techniques for proving the lower bounds.
\fi

\ifdefined\IsFullVersion\else
\vspace{-.3cm}
\fi
%\subsection{Notations}\label{sec:notations}
\paragraph{Notations}%\label{sec:notations}
We use calligraphic letters to denote sets, uppercase for random variables, lowercase for values, boldface for vectors, and sans-serif (\eg \Ac) for algorithms (\ie Turing Machines).
For $n\in\N$, let $[n]=\set{1,\cdots,n}$ and $(n)=\set{0,1,\cdots,n}$.  Let $\ham(x, y)$ denote the  hamming distance between $x$ and $y$.  For a set $\cS \subseteq [n]$ let $\oS =  [n]\setminus \cS$. For a set $\cR\subseteq \zo^n$, let $\cR|_{\cS}=\sset{\vx_\cS  \in \zo^{\size{\cS}}\st \vx\in \cR}$, \ie $\cR|_{\cS}$ is the projection of $\cR$ on the index-set $\cS$.

Fix an $n$-party randomized BA protocol $\Pi = (\Pc_1,\ldots,\Pc_n)$.   For presentation purposes, we  assume that
validity and agreement
%the validity and agreement of $\Pi$
hold \emph{perfectly}, and consider no setup parameters  (in the subsequent sections, we remove these assumptions). Furthermore, we only address here the case where the security threshold is $t>n/3$. The case $t>n/4$ requires an additional generic step that we defer to the technical sections of the paper. We denote by $\Pi(\vv;\vr)$ the  output of an honest execution of $\Pi$ on input $\vv \in \zn$ and randomness $\vr$ (\ie each party $\Party_i$ holds input $\vv_i$ and randomness $\vr_i$). We let $\Pi(\vv)$ denote the resulting random variable determined by the parties' random coins, and we write $\Pi(\vv) =\o$ to denote the event that the parties output $\o$ in an honest execution of $\Pi$ on input $\vv$.  All corrupt  parties described below are locally consistent (see \cref{sec:intro:model}).

\subsection{First-Round Halting}\label{sec:technique:1}
Assume the honest parties of $\Pi$ halt
\ifdefined\IsFullVersion
at the end of the first round
\else
after one round
\fi
with probability $\gamma>0$ when facing $t$ corruptions.
% (on every input).
Our goal is to upperbound the value of $\gamma$. Our approach is inspired by the analogous lower-bound for deterministic protocols (\cf \cite{FL82,DS83}). Namely, we start with a configuration in which validity assures the common output is $0$, and, while maintaining the same output, we gradually adjust it into a configuration in which validity assures the common output is $1$, thus obtaining a contradiction. For  randomized protocols, the challenge is to maintain the invariant of the output, even when the probability of halting  is far from $1$.  We make the following observations:

\begin{align} \label{eq:maj}
&\text{Supermajority execution:}    \quad  \ham(\vv,\o^n) \le t \implies \Pi(\vv)=\o.
\end{align}

That is, in an honest execution of $\Pi$, if there is a supermajority ($\ge n-t$) of $\o$'s in the input vector, then the parties output $\o$ with probability $1$.  \cref{eq:maj} follows by agreement and validity by considering an adversary corrupting exactly those parties with input $v_i\neq b$, and otherwise not deviating from the protocol.


\begin{align}\label{eq:nei}
&\text{Neighboring executions (N1):} \quad \ham(\vz,\vo)\le t \implies \ppr{\vr}{\Pi(\vz;\vr)=\Pi(\vo;\vr)}\ge \gamma.
\end{align}
That is, for two input vectors that are at most $t$-far (\ie the resiliency  threshold), the probability that the executions on these vectors yield the same output when using  the same randomness is bounded below by the halting probability.  To see why \cref{eq:nei} holds, consider the following adversary corrupting subset $\cC$, for $\cC$ being the set of indices where $\vv$ and $\vv'$ disagree. For an arbitrary partition $\sset{\oC_0,\oC_1}$ of $\oC$, the adversary instructs $\cC$ to send messages according to $\vz$ to $\oC_0$ and according to $\vo$ to $\oC_1$, respectively.  With probability at least $\gamma$, all parties halt at the first round, and, by perfect agreement, all parties compute the same output.\footnote{In the above, we have chosen to ignore a crucial subtlety. In an execution of the protocol, it may be the case that there is a suitable message (according to $\vz$ or $\vo$) to prevent halting, yet the adversary cannot determine which one to send. In further sections, we address this issue by taking a random partition of $\oC$ (rather than an arbitrary one). By doing so, we introduce an error-term of $1/2^{n-t}$ when we upper bound the halting probability $\gamma$.} Since parties in $\oC_\o$ cannot distinguish this execution from a halting execution of $\Pi(\vv_\o;\vr)$, \cref{eq:nei} follows.


We deduce that if there are more than $n/3$ corrupt parties, then the halting probability is $0$; this follows by combining the two observations above for $\vz=0^{n-t}1^{t}$ and $\vo=0^{t}1^{n-t}$. Namely, by \cref{eq:maj}, it holds that $\ppr{\vr}{\Pi(\vz;\vr)=\Pi(\vo;\vr)}=0$. Thus,  by \cref{eq:nei}, $\gamma=0$.

\ifdefined\IsFullVersion\else
\vspace{-.2cm}
\fi
\subsection{Second-Round Halting -- Arbitrary Protocols}\label{sec:technique:2}

In this subsection we explain our bound for second-round halting of arbitrary protocols. Assume the honest parties of $\Pi$ halt at the end of the second round with probability $\gamma>0$ when facing $t$ corruptions (on every input). Let $t=(1/3+\eps)n$, for $\eps>0$. In spirit, the attack follows the footsteps of the single-round case described above. We show that neighboring executions compute the same output with good enough probability (related to the halting probability), and  lower-bound the latter using the supermajority observation. There is, however,  a crucial difference between the first-round and second-round cases; the honest parties can use the second round to detect whether (some) parties are sending inconsistent messages. Thus, the second round of the protocol can be used to ``catch-and-discard'' parties that are pretending to have different inputs to different parties, and so our previous attack breaks down (In the one-round case, we exploit the fact that the honest parties cannot verify the consistency of the messages they received.). Still, we show that there is a different attack that violates the agreement of any ``too-good'' scheme.

%\nnote{I changed this paragraph ->} 
At a very high level, the idea  for proving the neighboring property is to \emph{gradually} increase  the set of honest parties towards which the adversary behaves according to $\vo$ (for the remainder it behaves according to $\vz$, which is a decreasing set of parties). While the honest parties might identify the attacking parties and discard their messages, they should still agree on the output and  halt at the conclusion of the second round with high probability. We exploit this fact to show that at the two extremes (where the adversary is merely playing honestly according to $\vz$ and $\vo$, respectively), the honest parties behave essentially the same. Therefore, if at one extreme (for $\vz$) the honest parties output $\o$, it follows that they also output $\o$ at the other extreme (for $\vo$), which proves the neighboring property for the second-round case.


We implement the above by augmenting the one-round attack as follows. In addition to corrupting a set of parties that feign different inputs to different parties, the adversary corrupts an extra set of parties that is inconsistent with regards to the messages it received from the first set of corrupted parties.  To distinguish between the two sets of corrupted parties, the former (first) will be referred to as ``pivot'' parties (since they pivot their input) and will be denoted $\cP$, and the latter will be referred to as ``propagating'' parties (since they carefully choose what message to propagate at the second round) and will be denoted $\cL$. We emphasize that the propagating parties deviate from the protocol only at the second round and only with regards to the messages received by the pivot parties (not with regards to their input -- as is the case for the pivot parties). In more detail, we partition $\oP = [n]\setminus \cP$ into $\ell=\lceil 1/\eps\rceil$ sets $\sset{\cL_1,\ldots,\cL_\ell}$, and we show that, unless there exists $i$ such that parties in $\cC= \cP\cup \cL_i$ violate agreement (explained below), the following must hold for neighboring executions.

\ifdefined\IsFullVersion\else
\vspace{-.1cm}
\fi
\begin{align} \label{eq:nei2}
&\text{Neighbouring executions (N2):} \quad \ham(\vz,\vo)\le n/3 \implies \\
 & \pr{\Pi(\vz)=\o\text{ in two rounds}} \ge \pr{ \Pi(\vo)=\o\text{ in two rounds}} - 2  (\ell +1  )^2  \cdot (1-\gamma). \nonumber
\end{align}
That is, for two input vectors that are at most $n/3$--far, the difference in probability that two distinct executions (for each input vector) yield the same output within two rounds is roughly upper-bounded by the quantity $(1-\gamma)/\eps^2$ (\ie non-halting probability divided by $\eps^2$). To see that \cref{eq:nei2} holds true, fix $\vz,\vo\in \zn$ of hamming distance at most $n/3$, and let $\cP$ be the set of indices where $\vz$ and $\vo$ differ. Consider the following $\ell+1$ distinct variants of $\Pi$, denoted $\set{\Pi_0,\ldots, \Pi_\ell}$; in protocol $\Pi_i$, parties in $\cP$ send messages to $\cL_1,\ldots, \cL_i$ according to the input prescribed by $\vo$ and to $\cL_{i+1},\ldots, \cL_\ell$ according to the input prescribed by $\vz$, respectively. All other parties follow the instructions of $\Pi$ for input $\vz$. We write $\Pi_i=\o$ to denote the event that the parties not in $\cP$ output $\o$. Notice that the endpoint executions $\Pi_0$ and $\Pi_{\ell}$ are identical to honest executions with input $\vz$ and $\vo$, respectively. Let $\halt_i$ denote the event that the parties not in $\cP$ halt at the second round in an execution of $\Pi_{i}$. We point out that $\pr{ \neg \halt_i}\le (\ell+1)\cdot (1-\gamma) $, since otherwise the adversary corrupting $\cP$ and running $\Pi_i$, for a random $i\in (\ell) = \set{0,\ldots,\ell}$, prevents halting with probability greater than $1-\gamma$. Next, we inductively show that
\ifdefined\IsFullVersion\else
\vspace{-.2cm}
\fi
\begin{align}
\pr{\Pi_i=\o \land \halt_i} \ge \pr{  \Pi_{0} =\o \land \halt_0} - 2i\cdot (\ell+1) \cdot (1-\gamma),\label{eq:teke}
\end{align}
 for every $i\in (\ell)$, which yields the desired expression for $i=\ell$. In pursuit of contradiction, assume \cref{eq:teke} does not hold, and let $i$ denote the smallest index for which it does not hold (observe that $i\ne 0$, by definition). Notice that
\ifdefined\IsFullVersion\else
\vspace{-.1cm}
\fi
\begin{align*}
& \hspace*{-2cm}\pr{(\Pi_{i-1}= \o  \land \halt_{i-1}) \land  (\Pi_{i }\neq \o\land  \halt_i)}\\
&\ge  \pr{\Pi_{i-1}=\o \land \halt_{i-1}}-\pr{ \Pi_{i }= \o\vee  \neg \halt_i}\\
&\ge  \pr{\Pi_{i-1}=\o \land \halt_{i-1}}-\pr{ \Pi_{i }= \o\land   \halt_i} - \pr{ \neg \halt_i}\\
&>    2\cdot (\ell+1) \cdot (1-\gamma) - \pr{ \neg \halt_i} \\
&= (\ell+1) \cdot (1-\gamma) \ge 0.
\end{align*}
\nnote{penultimate inequality follows from union bound and $A\lor \neg B\equiv (A\land B) \lor \neg B$, last inequality is induction hypothesis, and the equality at the end follows from the bound on the non-halting prob.}

It follows that an adversary corrupting $\cC=\cP \cup \cL_i$ causes disagreement with non-zero probability by acting as follows: parties in $\cP$ and $\cL_i$ send messages according to $\Pi_i$ and $\Pi_{i-1}$ to $\oC_0$ and $\oC_1$, respectively, where $\sset{\oC_0,\oC_1}$ is an arbitrary partition of $\oC= [n]\setminus \cP\cup \cL_i$. Since disagreement is ruled out by assumption, we deduce \cref{eq:teke,eq:nei2}. To conclude, we combine the supermajority property (\cref{eq:maj}) with the neighboring property (\cref{eq:nei2}) with $(\vz,\vo)=(0^{n-t}1^{t}, 0^{t}1^{n-t})$ and $\o=1$. Namely, $\pr{\Pi(\vz)=1\text{ in two rounds}}=0$, by supermajority, and $\pr{\Pi(\vo)=1^n\text{ in two rounds}}\ge\gamma$, by supermajority and halting. It follows that $ 0 \ge \gamma - 2  (\ell +1  )^2  \cdot (1-\gamma)$, by \cref{eq:nei2}, and thus $1-\frac{1}{2(\ell+1)^2+1} \ge \gamma$, which yields the desired expression.

\remove{
twice; once with $(\vz,\vv^\star)=(0^{n-t}1^{t},0^{\lfloor n/2\rfloor} 1^{ \lceil n/2 \rceil}) $ and $\o=0$ and once with $(\vv^\star,\vo)=(0^{\lfloor n/2\rfloor} 1^{ \lceil n/2 \rceil}, 0^{t}1^{n-t})$ and $\o=1$.   Namely, for both $\o\in\zo$, we have $\pr{\Pi(\vv^\star)=\o^n\text{ in two rounds}} \ge \pr{ \Pi(\vo)=1^n\text{ in two rounds}} - 2  (\ell +1  )^2  \cdot (1-\gamma) \ge \gamma - 2  (\ell +1  )^2  \cdot (1-\gamma)$ and thus $1-(\eps/5)^2\ge 1-\frac{1}{2(\ell+1)^2+1} \ge \gamma$.
}

\ifdefined\IsFullVersion\else
\vspace{-.2cm}
\fi
\subsection{Second-Round Halting -- Public-Randomness  Protocols}\label{sec:technique:3}

In \cref{sec:technique:2}, we ruled out ``very good'' second-round halting for arbitrary protocols via an efficient locally consistent attack. Recall that if the halting probability is too good (probability almost one), then there is a somewhat simple attack that violates agreement and/or validity. In this subsection, we discuss ruling out \emph{any} second-round halting, \ie halting probability bounded away from zero, for public-randomness protocols.

We first explain why the attack -- as is -- does not rule out second-round halting. Assume that at the first round, the parties of $\Pi$ send a deterministic function of their input, and at the second round  they send the messages they received at the first round together with a uniform random bit. On input $\vv$ and randomness $\vr$, the parties are instructed \emph{not} to halt at the second round, if a supermajority ($\ge n-t$) of the $v_i$'s are in agreement and $\maj(r_1,\ldots, r_n)\neq \maj(v_1,\ldots, v_n)$,  \ie the majority of the random bits does not agree with the supermajority of the inputs. In all other cases, the parties are instructed to output $\maj(r_1,\ldots, r_n)$. It is not hard to see that this protocol will halt with probability $1/2$, even in the presence of the previous locally consistent adversary (regardless of the choice of propagating parties $\cL_i$). More generally, if the randomness uniquely determines the output, the protocol designer can ensure that halting does not result in disagreement, by partitioning the randomness appropriately, and thus foiling the previous attack.\footnote{In \cref{sec:technique:2}, halting was close to $1$ and thus the randomness was necessarily ambiguous regarding the output.}


To overcome the above apparent obstacle, we introduce another dimension to our locally consistent attack; we instruct an extra set of corrupted parties to abort at the second round without sending their second-round messages. By utilizing aborting parties, the adversary can potentially decouple the output/halting from the parties' randomness and thus either prevent halting or cause disagreement.
\ifdefined\IsFullVersion
In \cref{sec:technique:3:1}, we explain
\else
We explain below
\fi
how to rule out second-round halting for a rather unrealistic class of public-randomness protocol. What makes the class of protocols unrealistic is that we assume security holds  against unbounded locally consistent adversaries, and the protocol prescribes only a single bit of randomness per party per round. That being said, this case illustrates nicely our attack, and it also makes an interesting connection to Boolean functions analysis (namely, the  KKL theorem~\cite{KKL88}).  For  general public-randomness protocols, we only  know how to analyze the aforementioned attack assuming
\ifdefined\IsFullVersion
\cref{con:intro:IsoBot}, as explained in   \cref{sec:technique:3:2}.
\else
\cref{con:intro:IsoBot}.
\fi



\ifdefined\IsFullVersion\else
\vspace{-.1cm}
\fi
\ifdefined\IsFullVersion
\subsubsection{``Superb'' Single-Coin Protocols}\label{sec:technique:3:1}
\else
\paragraph{``Superb'' Single-Coin Protocols.}%\label{sec:technique:3:1}
\fi
A BA protocol $\Pi$ is $t$-\emph{superb} if  agreement and validity hold perfectly against an adaptive \emph{unbounded} locally consistent adversary corrupting at most $t$ parties, \ie  the probability that such an adversary violates agreement or validity
%via a locally consistent attack
is $0$.  A  public-randomness protocol is \emph{single-coin}, if, at any given round, each party samples a single unbiased bit.

\begin{theorem}[Second-round halting, superb single-coin protocols]\label{bound:KKL}
For every $\eps> 0$ there exists $c>0$ such that the following holds for large enough $n$. For $t=(1/3+\eps) n$, let $\Pi$ be a $t$-superb, single-coin, $n$-party public-randomness Byzantine agreement protocol and let $\gamma$ denote the probability that the protocol halts in the second round under a locally consistent attack. Then, $\gamma\le n^{-c}$.
\end{theorem}

%Recall that a protocol admits public randomness if the parties' messages consist of their randomness together with some arbitrary function of their view.

We assume for simplicity  that the parties do not sample any randomness at the first round, and  write $\vr\in \zn$ for the vector of bits sampled by the parties at the second round, \ie $\vr_i$ is a uniform random bit sampled by $\Party_i$.




As discussed above, out attack uses an additional set of corrupted parties of size $\sigma\cdot n$, dubbed the ``aborting'' parties and denoted $\cS$, that abort indiscriminately at the second round (the value of $\sigma$ is set to $\lfloor \eps/4 \rfloor$ and $\ell=2\cdot \lceil1/\eps\rceil$ to accommodate for the new set of corrupted parties, \ie $\size{\cL_i}\le n\cdot \eps /2$). In more detail, analogously to the previous analysis, we consider $(\ell+1)\cdot \binom{n}{\sigma n}$ distinct variants of $\Pi$, denoted $\sset{\Pi^{\cS}_i}_{i,\cS}$ and indexed by $i\in (\ell)$ and $\cS \subseteq[n]$ of size $\sigma n$, as follows. In protocol $\Pi_i^\cS$, parties in $\cP$ send messages to $\cL_1,\ldots, \cL_i$ according to the input prescribed by $\vo$, and to $\cL_{i+1},\ldots, \cL_\ell$ according to the input prescribed by $\vz$ (recall that $\cP$ is exactly those indices where $\vz$ and $\vo$ differ). Parties in $\cS$ act according to $\cP$ or $\cL_j$, for the relevant $j$, except that they abort at the second round without sending their second round messages. We write $\Pi^{\cS}_i(\vr)=\o$ to denote the event that the parties not in $\cP\cup \cS$ output $\o$, where the parties' second-round randomness is equal to $\vr$. Let $\halt^{\cS}_i$ denote the event that all parties not in $\cP\cup \cS$ halt at the second round in an execution of $\Pi^{\cS}_i$, and define $\cR_{i }^{\cS}(\o)=\sset{\vr\in \zn\st \Pi^{\cS}_i(\vr)=\o \land \halt^{\cS}_i}$. The following holds:

\begin{align}\label{eq:nei3}
&\text{Neighbouring executions (N2$\dagger$):}\\
&\quad \forall \vz,\vo \in \zn  \text{ with }  \ham(\vz,\vo)\le t-\eps n,   \quad  \forall   \o\in \zo,i\in  [\ell] = \set{1,\ldots,\ell}\colon\nonumber\\
&\qquad\qquad \left (\forall \cS\colon \pr{\Pi^\cS_{i-1}=\o \land \halt^{\cS}_{i-1}}  \ge \gamma/2 \right)  \implies \left(\forall \cS\colon\pr{\Pi^\cS_{i}=\o \land \halt^{\cS}_{i}}\ge \gamma/2\right). \nonumber
\end{align}

\noindent
In words, for both $\o\in \zo$: if $\Pi_{i-1}^\cS=\o$ and halts in two rounds with large probability ($\ge \gamma/2$), for every $\cS$, then $\Pi_i^\cS=\o$ and halts in two rounds with large probability, for every $\cS$.  Before proving \cref{eq:nei3}, we show how to use it to derive \cref{bound:KKL}. We apply \cref{eq:nei3} for $(\vz,\vo)=(0^{n-t}1^{t}, 0^{t}1^{n-t})$, $\o=0$ and $i=\ell$, in combination with the properties of validity and supermajority, \cref{eq:maj}. Namely, by validity and supermajority, a random execution of $\Pi$ on input $\vz$ where the parties in $\cS$ abort at the second round yields output $0$ with probability at least $\gamma/2$, for every $\cS\in {\binom{[n]}{\sigma n}}$. Therefore, applying \cref{eq:nei3} for $(\vz,\vo)=(0^{n-t}1^{t}, 0^{t}1^{n-t})$, $\o=0$ and $i=\ell$, we deduce that a random execution of $\Pi$ on input $\vo$ where the parties in $\cS$ abort at the second round yields output $0$ with probability at least $\gamma/2$, for every $\cS\in {\binom{[n]}{\sigma n}}$. The latter violates either supermajority or validity -- contradiction. We conclude the proof of \cref{bound:KKL} by proving \cref{eq:nei3}.
We prove \cref{eq:nei3} using the following  corollary of the seminal KKL theorem \cite{KKL88} from
\citet{BKK14}.
%\citeauthor{BKK14}~\cite{BKK14}.
(Recall that $\cR|_{\wb{\cS}}$ is the projection of $\cR$ on the index-set $\wb{\cS}$.)

\ifdefined\IsFullVersion\else
\vspace{-.1cm}
\fi
\begin{lemma}\label{lem:KKL} For every $\sigma,  \delta\in (0,1)$, there exists $c>0$ s.t.\
%such that
the following holds for large enough~$n$. Let $\cR \subseteq \zn$ be s.t.\
%such that
$\ssize{\cR|_{\wb{\cS}}}\le (1-\delta)\cdot 2^{(1-\sigma)n}$,  for every $\cS\subseteq [n]$ of size $\sigma n$. Then,  $\size{\cR}\le n^{-c}\cdot 2^n$.
\end{lemma}

\ifdefined\IsFullVersion\else
\vspace{-.1cm}
\fi
Loosely speaking, \cref{lem:KKL} states that for a set $\cR\subseteq \zn$, if the size of every projection on a constant fraction of indices is bounded away from one (in relative size), then the size of $\cR$ is vanishingly small
\ifdefined\IsFullVersion
(again, in relative size).\footnote{In the jargon of Boolean functions analysis, since every large set has a $o(n)$-size index-set of influence almost one, it follows that some projection on a constant fraction of indices is almost full.}
\else
(again, in relative size).
\fi









Going back to the proof, suppose \cref{eq:nei3} does not hold for $b=0$,  \ie there exists $\cS$ such that $\ssize{\cR_{i}^{\cS}(0)}< \gamma/2 \cdot 2^{n}$, and $\ssize{\cR_{i-1}^{\cS'}(0)}\ge \gamma/2 \cdot 2^n$, for every relevant $\cS'$.  We prove \cref{eq:nei3} by proving \cref{eq:halt,eq:proj}, which result in contradiction via \cref{lem:KKL}.

\begin{align}
\text{Halting:}&\qquad  \ssize{\cR_{i }^{\cS}(1)}\ge \gamma/2 \cdot 2^{n}  \label{eq:halt} \\
\text{Perfect agreement:}&\qquad   \forall \cS' \colon \quad \ssize{\cR_{i}^{\cS}(1)|_{\wb{\cS}'}}\le (1-\gamma/2)\cdot 2^{(1-\sigma) n}   \label{eq:proj}
\end{align}
\noindent
\Inote{too technical? \rnote{yes!}}
\cref{eq:halt} follows by the halting property of $\Pi_{i}^\cS$, since the execution halts if and only if $\vr\in \cR_{i }^{\cS}(1)\cup \cR_{i }^{\cS}(0)$, and, by assumption, $\ssize{\cR_{i }^{\cS}(0)}< \gamma/2 \cdot 2^{n}$. \cref{eq:proj} follows from $\ssize{\cR_{i-1}^{\cS'}(0)}\ge \gamma/2 \cdot 2^{n}$ (by assumption), and $\cR_{i }^{\cS}(1)|_{\wb{\cS}'} \cap \cR_{i-1}^{\cS'}(0)|_{\wb{ \cS}'}=\emptyset$, for every $\cS'$. The latter holds since $\vr|_{\wb{\cS}'}=\vr'|_{\wb{\cS}'}$ and $\vr'\in \cR_{i-1}^{\cS'}(0)$ implies $\vr\in \cR_{i-1}^{\cS'}(0)$, and by considering the attacker controlling $\cP$, $\cL_{i}$, $\cS$ and $\cS'$ and sending messages according to $\Pi_{i
}^{\cS}$ and $\Pi_{i-1}^{\cS'}$ to $\oC_0$ and $\oC_1$, respectively, where $\sset{\oC_0,\oC_1}$ is an arbitrary partition of $\oC=[n]\setminus \cP\cup \cL_{i} \cup \cS\cup \cS'$.


%\nnote{Recall validity: Since $\wt{\cR}^\cS_{0}(0)\subseteq \cR_0^\cS(0)$, where $\wt{\cR}^\cS_{0}(0)$ is the set that is insensitive to the abort}




\begin{remark}
For superb, single-coin, public-randomness protocol, repeated application of \cref{eq:nei,lem:KKL} rules out  second-round halting for arbitrary (constant) fraction of corrupted parties  (and not only  $n/3$ fraction).
\end{remark}


\ifdefined\IsFullVersion\else
\vspace{-.3cm}
\fi
\ifdefined\IsFullVersion
\subsubsection{General (Public-Randomness) Protocols}\label{sec:technique:3:2}
\else
\paragraph{General (Public-Randomness) Protocols.}\label{sec:technique:3:2}
\fi
The analysis above crucially relies on the superb properties of the protocol. While it can be generalized for protocols with near-perfect statistical security and constant-bit randomness, we only manage to analyze the most general case (\ie protocols with non-perfect computational security and arbitrary-size randomness) assuming  \cref{con:intro:IsoBot}.  Very roughly (and somewhat inaccurately), when applying the above attack on general public-randomness protocols, the following happens for  some $\delta>0$ and both values of $\o\in \zo$:   for $(1-\delta)$-fraction of possible aborting subsets $\cS$, the probability that the honest parties  halt in two rounds and output the same value $\o$, whether parties in $\cS$ all abort or not,  is at least  $\lambda$ (\ie the halting probability). Assuming \cref{con:intro:IsoBot},  the above yields that with probability $\delta$ over the randomness and $\cS$, the honest parties under the attack output opposite values depending whether  the parties in $\cs$ abort or not. It thus follows that the agreement of the protocol is at most $\delta$. We refer the reader to \cref{sec:TwoRoundProtcol:PR} for the full details.


\remove{

\Inote{remove  from submission}

For reference, we restate here the conjecture (see \cref{sec:intro:ourResult} for a further discussion about the conjecture itself.). Recall that for $\vx\in \sn$ and $\cS \subseteq [n]$, vector $\bot_\cS(\vx) \in \set{\Sigma \cup \bot}^n$ is obtained by assigning all entries indexed by $\cS$ with the value $\bot$, and all other entries according to $\vx$, and $\bns$ denotes the distribution induced over subsets of $[n]$ by choosing each element with probability $\sigma$ independently at random.


\begin{conjecture}[Restatement of \cref{con:intro:IsoBot}]
\MainConj
\end{conjecture}

} 

\section{Our Lower Bounds}\label{sec:OurResult}

In this section, we formally state our lower bounds on the round complexity of Byzantine agreement protocols. The communication and adversarial models as well as the notion of Byzantine agreement protocols we consider are given in \cref{sec:OurResults:Model}, and our bounds are formally stated in \cref{sec:OurResult:Bounds}.

\subsection{The Model}\label{sec:OurResults:Model}

\subsubsection{Protocols}
All protocols considered in this paper are \ppt (probabilistic polynomial time): the running time of every party is polynomial in the (common) security parameter (given as a unary string). We only consider Boolean-input Boolean-output protocols: apart from the common security parameter, all parties have a single input bit, and each of the honest parties outputs a single bit. For an $n$-party protocol $\Pi$, an input vector $\vv\in \zn$ and randomness $\vr$, let $\Pi(\vv; \vr)$ denote the output vector of the parties in an (honest) execution with party $\Party_i$'s input being $\vv_i$ and randomness $\vr_i$.
For a set of parties $\cP \subseteq [n]$, we denote by $\Pi(\vv; \vr)_\cP$ the output vector of the parties in $\cP$.

The protocols we consider might have a \emph{setup phase} in which before interaction starts a trusted party distributes (correlated) values between the parties. We only require the security to hold for a \emph{single} use of the setup parameters (in reality, these parameters are set once and then used for many interactions). This, however, only makes our lower bound stronger.

The communication model is \emph{synchronous}, meaning that the protocols proceed in rounds. In each round every party can send a message to every other party over a private and authenticated channel. (Allowing the protocol to be executed over private channels makes our lower bounds stronger.) It is guaranteed that all of the messages that are sent in a round will arrive at their destinations by the end of that round.

\subsubsection{Adversarial Model}\label{sec:OurResults:Adv}

We consider both \adaptive and \nonadaptive (\aka static) adversaries. An \adaptive adversary can choose which parties to corrupt for the next round immediately after the conclusion of the previous round but before seeing the next round's messages. If a party has been corrupted then it is considered corrupt for the rest of the execution. A \nonadaptive (static) adversary chooses which parties to corrupt \emph{before} the execution of the protocol begins (\ie before the setup phase, if such exists). We measure the success probability of the latter adversaries as the expectation over their choice of corrupted parties.

We consider both \emph{rushing} and \emph{non-rushing} adversaries. A non-rushing adversary chooses the corrupted parties' messages in a given round based on the messages sent in the \emph{previous} rounds. In contrast, a rushing adversary can base the corrupted parties' messages on the messages sent in the previous rounds, and on those sent by the honest parties in the \emph{current} round.

\paragraph{Locally consistent adversaries.}
As discussed in \cref{sec:intro:model}, our attack requires very limited capabilities from each corrupted party: to prematurely abort, and to lie about its input bit and incoming messages from other corrupted parties. In particular, a corrupted party tosses its local coins honestly and does not lie about incoming messages from honest parties. We now present the formal definition.

\begin{definition}[locally consistent adversaries]\label{def:Semi-ConssitentParties}
Let $\Pi=(\Pc_1,\ldots,\Pc_n)$ be an $n$-party protocol and let $\sset{\nxtmsg_{i,i'}^j}_{i,i' \in [n],j\in \N}$ be its set of next-message functions, \ie
\[
%\alpha_{i,i'}^j(b,r,(m^1_{1},\ldots,m^1_{n}),\ldots, (m^{j-1}_{1},\ldots,m^{j-1}_{n}))
m^j_{i,i'}=\nxtmsg_{i,i'}^j\left(b;r;(m^1_{1,i},\ldots,m^1_{n,i}),\ldots, (m^{j-1}_{1,i},\ldots,m^{j-1}_{n,i})\right)
\]
is the message party $\Pc_i$ sends to party $\Pc_{i'}$ in the \jth round, given that its input bit is $b$, the random coins it flipped till now are $r$, and in round $j' < j$, it got the message $m^{j'}_{i'',i}$ from party $\Pc_{i''}$. An adversary taking the role of $\Pc_i$ is said to be {\sf locally consistent} \wrt $\Pi$, if it \emph{flips its random coins honestly}, and the message it sends in the \jth round to party $\Pc_{i'}$ takes one of the following two forms:

\begin{description}
	\item[Abort:] the message $\perp$.
	
	\item[Input and message selection:] a set of messages $\set{m_\ell}_{\ell=1}^k$, for some $k$, such that for each $\ell \in [k]$:
\[
m_\ell = \nxtmsg_{i,i'}^j\left(b_\ell;r;((m^1_{1})_1,\ldots,(m^1_{n})_1),\ldots, ((m^{j-1}_{1})_\ell,\ldots,(m^{j-1}_{n})_\ell)\right),
\]
where $b_\ell\in \zo$, $r$ are the coins $\Pc_i$ tossed (honestly) until now, and $(m^{j'}_{i''})_\ell$, for each $j'<j$ and $i''\neq i$, is one of the messages it received from party $\Pc_{i''}$ in the \iith{j} round.
\end{description}
\end{definition}
That is, a locally consistent party $\Pc_{i}$ might send party $\Pc_{i'}$ a sequence of messages
(and not just one as instructed), each consistent with a possible choice of its input bit, and some of the messages it received in the previous round. In turn, this will enable party $\Pc_{i'}$, if corrupted, the freedom to choose in the next rounds the message of $\Pc_{i}$ it would like to act according to. Note that \wlg, $\Pc_{i}$ will always sends a single message to the honest parties, as otherwise they will discard the messages.

A few remarks are in place.
\begin{enumerate}
	\item While the above definition does not enforce between-rounds consistency (a party might send to another party a first-round message consistent with input $0$ and a second-round message consistent with input $1$), compiling a given protocol so that every message party $\Pc_{i}$ sends to $\Pc_{i'}$ contains the previous messages $\Pc_{i}$ sent to $\Pc_{i'}$, will enforce such between-rounds consistency on locally consistent parties.
	
	\item Using standard cryptographic techniques, a protocol secure against locally consistent adversaries can be compiled into one secure against arbitrary malicious adversaries, without hurting the efficiency and round complexity of the protocol ``too much.'' If the protocol is \emph{public randomness} (see \cref{def:PR}) this reduction can be made extremely efficient, and in particular preserve the round complexity (see \cref{sec:intro:LocalToFull}).
	
	\item The locally consistent parties considered in \cref{sec:FirstRound,sec:SecondRound} do not take full advantage of the generality of \cref{def:Semi-ConssitentParties}. Rather, the parties considered either act honestly but abort at the conclusion of the first round, cheat in the first round and then abort, or cheat only in the second round and then abort.
\end{enumerate}

\subsubsection{Public-Randomness Protocols}\label{sec:OurResults:PR}
In \cref{sec:intro:model}, we showed that the description of many natural protocols can be simplified when security is required to hold only against locally consistent adversaries. In this relaxed description a trusted setup phase and cryptographic assumptions are not required, and every party can publish the coins it locally tossed in each round.

\begin{definition}[Public-randomness protocols]\label{def:PR}
A protocol has {\sf public randomness}, if every party's message consists of two parts: the randomness it sampled in that round, and an arbitrary message which is a function of its view (input, incoming messages, and coins tossed up to and including that point). The party's first message also contains its setup parameters, if such exist.
\end{definition}

\subsubsection{Byzantine Agreement}\label{sec:BA}

We now formally define the notion of Byzantine agreement. Since we focus on lower bounds we will consider only the case of a single input bit and a single output bit. A more general notion of Byzantine agreement will include string input and string outputs. A generic reduction shows that the cost of agreeing on strings rather than bits is two additional rounds~\cite{TC84}.

\begin{definition}[Byzantine Agreement]\label{def:BA}
We associate the following properties with a \ppt $n$-party Boolean input/output protocol $\Pi$.

\begin{description}
	\item[Agreement.] Protocol $\Pi$ has {\sf $(t,\alpha)$-\Agr}, if the following holds \wrt any \ppt adversary controlling at most $t$ parties in $\Pi$ and any value of the non-corrupted parties' input bits: in a random execution of $\Pi$ on sufficiently large security parameter, all non-corrupted parties output the \emph{same} bit with probability at least $1-\alpha$.\footnote{A more general definition would allow the parameter $\alpha$ (and the parameters $\beta,\gamma$ below) to depend on the protocol's security parameter. But in this paper we focus on the case that $\alpha$ is a fixed value.}
	
	\item[Validity.] Protocol $\Pi$ has {\sf $(t,\beta)$-\Vld}, if the following holds \wrt any \ppt adversary controlling at most $t$ parties in $\Pi$ and an input bit $b$ given as input to all non-corrupted parties: in a random execution of $\Pi$ on sufficiently large security parameter, all non-corrupted parties output $b$ with probability at least $1-\beta$.
	
	\item[Halting.] Protocol $\Pi$ has {\sf $(t,q,\gamma)$-\Halt}, if the following holds \wrt any \ppt adversary controlling at most $t$ parties in $\Pi$ and any value of the non-corrupted parties' input bits: in a random execution of $\Pi$ on sufficiently large security parameter, all non-corrupted parties halt within $q$ rounds with probability at least $\gamma$.
\end{description}
	
\noindent
Protocol $\Pi$ is a $(t,\alpha,\beta,q,\gamma)$-\BA, if it has $(t,\alpha)$-\Agr, $(t,\beta)$-\Vld, and $(t,q,\gamma)$-\Halt. If the protocol has a setup phase, then the above probabilities are taken \wrt this phase as well.
\end{definition}

\begin{remark}[Concrete security]
Since we care about fixed values of a protocol's characteristics (\ie agreement), the role of the security parameter in the above definition is to enable us to bound the running time of the parties and adversaries in consideration in a meaningful way, and to parametrize the cryptographic tools used by the parties (if there are any). Since the attacks we present are efficient assuming the protocol is efficient (in any reasonable sense), the bounds we present are applicable for a fixed protocol that might use a \emph{fixed} cryptographic primitive, \eg SHA-256.
\end{remark}

\subsection{The Bounds}\label{sec:OurResult:Bounds}
We proceed to present the formal statements of the three lower bounds.
%\rnote{do we need to restate the bounds here? We have the informal statements in the intro and the formal statements in the body.. \Inote{why not}}

\paragraph{First-round halting, arbitrary protocols.}
The first result bounds the halting probability of arbitrary protocols after a single round. Namely, for ``small'' values of $\alpha$ and $\beta$, the halting probability is ``small'' for $t\geq n/3$ and ``close to $1/2$'' for $t\geq n/4$.

\def\ThmFirstRoundArb
{
Let $\Pi$ be a \ppt $n$-party protocol that is $(t,\alpha,\beta,1,\gamma)$-\BA against locally consistent, static, non-rushing adversaries. Then,
\begin{itemize}
	\item $t \ge n/3$ implies $\gamma \le 5\alpha + 2\beta +\err$
	\item $t \ge n/4$ implies $\gamma \le 1/2 + 5\alpha + \beta + \err$,
\end{itemize}
for $\err= \FirstErr$ ($\err=0$ for public-randomness protocols whose security holds against rushing adversaries).
}
\begin{theorem}[restating \cref{thm:intro:FirstRound}]\label{thm:FirstRound:Arb}
	\ThmFirstRoundArb
\end{theorem}
%For $n\ge6$, taking $t = \min\set{t,\ceil{n/2}-1}$, the error term $\err$ in \cref{thm:FirstRound:Arb} is bounded by $2^{-\floor{n/2}}$
\paragraph{Second-round halting, arbitrary protocols.}
The second result bounds the halting probability of arbitrary protocols after two rounds.

\def\ThmSecondRoundArb
{
Let $\Pi$ be a \ppt $n$-party protocol that is a $(t,\alpha,\beta,2,\gamma)$-\BA against locally consistent, static, non-rushing adversaries for $t > n/4$. Then $\gamma \le 1 + 2\alpha + \frac\beta{w^2} -\frac1{2w^2}$ for $w = \ceil{(n-\ceil{n/4})/\floor{t - n/4}}+1$.
}
\begin{theorem}[restating \cref{thm:intro:SecondRound:Arb}]\label{thm:SecondRound:Arb}
	\ThmSecondRoundArb
\end{theorem}

In particular, for $t = (1/4 + \eps)\cdot n$ and ``small'' $\alpha$ and $\beta$, the protocol might not halt at the conclusion of the second round with probability $\approx 1/\eps^2$.


\paragraph{Second-round halting, public-randomness protocols.}
The third result bounds the halting probability of public-randomness protocols after two rounds. The result requires adaptive and rushing adversaries, and is based on \cref{con:IsoBot} (stated in \cref{sec:Iso} below).

\newcommand{\ept}{\eps_t}
\newcommand{\epg}{\eps_\gamma}

\def\ThmSeconRoundPR
{
Assume \cref{con:IsoBot} holds. Then, for any (constants) $\ept,\epg>0$ there exists $\alpha> 0$ such that the following holds for large enough $n$: let $\Pi$ be a \ppt $n$-party, public-randomness protocol that is $(t,\alpha,\beta= \epg^2/200,2,\gamma)$-\BA against locally consistent, rushing, adaptive adversaries. Then,
	\begin{itemize}
		\item $t \ge (1/3 + \ept)\cdot n$ implies $\gamma < \epg$.
		\item $t \ge (1/4 + \ept)\cdot n$ implies $\gamma < \frac12 + \epg$.
	\end{itemize}
}

\begin{theorem}[restating \cref{thm:intro:SecondRound:PR}] \label{thm:SecondRound:PR}
\ThmSeconRoundPR
\end{theorem}
In particular, assuming the protocol has perfect agreement and validity, the protocol never halts in two rounds if the fraction of corrupted parties is greater than $1/3$, and halts in two rounds with probability at most $1/2$ if the fraction of corrupted parties is greater than $1/4$.

The value of $\alpha$ in the theorem is (roughly) $\delta\cdot \ept\cdot \epg^2$ where $\delta$ is the constant guaranteed by \cref{con:IsoBot}. We were not trying to optimize over the constants in the above statement, and in particular it seems that $\beta$ can be pushed to $\epg^2$.



\subsection{The Combinatorial Conjecture}\label{sec:Iso}
Next, we provide the formal statement for the combinatorial conjecture used in \cref{thm:SecondRound:PR}.
For $n\in \N$ and $\sigma \in [0,1]$, let $\bns$ be the distribution induced on the subsets of $[n]$ by sampling each element independently with probability $\sigma$.
For a finite alphabet $\Sigma$, a vector $\vx\in \sn$, and a subset $\cS \subseteq [n]$, define the vector $\bot_\cS(\vx) \in \sn$ by
\[
\bot_\cS(\vx)_i=
\begin{cases}
\bot, & i\in \cs,\\
\vx_i, & \text{otherwise}.
\end{cases}
\]

%\begin{notation}\label{not:IsoBot}
%	For $\vx\in \sn$ and $\cS \subseteq [n]$, define  $\bot_\cS(x) \in \sn$ by $
%	\bot_\cS(x)_i=
%	\begin{cases}
%	\bot, & i\in \cs,\\
%	x_i, & \text{otherwise}.
%	\end{cases}
%	$.
%\end{notation}

\begin{conjecture}[restating \cref{con:intro:IsoBot}]\label{con:IsoBot}
For any $\sigma,\lambda >0$ there exists $\delta>0$ such that the following holds for large enough $n\in \N$. Let  $\Sigma$ be a finite alphabet and let $\cA_0,\cA_1 \subseteq \sbn$  be  two  sets such that for both $b\in \zo$:
\begin{align*}%\label{eq:1}
\ppr{\cs\gets \bns}{\ppr{\vr \gets \Sigma^n}{\vr,\bot_{\cS}(\vr) \in \cA_b} \ge  \lambda } \ge 1-\delta.
\end{align*}
Then,
\begin{align*}%\label{eq:2}
\ppr{\substack{\vr\gets \Sigma^n\\ \cS \gets \bns}}{\forall b\in \zo\colon  \set{\vr,\bot_{\cS}(\vr)}  \cap \cA_b \neq \emptyset} \ge  \delta.
\end{align*}
\end{conjecture}


\remove{
\begin{remark}\label{rem:IsoBot}
	We make the following observations.
	\begin{enumerate}
		
		\item The conjecture trivially holds then  the set $\cA_0,\cA_1$ are restricted to hamming balls and prefix sets
		
		\Inote{explain}
		
		\item  For our application, \cref{thm:SecondRound:PR}, we can weaken the conjecture allowing $\delta$ to be noticeable (\ie inverse polynomial) function of $n$ and $\size{\Sigma}$.
		
	
	\end{enumerate}
\end{remark}


\subsubsection{Towards Proving \cref{con:IsoBot}}
\Inote{the following theorem was proven by Alex}
\begin{notation}\label{not:Iso}
	For $n\in \N$, finite set $\Sigma$, $\vr\in \Sigma^n$ and $\cS\subseteq [n]$, let $F_\cS(\vr)$ denote the random variable  defined by
	%	\begin{align*}
	$F_\cS(\vr)_i=\begin{cases} \vr_i & \text{if } i\notin \cS \\ \wt{r}_i\la \Sigma &\text{ otherwise}.\end{cases}$.
	%\end{align*}
\end{notation}

\begin{theorem}[Alex]\label{thm:Iso}
	For every $\sigma, \lambda>0$ there exists $\eps>0$ such that the following holds for large enough $n\in \N$. Let  $\Sigma$ be a finite alphabet, and let $\cB_0,\cB_1 \subseteq \Sigma^n$ such that $\size{\cB_b}\ge \lambda \cdot \size{\Sigma}^n $, for both $b\in \zo$. Then, $$\ppr{\substack{\vr\la \Sigma^n\\ \cS\la \bns}}{\vr\in \cB_0 \land F_\cS(\vr)\in \cB_1}\ge \eps.$$
\end{theorem}

\Inote{I believe that the following variant of \cref{thm:Iso} should do for proving \cref{con:IsoBot}.}
\begin{conjecture}\label{con:Iso}
	For every $\sigma, \lambda,\delta>0$  the following holds for large enough $n\in \N$. Let  $\Sigma$ be a finite alphabet, and let $\cB_0,\cB_1 \subseteq \Sigma^n$ such that $\size{\cB_b}\ge \lambda \cdot \size{\Sigma}^n $, for both $b\in \zo$.
	Then, there exists a distribution $\bns'$ with $\ppr{\cs \gets  \bns'}{\size{\cs} \ge 2\sigma n} \in o(1)$ such that
	$$\ppr{\substack{\vr\la \Sigma^n\\ \cS\la \bns'}}{\vr_\oS\in (\cB_0)_\oS \cap (\cB_1)_\oS}\ge 1-\delta.$$
\end{conjecture}


}



\remove{
	
	
	
	
	
\cref{thmn:Iso} yields the following corollary.

\begin{corollary}\label{cor:IsoBot}
	For any $\sigma,\lambda >0$, there exists $\delta>0$ such that the following holds for large enough $n\in \N$. Let  $\Sigma$ be a finite alphabet, and let $\cA_0,\cA_1 \subseteq \sbn$  be  two  sets such that for both $b\in \zo$:
	
	\begin{align*}%\label{eq:1}
	\ppr{\vr \gets \Sigma^n}{\ppr{\cs\gets \bns}{\vr,\bot_{\cS}(\vr) \in \cA_b} \ge  1-\delta } \ge \lambda.
	\end{align*}
	Then,
	\begin{align*}%\label{eq:2}
	\ppr{\substack{\vr\gets \Sigma^n\\ \cS \gets \bns}}{\forall b\in \zo\colon  \set{\vr,\bot_{\cS}(\vr)}  \cap \cA_b \neq \emptyset} \ge  \delta.
	\end{align*}
\end{corollary}

\begin{proof}
	For $\sigma, \lambda>0$, let $\eps>0$ be the constant guaranteed by \cref{thmn:Iso} for the same $\sigma$, $\lambda$, and fix $\delta=\eps/2$. We assume that $\ppr{\vr \gets \Sigma^n}{\ppr{\cs\gets \bns}{\vr,\bot_{\cS}(\vr) \in \cA_b} \ge  1-\delta } \ge \lambda$, for both $z\in \zo$, and we show that $\ppr{\vr,\cS}{\vr\in \cA_0\land \bot_{\cS}(\vr)\in \cA_1 } \ge \delta$. For $z\in \zo$, we define $\cB_z=\set{\vr\in \Sigma^n \st \ppr{\cs\gets \bns}{\vr,\bot_{\cS}(\vr) \in \cA_z} \ge  1-\delta }$ and notice that $\size{\cB_z}\ge \lambda \cdot \size{\Sigma}^n$, for both $z\in \zo$. Next, observe that
	
	\begin{align}
	&\ppr{\substack{\vr\la \Sigma^n\\ \cS\la \bns}}{\vr\in \cA_0\land \bot_{\cS}(\vr)\in \cA_1 } \ge  \ppr{\substack{\vr\la \Sigma^n\\ \cS\la \bns}}{\vr\in \cB_0\land F_\cS(\vr)\in \cB_1 \land \bot_{\cS}(\vr)\in \cA_1 } =  \nonumber\\
	&\hspace*{2cm}\ppr{\substack{\vr\la \Sigma^n\\ \cS\la \bns}}{\vr\in \cB_0\land F_\cS(\vr)\in \cB_1 } - \ppr{\substack{\vr\la \Sigma^n\\ \cS\la \bns}}{\vr\in \cB_0\land F_\cS(\vr)\in \cB_1 \land \bot_{\cS}(\vr)\notin \cA_1 }\nonumber
	\end{align}
	and, by \cref{thmn:Iso},
	\begin{align}
	&\ppr{\substack{\vr\la \Sigma^n\\ \cS\la \bns}}{\vr\in \cA_0\land \bot_{\cS}(\vr)\in \cA_1 } \ge     \label{eq:low2} \eps - \ppr{\substack{\vr\la \Sigma^n\\ \cS\la \bns}}{\vr\in \cB_0\land F_\cS(\vr)\in \cB_1 \land \bot_{\cS}(\vr)\notin \cA_1 }.
	\end{align}
	Finally, by the definition of $\cB_z$, it holds that $\ppr{\cS}{\bot_{\cS}(\vr)\notin \cA_1 \mid \vr\in \cB_1 }\le \delta $. Thus,
	\begin{align}\label{eq:bad} \ppr{\substack{\vr\la \Sigma^n\\ \cS\la \bns}}{\vr\in \cB_0\land F_\cS(\vr)\in \cB_1 \land \bot_{\cS}(\vr)\notin \cA_1 } \le \ppr{\substack{\vr\la \Sigma^n\\ \cS\la \bns}}{\vr\in \cB_1 \land \bot_{\cS}(\vr)\notin \cA_1 }\le \delta.
	\end{align}
	Given our choice of $\delta$, we combine \cref{eq:low2,eq:bad}, and we conclude that $$\ppr{\substack{\vr\la \Sigma^n\\ \cS\la \bns}}{\vr\in \cA_0\land \bot_{\cS}(\vr)\in \cA_1 } \ge \delta.$$
\end{proof}
\subsection{Proving \cref{thmn:Iso}}
}









\section{Lower Bounds on First-Round Halting}\label{sec:FirstRound}

In this section, we present our lower bound for the probability of first-round halting in Byzantine agreement protocols.
\begin{theorem}[Bound on first-round halting. \cref{thm:FirstRound:Arb} restated]\label{thm:FirstRound:Arb:Res}
\ThmFirstRoundArb
\end{theorem}

Let $\Pi$ be as in \cref{thm:FirstRound:Arb:Res}. We assume for ease of notation that an honest party that runs more than one round outputs $\perp$ (it will be clear that the attack, described  below, does not benefit from this change). Finally, we omit the security parameter from the parties' input list, it will be clear though that the adversaries we present are efficient \wrt the security parameter.


\begin{lemma}[Neighboring executions]\label{lemma:FirstRound:Arb}
	Let $\vv,\vv' \in \zn$ be with $\ham(\vv,\vv') \le t$. Then for both $\o\in \zo$:
	\[
	\pr{\Pi(\vv') \in  \omin^n  \setminus \mon} \ge	\pr{\Pi(\vv) \in \omin^n} - (1- \gamma) - 4\alpha - \err.
	\]
\end{lemma}
Namely, the lemma bounds from below the probability that in a random honest execution of the protocol on input $\vv'$, at least one party halts in the first round while outputting $\o$.

We prove \cref{lemma:FirstRound:Arb} below, but first use it to prove \cref{thm:FirstRound:Arb:Res}. We also make use of the following immediate observation.
\begin{claim}[Supermajority execution]\label{claim:FirstRoundound:Arb:Validity}
Let $\vv \in \zn$  and $\o \in \zo$ be such that $\ham(\vv,\o^n) \le t$. Then,
$\pr{\Pi(\vv) \in \omin^n} \ge 1 - \alpha - \beta$.
\end{claim}
\begin{proof}
Let $\cA \subset [n]$ be a subset of size $n-t$ such that $\vv_\cA = \o^{\size{\cA}}$. The claimed validity of $\Pi$ yields that
\begin{align*}
\pr{\Pi(\vv)_\cA \notin \omin^{\size{\cA}}} < \beta.
\end{align*}
This follows from $\beta$-validity of $\Pi$ and the fact that an honest party cannot distinguish between an execution of $\Pi(\vv)$ and an execution of $\Pi(b^n)$ in which all parties not in $\cA$ act as if their input bit is as in $\vv$. Hence, by the claimed agreement of $\Pi$

\begin{align*}
\pr{\Pi(\vv) \notin \omin^n} < \alpha + \beta.
\end{align*}
\end{proof}




\begin{proof}[Proof of \cref{thm:FirstRound:Arb:Res}]~
We separately prove the theorem  for $t \ge n/3$  and for $t \ge n/4$.

\paragraph{The case $t \ge n/3$.}
We assume for simplicity that $(n-t)/2\in \N$, let $\vz= 0^t 1^{\cnt} 0^{\fnt} $ and let $\vo = 1^t 1^{\cnt} 0^{\fnt}$. Note that $\ham(\vz,\vo) = t$, and that  for both $\o\in \zo$ it holds that $\ham(\vv_\o,\o^n)\le t$. \Inote{the proof if $(n-t/2) \notin \N$ is a bit tedious. Matan, please verify} Hence, by \cref{claim:FirstRoundound:Arb:Validity},
%validity (\cref{claim:FirstRoundound:Arb:Validity}),
for both $\o\in \zo$:
\begin{align}
\pr{\Pi(\vv_\o) \in \omin^n} \ge 1 - \alpha - \beta.
\end{align}
Applying  \cref{lemma:FirstRound:Arb}  to $\vv= \vz$ and $\vv'= \vo$ yields that
	\begin{align*}
\pr{\Pi(\vo) \in \set{0,\perp}^n \setminus \mon} \ge 1 - 5\alpha - \beta - (1- \gamma) - \err,
\end{align*}
yielding that  $5\alpha + 2\beta + (1-\gamma)  + \err\ge 1$.

\paragraph{The case $t \ge n/4$.}  In this case there are no two vectors that are $t$ apart in Hamming distance, and still each of them has $n-t$ entries of opposite values. Rather, we consider the two vectors $\vz= 0^t 0^t 0^t 1^{n-3t}$ and  $\vo= 1^t 1^t 0^t 1^{n-3t}$ of distance $2t$. For both $\o\in \zo$, the vector $\vv_\o$ has at least $n-t$ entries with $\o$ and is of distance $t$ from the vector $\vstar =1^t 0^t 0^t  1^{n-3t}$.
%\rnote{as a result, for every $b\in\zo$, $\pr{\Pi(\vstar) \in \omin^n \setminus \mon}\leq 1/2$ \Inote{I dont think it is needed, but add it if you like}}

As in the first part of the proof, Applying  \cref{claim:FirstRoundound:Arb:Validity,lemma:FirstRound:Arb}  on $\vv_b$ and $\vstar$, for both $\vo\in \zo$, yields that
\begin{align*}
\pr{\Pi(\vstar) \in \omin^n \setminus \mon} \ge 1 - 5\alpha - \beta - (1- \gamma) - \err,
\end{align*}
yielding that $2(5\alpha + \beta + (1- \gamma) + \err)  \ge 1$.
\end{proof}
	
	
\newcommand{\PPf}{\Pi^\cP}	
\subsection{Proving Lemma~\ref{lemma:FirstRound:Arb}}
\begin{proof}[Proof of \cref{lemma:FirstRound:Arb}]
Fix $b \in \zo$ and let $\delta = \pr{\Pi(\vv) \in \omin^n}$.  Let $\cP$ be the coordinates in which $\vv$ and $\vv'$ differ, and let $\oP = n \setminus \cP$. Let $I$ be the index (a function of the parties' coins and setup parameters) of the smallest party in  $\oP$  that halts in the first round   and outputs the same value, both if the parties in $\cP$ send their messages according to input   $\vv$  and if they do that according to $\vv'$. We let  $I=0$  if there is no such party, and (abusing notation)  sometimes identify  $I$  with the event that $I\neq 0$, \eg $\pr{I}$ stands for $\pr{I\ne 0}$. \mnote{Huh?\Inote{better?}}  Clearly,


%and let $\gamma'$ be the probability that $\Pc_1$ halts in the first round of $\Pi(\vv)$. Recall that by our convention, an honest  party outputs value in $\zo$ iff it halts in the first round.

\begin{align*}
\delta \le  \pr{\Pi(\vv) \in  \omin^n \qand  I}+ (1- \pr{I})
\end{align*}
%Since $\gamma' \ge \gamma$,
and thus
\begin{align}\label{eq:FirstRound:Arb:1}
\pr{\Pi(\vv) \in  \omin^n \qand   I} \ge \delta - (1-\pr{I}).
\end{align}

	
 It follows that
\begin{align}\label{eq:FirstRound:Arb:2}
\pr{\Pi(\vv')  \in \omin^n \setminus \mon} &\ge  \pr{\Pi(\vv') \in  \omin^n \qand   I}\\
&=\pr{\Pi(\vv') \in  \omin^n \qand   \Pi(\vv')_I= b}\nonumber\\
&\ge \pr{\Pi(\vv')_I= b} -  \alpha\nonumber\\
&= \pr{\Pi(\vv)_I  =  \o} -  \alpha\nonumber\\
&\ge \pr{\Pi(\vv) \in  \omin^n \land   \Pi(\vv)_I  =  \o} - 2\alpha\nonumber\\
&= \pr{\Pi(\vv) \in  \omin^n \land   I} - 2\alpha\nonumber\\
&\ge \delta - (1-\pr{I}) - 2\alpha.\nonumber
\end{align}
The  first inequality and the equalities hold by the definition of $I$.  The second  and third  inequalities  hold by agreement, and  the last inequality holds by  \cref{eq:FirstRound:Arb:1}. We conclude the proof showing that:
\begin{align}\label{claim:FirstRound:Arb}
	\pr{I} \ge \gamma - \err  - 2\alpha.
\end{align}

  Let $E_h$ be the event that  each party  in $\oP$ either does not halt when the parties in $\cP$ act according to $\vv$ or does not halt when they act according to $\vv'$. Let $E_a$ be the event that $E_h$ does not occur, but $I = 0$  (\ie the parties that halt in the first round, output different values according the $\cP$ input.  Clearly $I = 0  \Longleftrightarrow E_h \lor  E_a$.

Consider the  adversary  that in the first round acts toward a random subset of $\cP$ according to input $\vv$ and towards the remaining parties according to  $\vv'$, and  aborts at the end of this  round. It is  clear that  if $E_a$ occurs, the  above adversary violates agreement with probability $1/2$. Thus, $\pr{E_a} \le 2\alpha$.

It is also clear that when $E_h$ occurs, the above attacker  fails to prevent an honest party
%parties
in $\oP$ from halting in the first round only if the following event happens:  each party in $\oP $ does not halt in $\Pi(\vv'')$ for some $\vv'' \in \set{\vv,\vv'}$, but the adversary acts towards each of these parties on the input in which it does halt. The latter event happen with probability at most  $2^{-\size{\oP}} \le \FirstErr = \err$. Thus, $\pr{E_h} \le 1 - \gamma - \err$. We conclude that
\begin{align*}
	\pr{I} \ge  1 - \pr{E_h} - \pr{E_a} \ge \gamma - \err  - 2\alpha.
\end{align*}
Finally, we note that if the protocol has public randomness, the (now rushing) attacker does not have to guess what input to act upon.
%on.
Rather, after seeing the first round randomness, it  \emph{finds}  an input $\vv'' \in\set{\vv,\vv'}$ such that at least one party in $\oP$ does not halt in $\Pi(\vv'')$ or violates agreement, and acts according to this input. Hence, the bound on $I$ changes to
\begin{align*}
\pr{I} \ge \gamma - \alpha,
\end{align*}
proving the theorem statement for such protocols.
\end{proof}

\newcommand{\DF}{D_\F}
\newcommand{\SDF}{\Supp(D_\F)}


\section{Lower Bounds on Second-Round Halting}\label{sec:SecondRound}
In this section, we prove lower bounds for second-round halting of Byzantine agreement protocols. In \cref{sec:TwoRoundProtcol:Arbitrary}, we prove a bound for arbitrary protocols, and in \cref{sec:TwoRoundProtcol:PR}, we give a much stronger bound for public-randomness protocols (the natural extension of public-coin protocols to the 'with-input' setting).

\subsection{Arbitrary Protocols}\label{sec:TwoRoundProtcol:Arbitrary}
%In this section, we prove
We start by proving our lower bound for second-round halting of arbitrary protocols.


\begin{theorem}[Bound on second-round halting, arbitrary protocols. \cref{thm:SecondRound:Arb} restated]\label{thm:SecondRound:Arb:Res}
\ThmSecondRoundArb
\end{theorem}

Let $\Pi$ be as in \cref{thm:SecondRound:Arb:Res}. We assume for ease of notation that an honest party that runs more than two rounds outputs $\perp$ (it will be clear that the attack, described below, does not benefit from this change). We also assume \wlg that the honest parties in an execution of $\Pi$ never halt in one round (by adding a dummy round if needed). Finally, we omit the security parameter from the parties' input list, it will be clear though that the adversaries we present are efficient \wrt the security parameter.

Let $k =\ceil{n/4}$ and let $h = \ceil{(n-k)/(t-k)}$. The theorem is easily implied by the next lemma.

\begin{lemma}[Neighboring executions]\label{lemma:SecondRound:Arb}
	Let $\vv,\vv' \in \zn$ be with $\ham(\vv,\vv') \le k$. Then, for every $\o\in \zo$:
\[
\pr{\Pi(\vv') = \o^n} \ge \pr{\Pi(\vv) = \o^n} - h(h+1)(2\alpha + 1-\gamma) - \alpha.
\]
%\rnote{$\o$ is too similar to $0$, maybe use a different notation\Inote{We also use it in the PR proof. Lets stick to that..}}
%Then $\eex{\cS \gets \bns}{I_{\vf,\cS}(i,b)} \ge 1 - \frac{i+1}{4h+8} - \chi$ for all $ i \in (h)$.
\end{lemma}
Namely, the lemma bounds from below the probability that in a random honest execution of the protocol on input $\vv'$ all parties halt within two rounds while outputting $\o$.


We prove \cref{lemma:SecondRound:Arb} below, but first use it to prove \cref{thm:SecondRound:Arb:Res}. We also make use of the following immediate observation.
\begin{claim}[Almost pre-agreement]\label{claim:SecondRound:Arb:Validity}
Let $\vv \in \zn$ and $\o \in \zo$ be such that $\ham(\vv,\o^n) \le t$. Then, $\pr{\Pi(\vv) = \o^n } \ge 1 - \alpha - \beta - (1- \gamma)$.
\end{claim}
\begin{proof}
The same argument as in the proof of \cref{claim:FirstRoundound:Arb:Validity} yields that
\begin{align*}
\pr{\Pi(\vv) \notin \omin^n} < \alpha + \beta
\end{align*}
Thus, by $\gamma$-second-round halting
\begin{align*}
\pr{\Pi(\vv) \ne \o^n} < \alpha + \beta + (1-\gamma)
\end{align*}
\end{proof}

\begin{proof}[Proof of \cref{thm:SecondRound:Arb:Res}]
Consider the vectors $\vz= 0^k 0^k 0^k 1^{n-3k}$, $\vo= 1^k 1^k 0^k 1^{n-3k}$ and $\vstar =1^k 0^k 0^k 1^{n-3k}$. Note that for both $\o\in \zo$ it holds that $\ham(\vv_b,\o^n)\le t$ since $n/4 \leq k\leq t$), and that $\ham(\vv_b,\vstar)=k$. Applying \cref{lemma:SecondRound:Arb,claim:SecondRound:Arb:Validity} for each of these vectors, yields that for both $\o\in \zo$:
\begin{align*}
\pr{\Pi(\vstar) = \o^n} &\ge 1- \alpha - \beta -(1- \gamma) - h(h+1)(2\alpha + 1-\gamma) -\alpha\\
&\ge 1- \beta - (h+1)^2(2\alpha + 1-\gamma).
\end{align*}
Note that $w=h+1$, which implies $\beta +w^2(2\alpha + 1-\gamma) \ge 1/2$, and the proof follows by a simple calculation.
\end{proof}

\subsubsection{Proving \cref{lemma:SecondRound:Arb}}
We assume for ease of notation that $\ham(\vv,\vv')=k$ (rather than $\le k$) and let $\ell=t-k$. Assume for ease of notation that $h\cdot \ell = n-k $ (\ie no rounding), and for a $k$-size
%$k$-sized
subset of parties $\cP \subset [n]$, let $ \cL^\cP_1,\dots,\cL_h^\cP$ be an arbitrary partition of $\oP = [n] \setminus \cP$ into $\ell$-size
%$\ell$-sized
subsets. Consider the following family of protocols:


\newcommand{\PP}{\Pi^\cP}

{ \samepage
\begin{protocol}[$\PP_{d}$]\label{prot:main:Arb} ~
\begin{description}
	\item [Parameters:] A subset $\cP\subseteq [n]$ and an index $d\in (h)$.
	\item [Input:] Every party $\Pc_i$ has an input bit $v_i\in\zo$.
	
	\item [First round:] ~
	\begin{description}
		\item[Party $\Pc_i \in \cP$.]
		If $d=0$ [\resp $d=h$], act honestly according to $\Pi$ \wrt input bit~$v_i$ [\resp $1-v_i$].
		Otherwise,		
		\begin{enumerate}
			\item Choose random coins honestly (\ie uniformly at random).
			
			\item To each party in $\bigcup_{j \in \set{1,\ldots,d}} \cL_j^\cP$: send a message according to input $1-v_i$.
			
			\item To each party in $\bigcup_{j \in \set{d+1 ,\ldots, h}} \cL_j^\cP$: send a message according to input $v_i$ (real input).
			
			\item Send no messages to the other parties in $\cP$.
		\end{enumerate}
		
		\item[Other parties.] Act according to $\Pi$.
	\end{description}
	
	\item [Second round:]~
	
	\begin{description}
		\item[Party $\Pc_i \in \cP$.]
        If $d=0$ [\resp $d=h$], act honestly according to $\Pi$ \wrt input bit~$v_i$ [\resp $1-v_i$]; otherwise, abort.
		
		\item[Other parties.] Act honestly according to $\Pi$.
	\end{description}
\end{description}
\end{protocol}
}

Namely, the ``pivot'' parties in $\cP$ gradually shift their inputs from their real input to its negation according to parameter $d$. Note that protocol $\PP_{0}(\vv)$ is equivalent to an honest execution of protocol $\Pi(\vv)$, and $\PP_{h}(\vv)$ is equivalent to an honest execution of $\Pi(\vv')$, for $\vv'$ being $\vv$ with the coordinates in $\cP$ negated. \cref{lemma:SecondRound:Arb} easily follows by the next claim about \cref{prot:main:Arb}. In the following we let $\delta = \pr{\Pi(\vv)_\oP = \o^{\size{\oP}}}$.

\newcommand{\ds}{{d^\ast}}
\begin{claim}\label{claim:SecondRound:Arb}
For any $k$-size
%$k$-sized
subset $\cP\subset [n]$ and $d\in (h)$ it holds that
\[
\pr{\PP_d(\vv)_\oP = \o^{\size{\oP}}} \ge \delta- d (h+1)(2\alpha + 1-\gamma)
\]
\end{claim}
We prove \cref{claim:SecondRound:Arb} below, but first use it to prove \cref{lemma:SecondRound:Arb}.
\begin{proof}[Proof of \cref{lemma:SecondRound:Arb}]
By \cref{claim:SecondRound:Arb},
\begin{align*}
\pr{\PP_h(\vv)_\oP = \o^{\size{\oP}}} \ge \delta - h(h+1)(2\alpha + 1-\gamma)
\end{align*}
Since $\PP_h(\vv)$ is just an honest execution of $\Pi(\vv')$, by agreement
\begin{align*}
\pr{\Pi(\vv') = \o^n} &\ge \delta - h(h+1)(2\alpha + 1-\gamma) - \alpha 
\end{align*}
\end{proof}




\begin{proof}[Proof of \cref{claim:SecondRound:Arb}]
The proof is by induction on $d$. The base case $d=0$ holds by definition. Suppose for contradiction the claim does not hold, and let $\ds\in(h-1)$
%$\ds\ge 0$
be such that the claim holds for $\ds$ but not for $\ds+1$. Let $\gamma_d$ be the probability that all honest parties halt in the second round of a random execution of $\PP_d(\vv)$. The assumption about $\ds$ yields that
\begin{align}\label{eq:main:Arb:01}
\pr{\PP_\ds(\vv)_\oP = \o^{\size{\oP}}} \ge \delta- \beta- \ds (h+1)(2\alpha + 1-\gamma)
\end{align}

and
\begin{align}\label{eq:main:Arb:02}
\pr{\PP_{\ds+1}(\vv)_\oP \in \zo^{\size{\oP}} \setminus \sset{\o^{\size{\oP}}}} > 1-\left(\delta - \beta - (\ds+1) (h+1)(2\alpha + 1-\gamma)\right)- (1-\gamma_d)
\end{align}


We note that for every $d\in (h)$
\begin{align}\label{eq:main:Arb:03}
\frac{1- \gamma_d}{h+1} \le 1-\gamma
\end{align}


Indeed, otherwise, the adversary that corrupts the parties in $\cP$ and acts like $\PP_d$ for a random $d\in (h)$, violates the $\gamma$-second-round-halting property of $\Pi$. We conclude that
\begin{align}\label{eq:main:Arb:1}
\lefteqn{\ppr{\vr}{\PP_\ds(\vv;\vr)_\oP = \o^{\size{\oP}} \qand \PP_{\ds+1}(\vv;\vr)_\oP \in \zo^{\size{\oP}} \setminus \sset{\o^{\size{\oP}}}}}\\
& \ge 1 - \left(1- \ppr{\vr}{\PP_\ds(\vv;\vr)_\oP = \o^{\size{\oP}}} \right) - \left(1- \ppr{\vr}{\PP_{\ds+1}(\vv;\vr)_\oP \in (\zo^{\size{\oP}} \setminus \sset{\o^{\size{\oP}}}}\right)\nonumber\\
& > (h+1) (2\alpha + 1 - \gamma) - (1-\gamma_d) \nonumber\\
& \ge (h+1) 2\alpha,\nonumber
\end{align}
for $\vr$ being the randomness of the parties. The first inequality is by \cref{eq:main:Arb:01,eq:main:Arb:02}, and the second one by \cref{eq:main:Arb:03}.

Consider the adversary that samples $d\gets (h)$, corrupts the parties in $\cP \cup \cL^\cP_{d+1}$, and acts towards a uniform random subset of the honest parties according to $\PP_d$ and to the remaining parties according to $\PP_{d+1}$. \cref{eq:main:Arb:1} yields that the above adversary causes disagreement with probability larger than $(h+1) 2\alpha/2(h+1) = \alpha$. Since it corrupts at most $t$ parties, this contradicts the assumption about $\Pi$.
%\rnote{does it need to be a random subset? \Inote{Yes, so with probability at least 1/2 two parties output different values} the size doesn't matter, right? \Inote{should be uniform}}
\end{proof}




\subsection{Public-Randomness Protocols}\label{sec:TwoRoundProtcol:PR}
%In this section, we
We proceed to prove our lower bound for second-round halting of public-randomness protocols.


\begin{theorem}[Lower bound on second-round halting, public-randomness protocols. \cref{thm:SecondRound:PR} restated] \label{thm:SecondRound:PR:Res}
	\ThmSeconRoundPR
\end{theorem}

Assume \cref{con:IsoBot} holds. Let $\Pi$ be as in the theorem statement, and assume $\gamma= \epg$ in the case $t \ge (1/3 +\ept)\cdot n$ and $\gamma= \half + \epg$ in the case $t \ge (1/4 +\ept)\cdot n$. Let $\lambda =\epg /10$ and $\sigma= \ept/4$. Recall that $\bot_\cS(\vx)$ is the string resulting by replacing all entries of $\vx$ indexed by $\cS$ with $\bot$.
\cref{con:IsoBot} yields that there exists $\delta>0$ such that the following holds for large enough $n$: let $\Sigma$ be a finite alphabet and let $\cA_0,\cA_1 \subset \sbn$ be two sets such that for both $b\in \zo$:
\begin{align*}%\label{eq:1}
\ppr{\cs\gets \bns}{\ppr{\vr \gets \sn}{\vr,\bot_{\cS}(\vr) \in \cA_b} \ge \lambda} \ge 1-\delta.
\end{align*}

Then,
\begin{align}\label{eq:IsoBot}
\ppr{\vr\gets \sn,\cS \gets \bns}{\forall b\in \zo\colon \set{\vr,\bot_{\cS}(\vr)} \cap \cA_b \neq \emptyset} \ge \delta.
\end{align}


In the following we assume $\alpha = \min\set{\delta\lambda \ept/10, \beta}$ and derive a contradiction, yielding that the agreement error has to be larger than that.


Fix $n$ that is large enough for \cref{eq:IsoBot} to hold and that $\ppr{\cs \gets \bns}{\size{\cs} > 2\sigma n} = 2^{- \Theta(n\cdot \sigma^2)} \le \alpha$, \ie $n> \Theta((\log 1/\alpha)/\sigma^2)$. As in the proof of \cref{thm:SecondRound:Arb:Res}, we assume for ease of notation that an honest party that runs more than two round outputs $\perp$, and that the honest parties in $\Pi$ never halt in one round. We also omit the security parameter from the parties input list. We assume \wlg that in the first round, the parties flip
%party flips
no coin, since such coins can be added to the setup parameter.

We use the following notation: the setup parameter and second-round randomness of the parties in $\Pi$ are identified with elements of $\cF$ and $\cR$, respectively. We denote by $f_i$ and $r_i$ the setup parameter and the second-round
%part
randomness of party $\Pc_i$ in $\Pi$, and let $\DF$ be the joint distribution of the parties' setup parameters (by definition, the joint distribution of the second-round randomness is the product distribution $\rn$). For $\vv \in \zn$, $\vf=(f_1,\ldots,f_n) \in \SDF$, and $\vr=(r_1,\ldots,r_n) \in\rn$, let $\Pi(\vv;(\vf,\vr))$ denote the execution of $\Pi$ in which party $\Pc_i$ gets input $v_i$, setup parameter $f_i$ and second-round randomness $r_i$. We naturally apply this notation for the variants of $\Pi$ considered in the proof.



For $\cs\subseteq[n]$, let $\Pi^\cs$ be the variant of $\Pi$ in which the parties in $\cs$ halt at the end of the first round. Let $k = \ceil{{t-\ept \cdot n}}$ (\ie $k = \ceil{n/3}$ if $t\ge (1/3 + \ept)\cdot n$, and $k = \ceil{n/4}$ if $t\ge (1/4 + \ept)\cdot n$). The heart of the proof lies in the following lemma.

\begin{lemma}[Neighboring executions]\label{lemma:SecondRound:PR}
Let $\vv,\vv' \in \zn$ be with $\ham(\vv,\vv') \le k$, let $\o\in \zo$, and let $\oS = [n] \setminus \cs$.  Then, with probability at least $\gamma- 7\lambda- \frac{\alpha + \pr{\Pi(\vv) \ne \o^n}}{\lambda}$ over $\vf \gets \DF$, it holds that
\begin{align*}
\ppr{\cs \gets \bns}{\ppr{\vr \gets \rn}{\Pi(\vv';(\vf,\vr)) = \o^n \qand \Pi^\cs(\vv';(\vf,\vr))_\oS = \o^{\size{\oS}} } \ge \lambda}\ge 1-\delta.
\end{align*}
\end{lemma}
Namely, in an execution of $\Pi(\vv')$, all honest parties halt after two rounds and output $\o$, regardless of whether a random subset of parties aborts after the first round. \cref{lemma:SecondRound:PR} is proven in \cref{sec:lemma:SecondRound:PR}, but let us first use it to prove \cref{thm:SecondRound:PR}. We make use of the following immediate observation:
\begin{claim}[Almost pre-agreement]\label{claim:SecondRound:PR:Validity}
Let $\vv \in \zn$ and $\o \in \zo$ be such that $\ham(\vv,\o^n) \le t$. Then, $\pr{\Pi(\vv) \in \set{\o,\perp}^n } \ge 1 - \alpha - \beta $.
\end{claim}
\begin{proof}
The proof of this claim uses an identical argument as in the proof of \cref{claim:FirstRoundound:Arb:Validity}.	
\remove{
Let $\cA \subset [n]$ be a subset of size $n-t$ such that $\vv_\cA = \o^{\size{\cA}}$. The claimed validity of $\Pi$ yields that
\begin{align*}
\ppr{\vr}{\Pi(\vv;(\vf,\vr))_\cA \notin\omin^{\size{\cA}}} < \beta_\vf(\vv).
\end{align*}
Hence	by the claimed agreement,
\begin{align*}
\ppr{\vr}{\Pi(\vv;(\vf,\vr)) \notin \omin^n} < \alpha_\vf(\vv) + \beta_\vf(\vv).
\end{align*}
}
%Finally by the claimed second0round halting
%\begin{align*}
%\ppr{\vr}{\Pi(\vv;(\vf,\vr)) \neq \o^n} < \alpha_\vf(\vv) + \beta_\vf(\vv) + (1-\beta_\vf(\vv)).
%\end{align*}
\end{proof}


\paragraph{Proving \cref{thm:SecondRound:PR}.}
\begin{proof}[Proof of \cref{thm:SecondRound:PR}]~
We separately prove the case $t \ge (1/3 + \ept)\cdot n$ and $t \ge (1/4 + \ept)\cdot n$.

\paragraph{The case $t \ge (1/3 + \ept)\cdot n$.}
Let $\vz= 0^k 1^{\cnk} 0^{\fnk} $ and let $\vo= 1^k 1^{\cnk} 0^{\fnk} $. Note that $\ham(\vz,\vo)=k$ and that for both $\o\in \zo$ it holds that $\ham(\vv_b,\o^n) \le t$. We will use \cref{lemma:SecondRound:PR,claim:SecondRound:PR:Validity} to prove that $\Pi(\vv_1) = 0^n$ with noticeable probability, contradicting the validity of the protocol.

Recall that, in this case, $\gamma = \epg$, that $\lambda = \epg/10$ and $\alpha,\beta \le \epg^2/200 = \lambda^2/2$.  \cref{claim:SecondRound:PR:Validity} yields that for both $\o\in\zo$:
\begin{align}\label{eq:SecondRound:PR:1}
\pr{\Pi(\vv_\o) \ne \oo^n }\ge \pr{\Pi(\vv_\o) \in \set{\o,\perp}^n} \ge 1- \alpha - \beta \ge 1- \lambda^2
\end{align}
Applying \cref{lemma:SecondRound:PR} \wrt $\vz$ and $\vo$ and $b=0$, yields that with probability at least
\[
\gamma- 7\lambda- \frac{\alpha + \pr{\Pi(\vz) \ne 0^n}}{\lambda} \geq 3\lambda- \lambda=2 \lambda
\]
over $\vf \gets \DF$, it holds that

\begin{align*}
\ppr{\vr}{\Pi(\vo;(\vf,\vr)) = 0^n}\ge \lambda
\end{align*}
and therefore
\begin{align*}
\pr{\Pi(\vo) = 0^n}\ge 2 \lambda^2
\end{align*}
in contradiction to \cref{eq:SecondRound:PR:1}.



\paragraph{The case $t \ge (1/4 + \ept)\cdot n$.} Consider the vectors $\vz= 0^k 0^k 0^k 1^{n-3k}$, $\vo= 1^k 1^k 0^k 1^{n-3k}$ and $\vstar =1^k 0^k 0^k 1^{n-3k}$. Note that for both $\o\in \zo$ it holds that $\ham(\vv_\o,\o^n) \le t$ and that $\ham(\vv_\o,\vstar)=k$. Applying \cref{lemma:SecondRound:PR,claim:SecondRound:PR:Validity} on $\vv_b$ and $\vstar$, for both $\o\in \zo$, yields that $\Pi^\cs(\vstar) = \o^n$ with noticeable probability over the choice of $\cs$. This will allow us to use \cref{con:IsoBot} to lowerbound the protocol's agreement.

Recall that, in this case, $\gamma=1/2+\eps_\gamma$. (Hence, $\lambda=20\gamma$.)
A similar calculation to the one in the previous case
%that done
%in the $t\ge (1/3 + \epg)\cdot n$ part
yields that by \cref{lemma:SecondRound:PR,claim:SecondRound:PR:Validity}, for both $\o\in \zo$: with probability at least $\frac12 + 2\lambda$ over $\vf \gets \DF $ it holds that
\begin{align*}
\ppr{\cs \gets \bns}{\ppr{\vr \gets \rn}{\Pi(\vstar;(\vf,\vr)) = \o^n \qand \Pi^\cs(\vstar;(\vf,\vr))_\oS = \o^{\size{\oS}} } \ge \lambda}\ge 1-\delta.
\end{align*}
It follows that there exists a set $\cT \subseteq \SDF$ with $\ppr{\vf \gets \DF}{\cT}\ge 4\lambda$, such that for every $\vf\in \cT$, for \emph{both} $\o\in \zo$:
\begin{align}\label{eq:mainPR:4}
\ppr{\cs \gets \bns}{\ppr{\vr \gets \rn}{\Pi(\vstar;(\vf,\vr)) = \o^n \qand \Pi^\cs(\vstar;(\vf,\vr))_\oS = \o^{\size{\oS}} } \ge \lambda}\ge 1-\delta
\end{align}

We assume \wlg that if a party gets $\perp$ as its second-round random coins, it aborts after the first round. For $\vr \in (\cR\cup\sset{\bot})^n$ let $ \cE(\vr)$ be the indices in $\vr$ of the value $\bot$.
%of the $\bot$'s in $\vr$.
For $\vf \in \SDF$ and $\o \in \zo$, let
\begin{align}
\cA^\vf_\o = \set{\vr \in \rbot \colon \Pi(\vstar;(\vf,\vr))_{\overline{\cE(\vr)}} = \o^{\size{\overline{\cE(\vr)}}}}
\end{align}
By \cref{eq:mainPR:4}, for $\vf \in \cT$ and $\o\in \zo$, it holds that
\begin{align}
\ppr{\cs\gets \bns}{\ppr{\vr \gets \rn}{\vr,\bot_{\cS}(\vr) \in \cA^\vf_\o} \ge\lambda} \ge 1-\delta
\end{align}
Hence by \cref{con:IsoBot}, see \cref{eq:IsoBot}, for $\vf \in \cT$ it holds that
\begin{align*}
\ppr{\vr\gets \rn,\cS \gets \bns}{\forall \o \in \zo\colon \set{\vr,\bot_{\cS}(\vr)}\cap \cA_b \neq \emptyset} > \delta.
\end{align*}
That is,
\begin{align}\label{eq:PR:3}
\ppr{\vr\gets \cR^n,\cS \gets \bns}{\forall \o\in \zo \quad \exists \cs_\o \in \set{\cs,\emptyset} \colon \Pi^{\cS_\o}(\vstar;(\vf,\vr))_{\overline{\cS_\o}}= \o^{ \size{\overline{\cS_\o}}}} >\delta
\end{align}

{\samepage
	\noindent
	Consider the following adversary:
	
	\begin{algorithm}[$\Ac$]~
		
		\begin{description}
			\item[Pre-interaction.]
			Corrupt a random subset $\cS \gets \bns $ conditioned on $\size{\cs} \le 2\sigma n$.
			
%			\rnote{why do we need to condition on $\size{\cs} \le 2\sigma n$? isn't it sufficient that it happens wp at least $1-\alpha$? \Inote{we could, but then it is not always a valid adversary (corrupts too many parties). I prefer to do it this way}}
			
			\item[First round.] Act according to $\Pi$.
			\item[Second round.] Sample $\cs_0,\cs_1$ at random from $\set{\emptyset,\cs}$, and act towards some honest parties according to $\Pi^{\cs_0}$ and towards the others according to $\Pi^{\cs_1}$ .
		\end{description}
	\end{algorithm}
}
Recall that $n$ is chosen so that $\ppr{\cs \gets \bns}{\size{\cs} >2\sigma n} \leq \alpha$ and that $\alpha < \delta/2$.
By \cref{eq:PR:3}, the above adversary violates the agreement of $\Pi$ on input $\vstar$ with probability larger than $\delta - \ppr{\cs \gets \bns}{\size{\cs} >2\sigma n} \ge \delta - \alpha >\alpha$, in contradiction with the assumed agreement of $\Pi$.
\end{proof}


\newcommand{\VV}{\cV^{\cP}}
\renewcommand{\PP}{\Pi^{\cP,\cS}}
\newcommand{\ops}{{\overline{\cP \cup \cS}}}

\subsubsection{Proving \cref{lemma:SecondRound:PR}}\label{sec:lemma:SecondRound:PR}
Fix $\vv,\vv \in \zn$ and $b\in \zo$ as in the lemma statement. We assume for simplicity that $\ham(\vv,\vv')=k$ (rather than $\le k$).
%(and not $\le k$).
Let $\ell= \floor{(t -k)/2}$ and let $h = \ceil{(n-k)/\ell}$. Assume for ease of notation that $h\cdot \ell = n-k $ (\ie no rounding), and for a $k$-size subset of parties $\cP \subset [n]$, let $ \cL^\cP_1,\dots,\cL_h^\cP$ be an arbitrary partition of $\oP = [n] \setminus \cP$ into $\ell$-size subsets. Consider the following protocol family.


{ \samepage
\begin{protocol}[$\PP_d$]\label{prot:main:2} ~
\begin{description}
    \item [Parameters:] subsets $\cP,\cS \subseteq [n]$ and an index $d\in (h)$.
    \item [Input:] Party $\Pc_i$ has a setup parameter $f_i$ and an input bit $v_i$.

	\item [First round:] ~
	\begin{description}
		\item[Party $\Pc_i \in \cP$.] ~
		If $d=0$ [\resp $d=h$], act honestly according to  $\Pi$ \wrt input bit $v_i$ [\resp $1-v_i$].
		Otherwise,
		\begin{enumerate}
			\item Choose random coins honestly (\ie uniformly at random).
			
			\item To each party in $\bigcup_{j \in \set{1,\ldots,d}} \cL_j^\cP$: send a message according to input $ 1-v_i$.
			
			\item To each party in $\bigcup_{j \in \set{d+1 ,\ldots, h}} \cL_j^\cP$: send a message according to input $v_i$ (real input).
			
			\item Send no messages to the other parties in $\cP$.
		\end{enumerate}
		
		\item[Other parties.] Act according to $\Pi$.
	\end{description}
	
	\item [Second round:]~
	\begin{description}
		\item[Parties in $\cP \setminus \cs$.] If $d=0$ [\resp $d=h$], act honestly according to $\Pi$ \wrt input bit $v_i$ [\resp $1-v_i$]; otherwise, abort.
		
		\item[Parties in $\cS$.] Abort.
		

		
		\item[Other parties.] Act according to $\Pi$.
	\end{description}
\end{description}
\end{protocol}
}
Namely, the ``pivot'' parties in $\cP$ shift their inputs from their real input to the flipped one according to parameter $d$. The ``aborting'' parties in $\cs$ abort at the end of the first round. Note that protocol $\PP_{0}$ is the same as protocol $\Pi^\cs$, and $\PP_h(\vv)$ acts like $\Pi^\cs(\vv')$, for $\vv'$ being $\vv$ with the coordinates in $\cP$ flipped.

For $\cP, \cS \subseteq [n]$, let $\ops = [n] \setminus (\cP \cup \cs)$, let $d\in (h)$, let $c\in \zo$, and let
\[
\VV_{d,c} = \set{(\vf,\cs,\vr) \colon \quad \PP_d(\vv;(\vf,\vr))_\ops = c^{\size{\ops}}}
\]
%Letting $\ops = [n] \setminus (\cP \cup \cs)$.
Namely, $\VV_{d,c}$ are the sets, setup parameters and random strings on which honest parties in $\PP_{d}$ halt in the second round and output $c$. Let $\chi = \pr{ \Pi(\vv) \ne \o^n}$ and let
\[
\cT_{d,c}^\cP = \set{\vf \colon \ppr{\cs \gets \bns}{\ppr{\vr \gets \rn}{(\vf,\cS,\vr),(\vf,\emptyset,\vr) \in \VV_{d,c} } \ge \lambda } \ge 1-\delta}
\]
The proof of  \cref{lemma:SecondRound:PR} immediately follows by the next lemma.
\begin{lemma}\label{lemma:SecondRound:PRR}
For every $k$-size subset $\cP \subset [n]$ and $d\in [h]$,
%\rnote{$d\in (h)$? \Inote{no. I dont prove it for $d=0$ and it is not needed for the proof of \cref{lemma:SecondRound:PR}}}
it holds that
\begin{align*}
\ppr{ \DF}{\cT_{d,\o}^\cP} \ge \gamma- 7\lambda- \frac{\chi + \alpha }{\lambda}
\end{align*}
%\rnote{what is $\pr{\cT_{d,\o}}$? is it $\pr{\cT_{d,\o}\neq\emptyset}$? \Inote{made explicit}}
\end{lemma}
\begin{proof}[Proof of \cref{lemma:SecondRound:PR}]
Immediate by \cref{lemma:SecondRound:PRR}.
\end{proof}
\newcommand{\tV}{\widetilde{\cV}}
\newcommand{\tT}{{\widetilde{\cT}}}


The rest of this subsection is devoted for proving \cref{lemma:SecondRound:PRR}. Fix a $k$-size subset $\cP \subset [n]$ and omit it from the notation when clear from the context.
Let
\[
\tV_{d,c} = \set{(\vf,\cs,\vr) \colon \forall a\in \zo \quad \PP_{d+a}(\vv;(\vf,\vr))_\ops = c^{\size{\ops}}}
\]
Namely, $\tV_{d,c} \subseteq \cV_{d,c}$ are the sets, setup parameters and random strings, on which honest parties in $\PP_{d+a}$ halt in the second round and output $c$, \emph{regardless} whether the parties in $\cS$ abort \emph{and} whether the parties in $\cP$ act toward those in $\cL_{d+1}$ according to input $0$ or $1$.
Let
\[
\tT_{d,c} = \set{\vf \colon \ppr{\cs \gets \bns}{\ppr{\vr \gets \rn}{(\vf,\cS,\vr),(\vf,\emptyset,\vr) \in \tV_{d,c} } \ge \lambda} \ge 1-\delta},
\]
let $\tT_{d} = \tT_{d,0}\cup \tT_{d,1}$, and let $\tT = \bigcap_{i \in (h-1)}\tT_{d}$. \cref{lemma:SecondRound:PRR} is proved via the following claims (the following probabilities are taken over $\vf \gets \DF$).


\begin{claim}\label{claim:PR:0}
	$\pr{\cT_{d+1,\o} \mid \tT} < \eta$ implies $\pr{\cT_{d,\oo} \mid \tT} \ge 1- \eta$.
\end{claim}
\begin{proof}[Proof of \cref{claim:PR:0}]
By definition, $\pr{\tT_{d+1} \mid \tT} =1$.	 Hence, $\pr{\tT_{d+1,\o} \mid \tT}< \eta$ implies $\pr{\tT_{d+1,\oo} \mid \tT}> 1 -\eta$, and the latter implies that $\pr{\cT_{d,\oo} \mid \tT}>  1 -\eta$.
\end{proof}


\begin{claim}\label{claim:PR:00}
$\pr{\tT} \ge \gamma- 6\lambda$.
\end{claim}


\begin{claim}\label{claim:PR:01}
$\pr{\cT_{1,\o} \mid \tT} \ge 1- (\chi + \alpha)/(\Pr[\tT]\cdot \lambda)$.
%$\pr{\cT_{1,\o} \mid \tT} \ge 1- \frac{\chi + \alpha }{\Pr[\tT]\cdot \lambda}$.
\end{claim}





\begin{claim}\label{claim:PR:1}
%$\pr{\cT_{d,0} \mid \tT} + \pr{\cT_{d,1} \mid \tT} \le 1 + \lambda/(h\cdot\Pr[\tT])$ for every $d \in [h-1]$.
For every $d \in [h-1]$.
\[
\pr{\cT_{d,0} \mid \tT} + \pr{\cT_{d,1} \mid \tT} \le 1 + \frac{\lambda}{h\cdot\Pr[\tT]}
\]
\end{claim}

We prove \cref{claim:PR:00,claim:PR:01,claim:PR:1} below, but first use the above claims for proving \cref{lemma:SecondRound:PR}.


\paragraph{Proving \cref{lemma:SecondRound:PRR}.}


\begin{proof}[Proof of \cref{lemma:SecondRound:PRR}.]
We first prove that for every $d\in [h]$:
\begin{align}\label{eq:SecondRound:PRR}
\pr{\cT_{d,\o} \mid \tT} \ge 1- \frac{ \chi + \alpha}{\Pr[\tT]\cdot \lambda} - \frac{d\lambda}{h\cdot\Pr[\tT]}
\end{align}	
The proof is by induction on $d$. The base case, $d=1$, is by \cref{claim:PR:01}. The induction steps follows by the combination of \cref{claim:PR:0,claim:PR:1}. Applying \cref{eq:SecondRound:PRR} for $d=h$, yields that
\begin{align*}
\pr{\cT_{h,\o} } \ge \Pr[\tT]- \frac{ \chi +\alpha}{\lambda} - \lambda,
%\pr{\cT_{h,\o} } \ge \pr{\tT}- \frac{\delta + \chi }{\lambda} - \lambda,
\end{align*}	
and the proof follows by \cref{claim:PR:00}.
\end{proof}


\newcommand{\PPP}{\Pi^{\cS}}


So it is left to prove \cref{claim:PR:00,claim:PR:01,claim:PR:1}. Note that the following adversaries corrupt at most $k + \ell + 2\sigma n \le t$ and thus they make a valid attack. Since our security model consider rushing adversaries, and $\Pi$ has
%is
pubic randomness, we assume the adversary knows $\vf = (f_1,\ldots,f_n)$ \emph{before} sending its first-round messages. In the following we let $\PPP_d = \PP_d$ and $\Pi_d = \Pi^{\emptyset}_d$.
\paragraph{Proving \cref{claim:PR:00}.}
This is the only part in proof where we exploit the fact that the protocol is secure against \emph{adaptive} adversaries.
\begin{proof}[Proof of \cref{claim:PR:00}.]
	

Consider the following \emph{rushing adaptive} adversary.

{ \samepage
\begin{algorithm}[$\Ac$]\label{alg:PR:00}~
		
\begin{description}
	
\item[Pre interaction:]~
Corrupt the parties in $\cP$.

\item[First round.] ~
Let $\vf$ be the parties' setup parameters.

Do $1/\lambda\delta$ times:
\begin{enumerate}
	\item  Sample  $\cS \gets \bns $ conditioned on $ \size{\cs} \le 2\sigma n$.
	\item  	For each   $i\in (h-1)$: estimate $\xi_i =  \ppr{\vr \gets \rn}{(\vf,\cs,\vr),(\vf,\emptyset,\vr)  \in \tV_{i}}$ by taking $\Theta( \log (h/ \lambda))$ samples of $\vr$.
	
	\item Let $d = \argmin_{i\in (h-1)} \set{\xi_i}$.
			
	\item If $\xi_d < 2\lambda$, break the loop.

\end{enumerate}
			
%		\item
        Corrupt the parties in $\cs \cup \cL_{d+1}$ ($\cs$ is the set sampled in the last loop), and act according to $\Pi_d$.
		
		\item[Second round.]~
		
		Let $\vr$ be the parties' second-round randomness.
		
		
		If $(\vf,\emptyset,\vr) \notin \cV_{d+a}$ for some $a\in \zo$,
		
		\quad act according to $\Pi_{d+a}$.
		
		Else,
		
		\quad Let $a\in \zo$ be such that $(\vf,\cs,\vr) \notin \cV_{d+a}$, set to $0$ if no such value exists.
		
		\quad Act according to $\Pi^{\cs}_{d+a}$.
			
	
\end{description}			
	\end{algorithm}
}

Let $D$ be the value of $d$ chosen by the adversary $\Ac$ (at the first round of the protocol).
Since $\ppr{\cs \gets \bns}{ \size{\cs} > 2\sigma n} \le \alpha < \delta/2$,
%\rnote{why? we know that $\alpha \leq \delta\lambda \ept/10$ \Inote{this is how we set $n$ and $\alpha$}}
if $\vf \notin \tT$ then except with probability $\lambda$ it holds that $\xi_D \le 2\lambda$. Where if $\xi_D \le 2\lambda$, then in the interaction with $\Ac$ the honest parties both halt in the second round and output the same value with probability at most $5\lambda$. Where since $\alpha < \lambda$, the honest parties halt in the second round of such interaction with probability smaller than $6\lambda$. We conclude that the honest parties halt in the second round under the above attack with probability smaller than $\Pr[\tT] + \Pr[\neg \tT]\cdot 6\lambda \le \Pr[\tT] + 6\lambda$, yielding that $\Pr[\tT] > \gamma- 6\lambda$.
\end{proof}




\paragraph{Proving \cref{claim:PR:01}.}


\newcommand{\oH}{{\overline{\cH}}}

\begin{proof}[Proof of \cref{claim:PR:01}.]
By definition, for $\vf \in \cT_{1,\o}$ it holds that
\begin{align*}
\ppr{\vr \gets \rn}{\Pi_1(\vv;(\vf,\vr))_\oH = \oo^{\size{\oH}} } = \ppr{\vr \gets \rn}{(\vf,\emptyset,\vr) \in \cV_{1,\oo} } \ge \lambda,
\end{align*}
letting $\cH = \cP \cup \cL_1$ and $\oH = [n] \setminus\cH$. Let $\eta = \ppr{\vf}{\cT_{1,\oo} \mid \tT}$, clearly,	$\ppr{\vf}{\cT_{1,\o} \mid \tT} = 1 -\eta$. By the above
\begin{align}
 \pr{\Pi_1(\vv)_\oH = \oo^{\size{\oH}}} \ge \Pr[\tT]\cdot \eta \cdot \lambda
\end{align}
(recall that $\Pi_1(\vv)$ stands for $\Pi_1(\vv;(\vf,\vr))$, for a random choice of $(\vf,\vr))$).
%\rnote{should we keep writing $\ppr{\vr \gets \rn}{\Pi_1(\vv;(\vf,\vr))_\oH = \oo^{\size{\oH}} }$? \Inote{No, $\vf$ is random now. See parenthesis above}}
Finally, we notice that
\begin{align}
\pr{\Pi_1(\vv) = \oo^{\size{\cH}}} + \pr{\Pi(\vv) = \o^n} \le 1+ \alpha
\end{align}

Otherwise, the adversary corrupting the parties in $\cH$, and acting toward the first honest parties according to $\Pi$ and toward the rest according to $\Pi_1$ violates the $\alpha$-agreement of $\Pi$. We conclude that $\Pr[\tT]\cdot \eta \cdot \lambda \le \chi + \alpha $, and therefore $\eta \le (\chi + \alpha)/(\Pr[\tT]\cdot \lambda)$.
%$\eta \le \frac{\chi + \alpha }{\Pr[\tT]\cdot \lambda}$.
 \end{proof}



\paragraph{Proving \cref{claim:PR:1}.}



The proof uses \cref{con:IsoBot} in a similar way to
%it is used in 	
the second part of the proof of the theorem.


\begin{proof}[Proof of \cref{claim:PR:1}.]
For $\vr \in (\cR\cup\sset{\bot})^n$ let $ \cE(\vr)$ be the indices in $\vr$ of the value $\bot$.
%of the $\bot$'s in $\vr$.
We assume \wlg that a party aborts upon getting $\perp$ as its second round random coins. For $\vf \in \SDF$, for $d\in [h-1]$, and for $\o \in \zo$, let
\begin{align}
\cA^\vf_\o = \set{\vr \in \rbot \colon \Pi_d(\vv;(\vf,\vr))_{\overline{\cP \cup \cL_d \cup \cE(\vr)}} = \o^{\size{\overline{\cP \cup \cL_d \cup \cE(\vr)}}}}
\end{align}
By definition, for $\vf \in \cT_{d,0} \cap \cT_{d,1}$ and $\o\in \zo$, it holds that
\begin{align}
\ppr{\cs\gets \bns}{\ppr{\vr \gets \rn}{\vr,\bot_{\cS}(\vr) \in \cA^\vf_\o} \ge \lambda} \ge 1-\delta
\end{align}
By \cref{con:IsoBot}, see \cref{eq:IsoBot}, for $\vf \in \cT_{d,0} \cap \cT_{d,1}$ it holds that
\begin{align*}%\label{eq:2}
\ppr{\vr\gets \rn,\cS \gets \bns}{\forall \o \in \zo \colon \set{\vr,\bot_{\cS}(\vr)}\cap \cA^f_b \neq \emptyset} > \delta
\end{align*}
That is,
\begin{align}\label{eq:PR:1:4}
\ppr{\vr\gets \cR^n,\cS \gets \bns}{\forall \o\in \zo \quad \exists \cs_\o \in \set{\cs,\emptyset} \colon \Pi^{\cS_\o}_{d}(\vv;(\vf,\vr))_{ \overline{\cP \cup \cL_d \cup\cS_\o} }= \o^{ \size{\overline{ \cP \cup \cL_d \cup\cS_\o}}}} >\delta
\end{align}
In pursuit of contradiction, assume that $\pr{\cT_{d,0} \mid \tT} + \pr{\cT_{d,1} \mid \tT} \ge 1 + \lambda/(h\cdot \Pr[\tT])$ for some $d \in [h-1]$. It follows that
\begin{align}\label{eq:PR:1:5}
\lefteqn{\ppr{\stackrel{\vf \gets \DF}{\vr\gets \cR^n,\cS \gets \bns}}{\forall \o\in \zo \quad \exists \cs_\o \in \set{\cs,\emptyset}\colon \Pi^{ \cs_\o}_{d}(\vv;(\vf,\vr))_{ \overline{\cP \cup \cL_d \cup \cs_\o} }= \o^{ \size{\overline{ \cP \cup \cL_d \cup \cs_\o}}}}}\\
&> \pr{\cT_{d,0} \cap \cT_{d,1}}\cdot \delta \hskip30em \nonumber\\
&\ge \Pr[\tT]\cdot \Pr[\cT_{d,0} \cap \cT_{d,1} \mid \tT] \cdot \delta \nonumber\\
&\ge \Pr[\tT]\cdot \frac{\lambda}{h\cdot \Pr[\tT]} \cdot \delta \nonumber\\
&= \lambda\delta /h \nonumber\\
&> 8\alpha.\nonumber
\end{align}
The first inequality is by \cref{eq:PR:1:4}, the second one by the assumption that $\Pr[\cT_{d,0} \mid \tT] + \Pr[\cT_{d,1} \mid \tT] \ge 1 + \lambda/(h\cdot \Pr[\tT])$, and the last one by the definition of $\alpha$.

{\samepage
\noindent
Consider the following rushing adversary:

\begin{algorithm}[$\Ac$]~

\begin{description}
\item[Pre-interaction.] ~
		
	\begin{enumerate}
		\item For each $i\in [h-1]$, estimate
		
		\[
        \xi_i = \ppr{\vr\gets \cR^n,\cS \gets \bns}{\forall \o\in \zo \quad \exists \cs_\o \in \set{\cs,\emptyset} \colon \Pi^{\cs_\o}_{d}(\vv;(\vf,\vr))_{\overline{ \cP \cup \cL_d \cup \cs_\o}}= \o^{ \size{\overline{ \cP \cup \cL_d \cup \cs_\o}}}}
         \]
         by taking $\Theta(\log (h/\alpha)/\alpha$ samples. Let $d = \argmax_{i\in [h-1]}\set{\xi_i}$.
		
		\item Sample a random $\cS \gets \bns $ conditioned on $\size{\cs} \le 2\sigma n$.
	\end{enumerate}
	
	 Corrupt the parties in $\cP \cup \cS\cup \cL_d$.
	
	\item[First round.] Act according to $\Pi_d$.
	
	\item[Second round.] Sample $\cs_0,\cs_1$ at random from $\set{\emptyset,\cs}$, and act towards some honest parties according to $\Pi^{\cs_0}_{d,0}$ and towards the others according to $\Pi^{\cs_1}_{d,1}$ .
\end{description}
\end{algorithm}
}
By \cref{eq:PR:1:5} and since $\ppr{\cs \gets \bns}{\cs \ge 2\sigma n} \le \alpha$, the above adversary violates the agreement of $\Pi$ on input $\vv$ with probability larger than $\alpha$, contradicting the assumed agreement of $\Pi$.
\end{proof}


\paragraph{Acknowledgements.}
We would like to thank Rotem Oshman, Juan Garay, Ehud Friedgut, and Elchanan Mossel for very helpful discussions.

\bibliographystyle{abbrvnat}
\bibliography{crypto}

\appendix
\section{Locally-Consistent Resilience Implies Malicious Resilience}\label{sec:local_const}
We next present a proof of \cref{thm:local_to_malicious}, but first some history \rnote{no need to write the history of VRF...} and definitions. Verifiable Random Functions (VRFs) were introduced by Micali, Rabin and Vadhan in \cite{MRV99}, who have also constructed a VRF based on the RSA assumption, although it was not efficient. Dodis and Yampolski \cite{DY05} presented an efficient VRF based on a decisional bilinear Diffie-Hellman inversion assumption. We give a definition of VRFs (\cite{MRV99,DY05}):

\newcommand{\gen}{\text{GEN}}
\newcommand{\pro}{\text{PROVE}}
\newcommand{\ver}{\text{VERIFY}}

\begin{definition}[Verifiable Random Function Family] \label{def:vrf}
	Let $k \in \N$, $p,q,s:\N \mapsto \N$, $\upsilon(k) > 0$. $F_{(\cdot)}(\cdot): \zo^{p(k)} \mapsto \zo^{q(k)}$ is an $\upsilon$-Verifiable Random Function Family (VRF) if there exists a triplet of algorithms $(\gen, \pro,\ver)$ such that $\gen(1^k) = (pk,sk)$, $\pro_{sk}(x) = (F_{sk}(x), \pi_{sk}(x))$, $\ver_{pk}(x,y,\pi) \zo$ and $\ver_{pk}(x,y,\pi) = 1$ if and only $F_{sk}(x) = y$ and $\pi$ is a proof of that statement. I.e, if the following three conditions hold:
	\begin{enumerate}
		\item $\forall (pk,sk) \in Im(\gen(1^k)), x, y_1, y_2$, if $F_{sk}(x) = y_1$ then there exists $\pi_1$ such that $\ver_{pk}(x,y_1,\pi_1) = 1$ and there does not exist $\pi_2$ such that $\ver_{pk}(x,y_2,\pi_2) = 1$.
		\item if $\pro_{sk}(x) = (y,\pi)$ then $\ver_{pk}(x,y,\pi) = 1$.
		\item No adversary executing more than $s(k)$ steps can win in the VRF game (defined below) with probability greater than $1/2 + \upsilon(k)$.
	\end{enumerate}
\end{definition}

\rnote{the definition above is somewhat messy and I think a bit buggy, please copy a good definition from one place and cite it}

To round out \cref{def:vrf} we give a definition of the VRF game:
\begin{definition}[VRF Game] \label{def:vrf_game}
The VRF game is defined by the following steps:
\begin{enumerate}
	\item A pair of keys $(pk,sk)$ is sampled from $\gen(1^k)$, and the adversary is given $pk$.
	\item The adversary is given oracle access to $\pro(\cdot)$, and it outputs a value $x$.
	\item Let $y_0 = F_{sk}(x), y_1 \leftarrow \zo^{q(k)}, b \leftarrow \zo$.
	\item The adversary is given $y_b$ along with oracle access to $\pro_{sk}(\cdot)$ and it outputs $b' \in \zo$.
	\item The adversary wins if and only if $b = b'$.
\end{enumerate}	
\end{definition}

\newcommand{\sig}{\text{SIGN}}

This completes the definition of VRFs. We will now formally define digital signature schemes:
\begin{definition}[Signature Scheme] \label{def:sig_sch}
For $k \in \N$ a triplet of PPTs $(\gen,\sig,\ver)$ with the following conditions
\begin{enumerate}
	\item $\gen(1^k) = (sk,vk) \in \zo^* \times \zo^*$.
	\item For $m \in \zo^k$, $\sig_{sk}(m) = \sigma \in \zo^*$.
	\item $\ver_{vk}(m,\sigma) \in \zo$.
\end{enumerate}	
is called a signature scheme if for all $(sk,vk) \in Supp(\gen(1^k))$, $m \in \zo^*$, then $\ver_{vk}(m,\sigma) = 1$ if and only if $\sigma \in Supp(\sig_{sk}(m))$.
\end{definition}
\rnote{where did you copy this definition from?}

We now present the notion for security for signature schemes:
\begin{definition}[Existential Unforgeability] \label{def:eu}
	For $\epsilon(k) > 0$ a signature scheme $(\gen,\sig,\ver)$ has $\epsilon$-existential unforgeability if for any PPT $\cA$ querying at most $p(k)$ queries, for some polynomial $p$ the following holds:
	$$\ppr{(sk,sv) \leftarrow \gen(1^k)}{\cA^{\sig_{sk}}(1^k,v) = (m,\sigma) \text{ s.t } \ver_{vk}(m,\sigma) = 1 \land \cA \text{ did not query } m } < \epsilon(k) $$
\end{definition}

We can now state a formal version of \cref{thm:local_to_malicious}. We first point out that it is well known that given PKI (along with some intractability assumptions) one can construct both existentially unforgeable signature schemes (\cite{RSA78, Merkle89, Rabin1979} and many more) and VRFs (\cite{DY05}). \rnote{Why do we need PKI to construct signatures/VRF?} We begin with a definition.
\newcommand{\euval}{(n-t)\cdot \epsilon(r)}
\newcommand{\vrfval}{\frac{t\cdot r}{n-t}\cdot \upsilon}
\newcommand{\rnd}{\Delta^{\epsilon,\upsilon}_{n,t,r}}

\begin{definition}[$\rnd$]
	In an $n$-party $t$-resilient Byzantine agreement executing for $r$ rounds where all parties have access to an $\epsilon$-EU signature oracle and an $\upsilon$-VRF we define $\rnd$ as:
	\[
	\rnd = \max(\euval, \vrfval)
	\]
\rnote{I don't understand what you want to define here... Why do we give signature oracle? We need to define the PKI for signatures and VRF}
\end{definition}

We present a compiler which receives as input a general Byzantine agreement protocol $\Pi$ which is secure against locally-consistent adversaries, along with a PKI for an existentially unforgeable digital signature scheme and a VRF and outputs $\Pi'$ which is a general Byzantine agreement protocol secure against malicious adversaries.

We give a shorter description \rnote{we already have a short description in the intro, let's move directly to the theorem and proof} of the compiler that is described in \cref{sec:intro:LocalToFull}. Given a Byzantine agreement $\Pi$ which is secure against $t$ locally-consistent adversaries we can construct $\Pi'$, a Byzantine agreement which is secure against $t$ malicious adversaries, in the following way: Party $\QParty_i \in \Pi'$ simulates an execution of party $\Party_i \in \Pi$. In round $r$ it computes $\pro_{sk_i}(i,r) = (y_{i,r},\pi_{i,r}) = (F_k(i,r), \pi_k(i,r)) $ and sets $\Party_i$'s random tape to be $y$. It then simulates $\Party_i$'s execution for that round. Next, it signs and sends messages generated by $\Party_i$ in that round and adds a proof in zero knowledge that (1) it acted consistently with randomness output by the VRF, some input bit, and verified messages it received in previous rounds, and (2) that all messages sent so far in all previous rounds are consistent \wrt the same input bit, randomness and incoming messages. (in short - that it acted consistently with its view) Please advide \cref{sec:intro:LocalToFull} for further details of this proof. If at round $r$ it receives a message from $\QParty_j$ that either (1) cannot be verified as being sent by $\QParty_j$, (2) has a false zero-knowledge proof, or (3) is inconsistent with either its randomness and/or its view as previously proved, then it ignores this message, sets $\Party_i$'s incoming message tape from $\Party_j$ to an empty string (i.e abort) and adds $\QParty_j$ to the set $\cC_i$ of corrupt parties. In subsequent rounds it ignores (\ie treats as aborted) all parties in $\cC_i$.

We can now state our theorem. We assume for the sake of simplicity that the zero-knowledge proofs above (except for the VRF proof) have perfect soundness. \rnote{it's easier to define the compiler in the ZK-hybrid model, and then simply replace it with a composable ZK protocol} This prevents extra complication (as well as another error term) in $\rnd$. This term is nullified if $\Pi$ is a public randomness protocol. We will also consider digital signature schemes as well as VRFs as being executed by calls to ideal functionalities.

\begin{theorem}[Locally consistent to malicious security, formal]\label{thm:local_to_malicious_formal}
	If $\Pi$ is an $n$-party protocol which is a $(t,\alpha,\beta,r,\gamma)$-\BA resilient against locally-consistent adversaries. Assuming that there exists PKI for $\epsilon$-EU digital signature schemes and $\upsilon$-VRF, and a $c>0$ round zero-knowledge proof of knowledge as mentioned above, then there exists an $n$-party protocol $\Pi'$ which is a $(t,\psi_1 = \alpha + \rnd,\psi_2 = \beta + \rnd,r \cdot c,\gamma)$-\BA resillient against malicious adversaries. Further, if $\Pi$ is a public randomness protocol, then $\Pi'$ is a $(t,\psi_1 = \alpha + \rnd,\psi_2 = \beta + \rnd,r + d,\gamma)$-\BA, for a constant $d > 0$.
\end{theorem}

\rnote{where did you define the compiler? What is the protocol $\Pi'$?} \rnote{in particular need to define the ZK -relation}
We will prove \cref{thm:local_to_malicious_formal} for the general case, as the case for public-randomness protocols easily follows. \cref{thm:local_to_malicious_formal} follows from the next two claims:

\begin{claim}\label{clm:locally_consistent_adversaries}
	Any adversary in an $r$-round execution of $\Pi'$ controlling no more than $t$ parties cannot break the security of either (1) the signature scheme or (2) the VRF with probability greater than $\rnd$.
\end{claim}

\begin{claim}\label{clm:locally_consistent_view}
	Let $\QParty_i$ be an honest party in an execution of $\Pi'$ with $\Party_i$ being the party in $\Pi$ it simulates. If an adversary controlling no more than $t$ parties in $\Pi'$ did not break the security of either (1) the signature scheme, or (2) the VRF then $\Party_i$' s view is consistent with an execution of $\Pi$ with no more than $t$ locally-consistent corrupted parties.
\end{claim}

We first prove \cref{thm:local_to_malicious_formal} using \cref{clm:locally_consistent_adversaries,clm:locally_consistent_view}.

\begin{proof}[Proof of \cref{thm:local_to_malicious_formal}.]
Assume by way of contradiction that there exists an adversary $\cA$ controlling no more than $t$ parties in $\Pi'$ such that it causes the honest parties in $\Pi'$ to output disagreing values with probability greater than  $\alpha + \rnd$ (The converse for validity is symmetric and will therefore be ommitted from the proof).  Let $\cT$ be the event that two honest parties interacting with $\cA$ in an execution of $\Pi'$ output differing values. By assumption we have
\begin{align*}
 	\pr{\cT} > \alpha + \rnd.
\end{align*}
Let $\cE$ be the event that $\cA$ either (1) forged an honest party's signature, or (2) broke the security of the VRF in an execution of $\Pi'$. Note that by \cref{clm:locally_consistent_adversaries}
\begin{align*}
	\pr{\cE} < \rnd.
\end{align*}
Thus,
\begin{align*}
	\pr{\cT \land \cE} > \alpha.
\end{align*}
By \cref{clm:locally_consistent_view} in an execution of $\Pi'$ in which $\cE$ did not occur, the honest parties' simulated parties see a view consistent with an execution of $\Pi$ with no more than $t$ locally-consistent adversaries. Thus, since honest parties in $\Pi'$ output the same output as their simulated counterparts in an execution of $\Pi$, and the view of all simulated parties in $\Pi'$ is that of an execution in which they face no more than $t$ locally-consistent adversaries, we have that
\begin{align*}
\pr{\cT \land \cE} < \alpha
\end{align*}
leading to a contradiction.
\end{proof}

We can now prove \cref{clm:locally_consistent_adversaries,clm:locally_consistent_view}. To prove \cref{clm:locally_consistent_adversaries} we prove two sub-claims.

\begin{claim}\label{clm:locally_consistent_eu}
	Let $\cE_1$ be the event that in an $r$-round execution of $\Pi'$ with adversary $\cA$ controlling no more than $t$ parties, $\cA$ managed to forge an honest party's signature. Then
	\begin{align}\label{eq:locally_consistent_eu}
		\pr{\cE_1} < \euval.
	\end{align}
\end{claim}

\begin{claim}\label{clm:locally_consistent_vrf}
	Let $\cE_2$ be the event that in an $r$-round execution of $\Pi'$ with adversary $\cA$ controlling no more than $t$ parties, $\cA$ managed to produce $\pi$, a false proof accepted by an honest party. Then
	\begin{align}\label{eq:locally_consistent_vrf}
	\pr{\cE_2} < \vrfval.
	\end{align}
\end{claim}

These two claims are enough to prove \cref{clm:locally_consistent_adversaries} as we can simply note that $\cE \equiv \cE_1 \lor \cE_2$. We prove them next.
\begin{proof}[Proof of \cref{clm:locally_consistent_eu}.]
	Assume there exists an adversary $\cA$ for which \cref{eq:locally_consistent_eu} does not hold. Consider the following adversary $\cB$ to the signature scheme:
	{\samepage
		\begin{description}
		\item[$\cB^{\sig}$.]		
		\begin{enumerate}
			\item Simulate an execution of $\Pi'$ with adversary $\cA$.
			
			\item Let $\cP_\cA$ be the set of parties adversary $\cA$ has chosen to corrupt. Sample party ${\cQ \leftarrow \cP \setminus \cP_\cA}$ uniformly at random.
			
			\item In each round of $\Pi'$, if $\cQ$ queries its signing oracle with $q$, then query $\sig(q)$ and respond to $\cQ$ with the result of that query. For other parties $\cB$ simply simulates a consistent view of an oracle for a signature scheme and an oracle for a VRF.
			
			\item If at any round $\cA$ forged a signature $\sigma$ on a message $m$ by $\cQ$ then output $(m,\sigma)$.
			
			\item else, output a random message $m$ and random value $\tilde{\sigma}$ from the image of \sig.
		\end{enumerate}
	\end{description}
}

Let $\cF$ be the event that $\cB$ forged a valid signature and let $\cG$ be the event that $\cA \text{ forged a signature by } \cQ \text{ and } \cQ \text{ was selected by } \cB$. Notice first that the simulated $\cA$'s view in $\cB$'s execution is clearly consistent with that of an $r$-round execution of $\Pi'$, as all oracle queries are answered consistently according to step (3). Thus,
\begin{align*}
\pr{\cF} &= \pr{\cG}\\&+ \pr{\neg \cG \land \cB \text{ randomly sampled a valid message and signature pair in step 5 }}\\
&>\pr{\cG} \ge \pr{\cA \text{ forged a signature by } \cQ} \cdot \frac{1}{n-t} \ge \epsilon(r)
\end{align*}
where the final two inequalities are by $\cA$ corrupting at most $t$ parties and the assumption over $\cA$'s forging probability, respectively. Thus, $\cB$ manages to forge a signature with $r$ queries with probability greater than $\epsilon(r)$, in contradiction to the assumption over the signature scheme.
\end{proof}

\begin{proof}[Proof of \cref{clm:locally_consistent_vrf}.]
	Assume there exists an adversary $\cA$ for which \cref{eq:locally_consistent_vrf} does not hold. Consider the following adversary $\cB$ in the role of prover $P$ in communication with verifier $V$ for the VRF.
	{\samepage
			\begin{description}
		\item[$<\cB, V>$]		
		\begin{enumerate}
			\item Simulate an execution of $\Pi'$ with adversary $\cA$.
			
			\item Let $\cP_\cA$ be the set of parties adversary $\cA$ has chosen to corrupt. Sample party ${\cQ_p \leftarrow \cP_\cA}$ and party $\cQ_v \leftarrow \cP \setminus \cP_\cA$ uniformly at random. Also sample $d \leftarrow [r]$ uniformly at random.
			
			\item In each round of $\Pi'$ $\cB$ provides a consistent view of a signing oracle to all parties. At round $d$ it performes the following in the proving phase:
			
			\item Let $m, \pi^{p,v}_d$ be the message, proof pair sent from party $\cQ_p$ to party $\cQ_v$ - $\cB$ sends $m, \pi^{p,v}_d$ to $V$.
			
			\item $\cB$ then halts.
		\end{enumerate}
	\end{description}
}

Let $\cD$ be the event that $\cB$ managed to make $V$ accepts a non-valid proof. By a similar reasoning to the one in the proof of \cref{clm:locally_consistent_eu} we have our claim, as the proofs of the VRF are independent between rounds and between parties (\ie they stand on their own without context). Thus, the view of $V$ is that of an execution of the proving part of the VRF protocol with prover $\cQ_p$, and the view of $\cA$ is consistent with an execution of $\Pi'$.
\end{proof}

Finally, a proof of \cref{clm:locally_consistent_view} will give us \cref{thm:local_to_malicious_formal}.

\begin{proof}[Proof of \cref{clm:locally_consistent_view}.]
	
	First, recall the definition of a locally-consistent adversary. A locally consistent adversary is one that can (1) abort prematurely, (2) send messages based on differing input bits and messages from honest parties, (3) sample their random coins honestly, and (4) cannot lie about messages received from honest parties. Given this, the proof of this claim is fairly simple, as in an execution of $\Pi'$ in which no adversary has forged an honest party's signature and all proofs were accepted, all honest parties have received messages consistent with other honest parties' messages in the execution, and as no adversary has managed to provide a false proof of consistency then it either was flagged as being aborted (which, again, it can do as a locally-consistent adversary) by an honest party $\cQ$ or has acted honestly \wrt its randomness. Thus, the view of parties in $\Pi$ simulated in $\Pi'$ is consistent with an execution of $\Pi$ in the presence of a locally-consistent adversary controlling no more than $t$ parties.
\end{proof}
	


\end{document}

