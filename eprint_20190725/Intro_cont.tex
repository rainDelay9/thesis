\subsection{Locally Consistent Security to Malicious Security}\label{sec:intro:LocalToFull}

As briefly mentioned in \cref{sec:intro:model}, protocols that are secure against locally consistent adversaries can be compiled to tolerate arbitrary malicious adversaries.
The compiler requires a PKI for digital signatures and verifiable random functions (VRFs)~\cite{MRV99}. A VRF is a pseudorandom function with an additional property: using the secret key and an input $x$, the VRF outputs a pseudorandom value $y$ along with a proof string $\pi$; using the public key, everyone can use $\pi$ to verify whether $y$ is the output of $x$. We consider a trusted setup phase for establishing the PKI, where every party generates keys for a VRF and for a signature scheme, and publishes the corresponding public keys.

Given a protocol that is secure against locally consistent adversaries, the compiled protocol proceeds as follows, round by round.
Each party $\Party_i$ sets its random coins for the \rth round $\rho_i^r$ (together with a proof $\pi_i^r$) by evaluating the VRF over the pair $(i,r)$.
Next, for every $j\in[n]$, party $\Party_i$ uses these coins to compute the message $m^r_{i\to j}$ for $\Party_j$, signs $m^r_{i\to j}$ along with the VRF proof $\pi^r_i$ as $\sigma^r_{i\to j}$, and sends $(m^r_{i\to j},\pi_i^r,\sigma^r_{i\to j})$ to $\Party_j$.
Finally, $\Party_i$ proves to each $\Party_j$ using a zero-knowledge proof of knowledge that:
\begin{enumerate}
    \item
    There exist an input bit $b$, random coins $\rho_i^r$, as well as random coins and incoming messages $\rho^{r'}_i$ and $(m^{r'}_{1\to i},\ldots,m^{r'}_{n\to i})$ for every $r'<r$, such that: (1) $\pi_i^r$ verifies that $\rho_i^r$ is the VRF output of $(i,r)$ (using the VRF public key of $\Party_i$), and (2) the message $m^r_{i\to j}$ is the output of the next-message function of $\Party_i$ when applied to these values.
    \item
    For every $r'<r$, the input bit $b$ and the random coins $\rho^{r''}_i$ and incoming messages $(m^{r''}_{1\to i},\ldots,m^{r''}_{n\to i})$ for every $r''<r'$, are the same as those used to generate $m^{r'}_{i\to j}$.
    \item
    For $r>1$, the messages received in the previous round are properly signed. That is, for every $k\in[n]$, there is a signature $\sigma^{r-1}_{k\to i}$ of the message $m^{r-1}_{k\to i}$ that verifies under the signature-verification key of $\Party_k$.
\end{enumerate}

When considering public-randomness protocols, the above compilation can be made much more efficient. Instead of proving in zero knowledge the consistency of each message, each party $\Party_i$ concatenates to each message all of its incoming messages from the previous round. A receiver can now locally verify the coins used by $\Party_i$ are the VRF output of $(i,r)$ (as assured by the VRF), that the incoming messages are properly signed, and that the message is correctly generated from the internal state of $\Party_i$ (which is now visible and verified).

%\rnote{the VRF keys should be honestly generated, or alternatively we need a common random string that is independent of the PKI and is used as an additional seed for the VRF}
\begin{theorem}[Locally consistent to malicious security, folklore, informal]\label{thm:local_to_malicious}
Assume PKI for digital signatures and VRF, then a BA protocol secure against locally consistent adversaries, can be compiled into a maliciously secure BA protocol with the same parameters, apart from a constant blowup in the round complexity (no blowup for public-randomness protocols).
\end{theorem}

\Inote{prove in the appendix}
\Inote{when proved, give reference to the section }

\subsection{Additional Related Work}\label{sec:relatedWork}

Following the work of \citet{FM97} in the two-thirds majority setting, \citet{KK06} improved the expected round complexity to $23$, and \citet{Micali17} to $9$. In the honest-majority setting, \citet{FG03} showed expected-constant-round protocol and \citet{KK06} expected $56$ rounds. \citet{MV17} adjusted the technique from \cite{Micali17} to the honest-majority case. \citet{ADDNR19} achieved expected $10$ rounds assuming static corruptions and expected $16$ rounds assuming adaptive corruptions. \citet{ACDNPRS19} constructed an expected-constant-round protocol tolerating $(1/2-\epsilon)\cdot n$ adaptive corruptions with sublinear communication complexity. In the dishonest-majority setting, \citet{GKKO07} constructed a broadcast protocol with expected $O(k)$ rounds, tolerating $t<n/2+k$ corruptions.

%\rnote{add more asynchronous BA references. In the asynchronous setting, \citet{CR93} gave the first analogue result to \cite{FM97}.}

\citet{AH10} extended the results of \citet{CMS89} and of \citet{KY86} on guaranteed termination of randomized BA protocols to the asynchronous setting, and provided a tight lower bound.

Randomized protocols with expected constant round complexity have \emph{probabilistic termination}, which requires delicate care \wrt composition (\ie their usage as subroutines by higher-level protocols). Parallel composition of randomized BA protocols was analyzed in \cite{Ben-Or83,FG03}, sequential composition in \cite{LLR06}, and universal composition in \cite{CCGZ16,CCGZ17}.
